\documentclass[
    % Options for apa7
    a4paper,                % Paper size
    11pt,                   % Font size
    stu,                    % Format as assignment
    donotrepeattitle,       % Start body text without repeating title
    floatsintext,           % Insert tables and figures with texts
    biblatex,               % Use BibLaTeX for references
    % Options for hyperref
    colorlinks=true,        % Colour all links
    linkcolor=red,          % Cross-references in red
    anchorcolor=black,      % Keep anchors black
    citecolor=blue,         % In-text-referencs in blue
    urlcolor=blue,          % DOIs and URLs are in blue
    bookmarks=true,         % Generate bookmarks for PDF readers
    bookmarksopen=false,    % Expand all bookmarks as default
    bookmarksnumbered=true, % Keep section number in bookmarks
    % Options for xcolor
    dvipsnames              % Use colour BrickRed and PineGreen
]{apa7}

% Specify absolute path to the tt.sty file, depending on the operating system
\usepackage{ifplatform}
\ifwindows
    \usepackage{M:/pc/Dokumenter/tt}
\else
    \usepackage{/home/tony/uio/pc/Dokumenter/tt} % Must not use ~ for home directory. Spell in full.
\fi

\title{Evaluating Norway's 2020 Curriculum Reform using PISA Data}
\authorsnames{Tony C. A. Tan}
\authorsaffiliations{{Department of Teacher Education and School Research, University of Oslo}}
\course{Cover Letter}
\professor{Dr Andreas Pettersen \& Prof Lovisa Sumpter}
\duedate{6 July 2022}

\begin{document}
\setcounter{page}{0}
\maketitle

\begin{flushright}
    Wednesday 6 July 2022
\end{flushright}

\noindent Dear Selection Committee Members,

\bigskip

I am excited to present myself for ILS's PhD research fellow in mathematics education, Position Number 227662.

Being able to work with PISA data is my chief motivation behind this application. I have benefited from my master thesis with in-depth understanding of PISA's data structure and sampling procedure. I have also acquired statistical modelling (e.g., item response theory, multilevel models) and the computer software skills (e.g., R and Mplus) for carrying out empirical research projects using PISA datasets. Additionally, I trust that my mathematics teaching experience from Australia would also inform and contribute to this PhD project through my first-hand appreciation of the interplay between curriculum, pedagogy, and assessment.

ILS's research environment is another motivator for my application. Since 2020, I have had the privilege to serve as a research assistant for the school environment systematic review led by Prof Trude Nilsen, and for Dr Nani Teig's science teaching and learning project as well as TIMSS data preparation and management. I am currently associated with ILS's ongoing project that estimates the impact of the COVID pandemic led by Prof Nilsen, and a systematic review into the relationship between socio-economic status and learning outcome led by Oleksandra Mittal. I believe my familiarity with the organisational structure and strong work relationship with ILS research teams will greatly assist my transition into this PhD position.

I grew up with education and work experience with heavy quantitative emphases. My econometrics training has equipped me with strong mathematical and theoretical foundation for regression analyses. I then acquired professional skills in managing complex data sources working as a taxation accountant. I put these number skills to my subsequent teaching and CEMO degrees and delivered an A-grade master thesis applying multilevel structural equation models to 2018 PISA data. Recently, I have been working with Norwegian national register data under the TSD environment and used item response theory to examine Norwegian teachers' grading practices. I trust my research interests and prior experience in regression analyses, SEM and IRT will all serve this PhD project's objectives well.

I consider myself a structured, efficient, and hard-working team player who also function well independently. I am a qualified teacher in both Australia and Norway, and I master several languages. I was invited by the EngageLab during my master's degree to participate in the Student Innovation project, leading to subsequent scholarship and prototype development. In addition, I was a member of CEMO's Admission Committee and Student Board. In my spare time, I work as the elected treasurer for Oslostudentenes Idrettsklubb Svømming.

Should you desire any additional information, please do not hesitate to contact me by phone 92222056 or by email \href{tctan@uio.no}{tctan@uio.no}.

\bigskip

\noindent I look forward to hearing from you soon.

\bigskip

\noindent  Warm regards,

\noindent \includegraphics[scale=0.15]{M:/pc/Dokumenter/signature.jpg}

\noindent Tony Tan

\end{document}
