\documentclass[
    % Options for apa7
    a4paper,                % Paper size
    11pt,                   % Font size
    stu,                    % Format as assignment
    donotrepeattitle,       % Start body text without repeating title
    floatsintext,           % Insert tables and figures with texts
    biblatex,               % Use BibLaTeX for references
    % Options for hyperref
    colorlinks=true,        % Colour all links
    linkcolor=red,          % Cross-references in red
    anchorcolor=black,      % Keep anchors black
    citecolor=blue,         % In-text-referencs in blue
    urlcolor=blue,          % DOIs and URLs are in blue
    bookmarks=true,         % Generate bookmarks for PDF readers
    bookmarksopen=false,    % Expand all bookmarks as default
    bookmarksnumbered=true, % Keep section number in bookmarks
    % Options for xcolor
    dvipsnames              % Use colour BrickRed and PineGreen
]{apa7}

% Specify absolute path to the tt.sty file, depending on the operating system
\usepackage{ifplatform}
\ifwindows
    \usepackage{M:/pc/Dokumenter/tt}
\else
    \usepackage{/home/tony/uio/pc/Dokumenter/tt} % Must not use ~ for home directory. Spell in full.
\fi

\title{Evaluating Norway's 2020 Curriculum Reform using PISA Data}
\authorsnames{Tony C. A. Tan}
\authorsaffiliations{{Department of Teacher Education and School Research, University of Oslo}}
\course{Project Proposal}
\professor{Dr Andreas Pettersen \& Prof Lovisa Sumpter}
\duedate{6 July 2022}

\begin{document}
\maketitle

\section{Introduction and Rationale}

The curriculum revision (\textit{fagfornyelsen}) in 2020 (K20) marks a major change in how students are taught in Norway \parencite{udir:2020}. It was the first time a substantial reformation of curricula was implemented since the 2006 reform (\textit{kunnskapsl{\o}ftet} 2006, K06). In mathematics, one major change was the establishment of core elements (\textit{kjerneelementer}) across all curricula spanning from Year 1 to 10. These core elements, namely, \textit{inquiry and problem-solving}, \textit{modelling and applications}, \textit{reasoning and argumentation}, \textit{representation and communication}, \textit{abstraction and generalisations}, and \textit{mathematical domains}, to a large degree resemble the PISA mathematics framework \parencite{oecd:2018}---both share genesis with the eight competencies firstly proposed by Danish mathematician and educator Mogens Niss \parencite{niss:2003,niss:2011,niss:2019}.

The implementation of the core elements in K20 and their close alignment with the PISA framework provides a golden opportunity to study Norwegian students' learning outcomes using one cycle of PISA \emph{before}, and one \emph{after}, the introduction of K20. Since mathematics was the major domain in PISA 2012 and once again in 2022, these two time points may serve as the pre-test and post-test in an ``experiment'' with K20 being the ``treatment'' \parencite{shadish:2002}. Two factors, however, complicate this quasi-experimental interpretation. First, K20 was implemented \emph{concurrently} with the COVID-19 school closures and the resultant home schooling. Separating the effects attributable to the pandemic from those of K20 is therefore one chief task in this PhD project. Second, PISA employs a cross-sectional rather than longitudinal design, limiting any causal inferences. Yet, PISA data sets, and especially when combined with Norwegian register data, are the best data sources available in Norway to study the effect of K20.

Mapping students' knowledge, understanding and skills within these core elements, in particular, \emph{problem-solving}, \emph{modelling} and \emph{reasoning}, is important for students, teachers, and curriculum evaluation purposes. First of all, an in-depth understanding of students' mastery of these key capabilities would provide insight into their command of 21st Century skills \parencite[][p. 31]{oecd:2018}. Secondly, teachers may also benefit from a clearer understanding of competency demand \parencite{pettersen:2018}, a pedagogical factor shown to be associated with learners' outcomes \parencite{pettersen:2019}. Lastly, since its initial introduction into the Norwegian school curriculum in K06, \emph{competencies} and the ability to communicate one's knowledge and skills in a context, have been further elevated towards the work of Niss in K20. It is therefore important to examine the consequences of these policy shifts for students and to build on previous examinations of competency-based pedagogy \parencite[e.g., ][]{pettersen:2018}.

%//mark Here you could write about how nobody has used the golden opportunity to examine K20 in light of PISA data before and after the introduction of K20.

International large-scale assessments have great potential to influence and benefit education policies \parencite{nortvedt:2018}. Two years past its introduction, it remains open whether the K20 curriculum reform has made any difference in students' overall mathematics competence, if so in which competency domain(s), and to which cohort of students. The availability of PISA 2012 and 2022 data presents a unique opportunity to evaluate the impact of Norway's K20 and to inform future reforms.

\section{Overall Aim and Research Questions}

In this PhD, I wish to address the open questions mentioned above through this overarching aim: To examine students' competencies within the core elements of K20 with specific focus on \textit{problem-solving}, \textit{modelling} and \textit{reasoning} using primarily PISA 2012 and 2022 data.

This aim can be operationalised through the following research questions:
\begin{APAenumerate}
    \item What is the alignment between (a) PISA 2012 and 2022 mathematics framework and (b) the mathematics curriculum of K20?
    \item How have students' competencies in the core elements of K20, especially \emph{problem-solving}, \emph{modelling} and \emph{reasoning}, changed from  2012 to 2022, and what characterises these changes?
    \item Who were the most impacted by the changes in students' achievement resultant from K20, having controlled for the pandemic effects?
\end{APAenumerate}

\section{Theoretical Framework}

%//mark This section starts by providing an overview of the Norwegian education system, and curricular changes.
%//mark It then provides a brief overview of K20.
%//mark A theoretical framework for mathematics competencies in general, with specific overviews of previous research on:
%//mark - problem-solving
%//mark - Modelling, and
%//mark - Reasoning
%//mark constitutes the last and main part of the theoretical framework.

\subsection{The Norwegian Education System}

Norway follows the Nordic education model characterised by qualities such as social justice, equity, equal opportunities, inclusion, education for all, and nation building \parencite{imsen:2017}. Young Norwegians are expected to complete seven years of primary schools (\textit{barneskole}) followed by three years of lower secondary schools (\textit{ungdomsskole}), typically reaching 15 years of age upon completion. Students may then choose to either continue academic pathways through upper secondary schools (\textit{videreg{\aa}ende skole}) in order to enter universities, or to undertake vocational education in training schools (\textit{fagskole}). PISA's targeted population of 15-year-old learners happens to match young Norwegians' completion of their lower secondary schooling, making comparison with register data particularly desirable.

\subsection{Norway's Recent Curricular Changes}

Over the decades, Norway implemented multiple rounds of reforms to school curricula. Since the 1997 expansion of compulsory education into ten years, the old-school knowledge-based practices (e.g., \textit{puggeskole}, learning by heart) have given way to a more competency-based curriculum where \emph{communication} of knowledge and skills, as well as \emph{applying} them in different contexts became important \parencite{imsen:2017}. The major reform of K06 firmly established \emph{competencies} at the centre of modern-day curriculum design, a trend to be further affirmed by the recent K20.

\subsection{Mathematical Competencies}
%//mark This section should be the main section with the theory section, and take up more space than the Norwegian curricular stuff
%//mark About the framework of mathematical competencies, use Niss (2003)

Norway's mathematics curricula have experienced increasing convergence with the theory of mathematical competencies resultant from Denmark's KOM project \parencite{niss:2003,niss:2011,niss:2019}. Under this framework, a mathematical competenc\emph{y} is one constituent of the overarching mathematical competenc\emph{e}. \textcite{niss:2019} conceptualise \emph{mathematical competence} as ``someone's insightful readiness to act appropriately in response to all kinds of \emph{mathematical} challenges pertaining to given situations.'' (p. 12, emphasis in original), where ``readiness'' narrowly refers to an individual's \emph{cognitive} prerequisites for engaging in certain activities in contrast to \emph{disposition} which covers their affects, attitudes and will power when carrying out such activities. More specifically, competence has three main characteristics, with the first being oriented towards action---not only physical and mental activations, including decision-making, can be thought as one's readiness to act, but also their conscious and explicit decisions to refrain from undertaking particular actions in a given situation. Secondly, neither acting without insight nor being merely insightful is considered as an instance of competence. Thirdly, ``meeting the challenges'' must always be understood in the duality between subjective and socio-cultural aspects as to whom the judges are in deriving meaning and legitimacy to the actions. In contrast to traditions that heavily emphasise content knowledge and related procedural skills, Niss and colleagues position the \emph{enactment} of mathematics at the core of their mathematical competence framework.

\textcite{niss:2019} further propose that, while mathematical competenc\emph{e} refers to an activation of mathematics to deal with \emph{all} kinds of challenges, a mathematical competenc\emph{y} focuses on the activation to deal with a \emph{specific} sort of challenge. It is these specific activations of mathematics learners rely on in order to understand phenomena and relationships, answer questions, solve problems, and to make decisions. The authors subsequently structured their taxonomy of mathematical competencies into two categories, with each containing four elements. The first category looks into learners' ability to pose and answer questions in and by means of mathematics, whereas the second category deals with their ability to handle the language, constructs and tools of mathematics. This PhD project wishes to focus particularly on the first category ``how learners answer mathematical questions'' as measurable performance indicators and would like to zoom into the following three elements of Niss's original eight mathematical competencies:

\subsubsection{Problem-solving}
%//mark What is it, why is it important for students' learning of mathematics?

In a revised formulation, \textcite{niss:2019} limit this competency to intra-mathematical problems only. It refers to learners' ability to solve different kinds of mathematical problems within and across a variety of mathematical domains, then to critically reflect on their approaches and solutions. Problem-solving is the key competency for students' mathematical success since it directly relates to learners' ability to use mathematics for handling real-world challenges.

\subsubsection{Mathematical Modelling}
%//mark What is it, why is it important for students' learning of mathematics?

In contrast to problem-solving, mathematical modelling is the process of transforming an extra-mathematical question into a mathematical one. It also requires the learners to take into consideration the purposes, data, and constraints of the extra-mathematical domains (e.g., parameters cannot be negative) while analysing and evaluating the proposed models. Being able to reformulate complex real-world challenges into mathematical problems greatly expands students' mathematical success to many other life domains.

\subsubsection{Mathematical Reasoning and Argumentation}
%//mark What is it, why is it important for students' learning of mathematics?

The third competency examines learners' ability to construct chains of logical statements in order to justify their mathematical claims. It looks for learners' production of mathematical justification as well as their evaluation of justification attempts made by others. Justification may take many forms ranging from providing examples and counter-examples to rigorous proofs. This competency is central when learners communicate their mathematical learning to others. It serves to support the conclusions and decisions learners propose and upholds mathematics as a scientific endeavour that requires not only substantive (\emph{What} is true?) but also procedural (\emph{How} do you know this is true?) validity.

\section{Methodology}

\subsection{Data and Sample}

The present study will primarily use data sources from the Program for International Student Assessment (PISA). PISA is a major international large-scale assessment project conducted by the Organisation for Economic Co-operation and Development (OECD) every three years. PISA aims to assess 15-year-old students' literacy in reading, mathematics and science, with one literacy being the main focus in each cycle. Mathematics served as the major domain in 2012 and 2022, giving stakeholders significant insight into mathematics teaching and learning around the globe. PISA uses the two-stage sampling procedure and rotating booklet design to produce multiple plausible values (five for the 2012 cycle and ten for 2022) to represent candidates' mathematical literacy \parencite{rust:2014}. Statistical analyses often need to accommodate complex design features by incorporating weights, scalings and the hierarchical data structure. Although differ in wording, both the 2012 and 2022 PISA framework for mathematics recognises the interrelated aspects of process, content and context \parencite{oecd:2013}. The process aspect refers to an individual's capacity to formulate situations mathematically, then to employ mathematical concepts, facts, procedures, and reasoning to interpret, apply and evaluate mathematical outcomes \parencite[][p. 28]{oecd:2013}. The 2022 framework, furthermore, highlighted the mathematical reasoning and problem-solving elements of the process aspect, and introduced 21st Century skills into the context dimension in recognition of youth as consumers of quantitative, sometimes statistical, arguments \parencite{oecd:2018}.

To complement PISA data, Norwegian national register can be sourced as the secondary database for this PhD project. Year 10 students' education attainment records can be extracted from the administrative archive from 2012 to 2022 as corroborations. Register data provide unique statistical insight because it captures the entire Norwegian Year 10 student \emph{population} rather than its samples. School-level data such as lengths of COVID closures as well as students' socio-economic compositions were also retained by the national register. Sample sizes are expected to be 4,700 for the PISA 2012 data file \parencite{oecd:2014} and approximately 60,000 for a typical Year 10 cohort from national register.

\subsection{Methods of Analyses}

This PhD project involves both document analyses and quantitative computations. Texts from K20 \parencite{udir:2020} and PISA mathematics framework \parencite{oecd:2013,oecd:2018} are to be synthesised to address the first research question. The 2012 and 2022 PISA tests will then be aligned following procedures prescribed in \textcite{kolen:2014}. Difficulty parameters for each competency, as per the second research question, will be ascertained using item response theory \parencite{deayala:2022}, in particular partial credit models \parencite{masters:1982}. I intend to approach the last research question using multilevel structural equation models in order to account for the hierarchical nature of the PISA data, with both measurement and sampling errors being accounted for using techniques prescribed by \textcite{ludtke:2008} and \textcite{marsh:2009}.

\section{Articles}

I would like to organise my publications by ``whether'', ``how'', and ``who'' 2022 differed from 2012 in terms of Norwegian students' mathematics performance.

\subsection{Article 1}

The first article aims to cover the curriculum aspect of this PhD project. It will compare and contrast the 2012 and 2022 PISA mathematics framework and questionnaires, then map these differences to Norway's K20 curriculum reform. Particular attention will be paid to students' performance differences in tasks with high cognitive demand on problem-solving, modelling, and reasoning, respectively. Although proving causality is not a major theme in this article, detections of the presence or absence of performance variations after K20 would serve as a starting point for subsequent research and analyses.

\subsection{Article 2}

Should the first article report systematic performance differences between 2012 and 2022 PISA results amongst Norwegian students, the second article would investigate the patterns of such differences. IRT models will be applied to 2012 and 2022 datasets separately to ascertain the difficulty parameters for problem-solving, modelling, and reasoning competencies. I will then compare variations in these difficulty parameters to see whether all three competencies differed uniformly between the two cycles.

\subsection{Article 3}

The third article focuses on separating the effect of COVID-19 from that of K20. Using Norway's register data between 2010 and 2019, an ``ideal'' 2020 dataset can be extrapolated. Comparing the actual 2020 distribution with the expected distribution, I will be able to estimate the magnitude of COVID-19's impact on students' academic performance. Such procedure can be repeated for subsequent years until 2022. These population parameters obtained from national register can then inform the decomposition of PISA 2022 effects into the COVID-related and K20-related effects.

\subsection{Article 4}

The fourth article will address the fairness and equity considerations resultant from the K20 reform. Using multilevel structural equation models, this paper wishes to measure whether the improvement/deterioration of learners' problem-solving, modelling, and reasoning competencies are distributed equally across the socio-economic spectrum. An affirmative answer will lend social legitimacy to K20 while a negative finding will inform the distribution of remedial educational resources.

\section{Progress Plan}

I submit the following table summarising my proposed PhD progression:

\begin{table}[htbp]
    \begin{threeparttable}
    \caption{PhD Candidacy Time Frame}
    \label{tab:timeframe}
    \begin{tabular}{lcccccccc}
        \toprule
        \multicolumn{1}{c}{Milestone} & 2022H & 2023V & 2023H & 2024V & 2024H & 2025V & 2025H & 2026V \\
        \midrule
        Coursework & $\checkmark$     & $\checkmark$     & $\checkmark$     & $\checkmark$      & $\checkmark$      &       &       &  \\
        Align K20 and PISA & $\checkmark$     & $\checkmark$       &       &       &       &       &       &  \\
        Merge with register data & $\checkmark$     & $\checkmark$     & $\checkmark$       &       &       &       &       &  \\
        Article 1 &       & $\checkmark$     & $\checkmark$     &       &       &       &       &  \\
        Article 2 &       &       & $\checkmark$     & $\checkmark$     & $\checkmark$     &       &       &  \\
        Article 3 &       &       &       &       & $\checkmark$     & $\checkmark$     & $\checkmark$     &  \\
        Article 4 &       &       &       &       &       & $\checkmark$     & $\checkmark$     &  \\
        Kappe &       &       &       &       &       &       & $\checkmark$       & $\checkmark$ \\
        \bottomrule
        \end{tabular}
        \tablenote{H = Autumn semester; V = Spring semester.}
    \end{threeparttable}
\end{table}

\printbibliography

\end{document}
