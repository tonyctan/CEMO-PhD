\documentclass[
    % Options for apa7
    a4paper,                % Paper size
    11pt,                   % Font size
    stu,                    % Format as assignment
    donotrepeattitle,       % Start body text without repeating title
    floatsintext,           % Insert tables and figures with texts
    biblatex,               % Use BibLaTeX for references
    % Options for hyperref
    colorlinks=true,        % Colour all links
    linkcolor=red,          % Cross-references in red
    anchorcolor=black,      % Keep anchors black
    citecolor=blue,         % In-text-referencs in blue
    urlcolor=blue,          % DOIs and URLs are in blue
    bookmarks=true,         % Generate bookmarks for PDF readers
    bookmarksopen=false,    % Expand all bookmarks as default
    bookmarksnumbered=true, % Keep section number in bookmarks
    % Options for xcolor
    dvipsnames              % Use colour BrickRed and PineGreen
]{apa7}

% Specify absolute path to the tt.sty file, depending on the operating system
\usepackage{ifplatform}
\ifwindows
    \usepackage{M:/pc/Dokumenter/tt}
\else
    \usepackage{/home/tony/uio/pc/Dokumenter/tt} % Must not use ~ for home directory. Spell in full.
\fi

\usepackage{amssymb} % Check mark
\usepackage{rotating} % Rotate page to landscape

\title{Evalutaion of Norway's 2020 Curriculum Reform using PISA Data}
\authorsnames{Tony C. A. Tan}
\authorsaffiliations{{Department of Teacher Education and School Research, University of Oslo}}
\course{Project Proposal}
\professor{Dr Andreas Pettersen \& Prof Lovisa Sumpter}
\duedate{6 July 2022}

\begin{document}
\maketitle

% Restart page number at 1. Delete coverpage.
\setcounter{page}{1}

\section{Introduction and Rationale}

The curriculum revision (\textit{fagfornyelsen}) in 2020 (\textit{kunnskapsl{\o}ftet} 2020, K20) marks a major change in how students are taught in Norway \parencite{udir:2020}. It was the first time a substantial reformation of curricula was implemented since the 2006 reform (\textit{kunnskapsl{\o}ftet} 2006, K06). In mathematics, one major change was the establishment of core elements (\textit{kjerneelementer}) across all curricula spanning from Year 1 to 10. These core elements, namely, inquiry and problem solving, modelling and applications, reasoning and argumentation, representation and communication, abstraction and generalisations, and mathematical domains, to a large degree resemble the PISA mathematics framework \parencite{oecd:2018}---both share genesis with the eight competencies firstly proposed by Danish mathematician and educator Mogens Niss \parencite{niss:2003,niss:2011,niss:2019}.

The implementation of the core elements in K20 and their close alignment with the PISA framework provides a golden opportunity to study Norwegian students' learning outcomes using one cycle of PISA \emph{before}, and one \emph{after}, the introduction of K20. Since mathematics was the major domain in PISA 2012 and once again in 2022, these two time points may serve as the pre-test and post-test in an ``experiment'' with K20 being the ``treatment'' \parencite{shadish:2002}. Two factors, however, complicate this quasi-experimental interpretation. First, K20 was implemented \emph{concurrently} with the COVID-19 school closures and the resultant home schooling. Separating effects attributable to the pandemic from those of K20 is therefore a chief task in this project. Second, PISA employs a cross-sectional rather than longitudinal design, limiting any causal inferences. Yet, PISA data sets, and especially combined with Norwegian register data (e.g., national test results, as well as teacher-assigned and exam grades), are the best data sources available in Norway to study the effect of K20.

Mapping students' knowledge, understanding and skills within these core elements, in particular, \emph{problem solving}, \emph{modelling} and \emph{reasoning}, is important for three reasons. First of all, an in-depth understanding of students' mastery of these key capabilities would provide insight into their command of 21st Century skills \parencite{rizki:2019}. Secondly, \parencite{pettersen:2018,pettersen:2019} \emph{overall} mathematics

\section{Overall Aims and Research Questions}

In this PhD, I wish to address this gap in research and the need to

\section{Theoretical Framework}

%//mark This section starts by providing an overview of the Norwegian education system, and curricular changes.
%//mark It then provides a brief overview of K20.
%//mark A theoretical framework for mathematics competencies in general, with specific overviews of previous research on:
%//mark - Problem solving
%//mark - Modelling, and
%//mark - Reasoning
%//mark constitutes the last and main part of the theoretical framework.

\subsection{The Norwegian Education System}

\subsection{Norway's Recent Curricular Changes}

\subsection{Mathematical Competencies}

\subsubsection{Mathematical Modelling}

\subsubsection{Mathematical Reasoning and Argumentation}

\subsubsection{Problem Solving}

\section{Methodology}

\subsection{Data and Sample}

\subsection{Methods of Analyses}

\section{Articles}

\subsection{Article 1}

\subsection{Article 2}

\subsection{Article 3}

\subsection{Article 4}

\section{Progress Plan}

\begin{table}[htbp]
    \begin{threeparttable}
    \caption{PhD Candidacy Timeframe}
    \label{tab:timeframe}
    \begin{tabular}{lcccccccc}
        \toprule
        \multicolumn{1}{c}{Milestone} & 2022H & 2023V & 2023H & 2024V & 2024H & 2025V & 2025H & 2026V \\
        \midrule
        Coursework & $\checkmark$     & $\checkmark$     & $\checkmark$     & $\checkmark$      & $\checkmark$      &       &       &  \\
        Align K20 and PISA & $\checkmark$     &       &       &       &       &       &       &  \\
        Merge with register data & $\checkmark$     & $\checkmark$     &       &       &       &       &       &  \\
        Article 1 &       & $\checkmark$     & $\checkmark$     &       &       &       &       &  \\
        Article 2 &       &       & $\checkmark$     & $\checkmark$     & $\checkmark$     &       &       &  \\
        Article 3 &       &       &       &       & $\checkmark$     & $\checkmark$     & $\checkmark$     &  \\
        Article 4 &       &       &       &       &       & $\checkmark$     & $\checkmark$     &  \\
        Kappe &       &       &       &       &       &       &       & $\checkmark$ \\
        \bottomrule
        \end{tabular}
    \end{threeparttable}
\end{table}

\printbibliography

\end{document}
