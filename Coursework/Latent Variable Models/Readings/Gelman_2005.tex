\documentclass[stu,11pt]{apa7}

\usepackage{amsmath,amssymb}

\usepackage{parskip}
\setlength{\parindent}{0.5in}

\title{Why I don't use the term ``fixed and random effects''}
\author{Andrew Gelman}
\date{Posted on January 25, 2005 6:41 PM}

\begin{document}
\maketitle

\setcounter{page}{1}

People are always asking me if I want to use a fixed or random effects model for this or that. I always reply that these terms have no agreed-upon definition. People with their own favorite definition of ``fixed and random effects'' don't always realize that other definitions are out there. Worse, people conflate different definitions.

\section{Five definitions}

Here are the five definitions I've seen:

\begin{enumerate}
    \item Fixed effects are constant across individuals, and random effects vary. For example, in a growth study, a model with random intercepts $a_i$ and fixed slope $b$ corresponds to parallel lines for different individuals $i$, or the model $y_{it} = a_i + b_t$. Kreft and De Leeuw (1998) thus distinguish between fixed and random coefficients.
    \item Effects are fixed if they are interesting in themselves or random if there is interest in the underlying population. Searle, Casella, and McCulloch (1992, Section 1.4) explore this distinction in depth.
    \item ``When a sample exhausts the population, the corresponding variable is fixed; when the sample is a small (i.e., negligible) part of the population the corresponding variable is random.'' (Green and Tukey, 1960)
    \item ``If an effect is assumed to be a realized value of a random variable, it is called a random effect.'' (LaMotte, 1983)
    \item Fixed effects are estimated using least squares (or, more generally, maximum likelihood) and random effects are estimated with shrinkage (``linear unbiased prediction'' in the terminology of Robinson, 1991). This definition is standard in the multilevel modeling literature (see, for example, Snijders and Bosker, 1999, Section 4.2) and in econometrics.
\end{enumerate}

\section{The definitions are all different!}

Of these definitions, the first clearly stands apart, but the other four definitions differ also. Under the second definition, an effect can change from fixed to random with a change in the goals of inference, even if the data and design are unchanged. The third definition differs from the others in defining a finite population (while leaving open the question of what to do with a large but not exhaustive sample), while the fourth definition makes no reference to an actual (rather than mathematical) population at all. The second definition allows fixed effects to come from a distribution, as long as that distribution is not of interest, whereas the fourth and fifth do not use any distribution for inference about fixed effects. The fifth definition has the virtue of mathematical precision but leaves unclear when a given set of effects should be considered fixed or random. In summary, it is easily possible for a factor to be ``fixed'' according to some of the definitions above and ``random'' for others.

Because of these conflicting definitions, it is no surprise that ``clear answers to the question `fixed or random?' are not necessarily the norm'' (Searle, Casella, and McCulloch, 1992, p. 15).

\section{In summary \dots}

If you use a particular definition of fixed and random effects, don't automatically assume that the other definitions apply also. For example, if an effect is interesting in itself (see definition 2), it is not necessary to estimate it using least squares (see definition 5).

(See also Kreft and De Leeuw, 1998, Section 1.3.3, for a discussion of the multiplicity of definitions of fixed and random effects and coefficients, and Robinson, 1998, for a historical overview.)

\section{The paper}

This is all taken from my paper, ``Analysis of variance: why it is more important than ever'' (with discussion), Annals of Statistics, 2005.

Or it may be more fun to start with my rejoinder to the discussion.

\newpage

\section{References}

\hangindent=0.5in Kreft, I., and De Leeuw, J. (1998). \textit{Introducing Multilevel Modeling}. London: Sage.

Searle, S. R., Casella, G., and McCulloch, C. E. (1992). \textit{Variance Components}. New York: Wiley.

Green, B. F., and Tukey, J. W. (1960). Complex analyses of variance: General problems. \textit{Psychometrika}, \textit{25} 127–-152.

LaMotte, L. R. (1983). Fixed-, random-, and mixed-effects models. In S. Kotz, N. L. Johnson, and C. B. Read (ed.) \textit{Encyclopedia of Statistical Sciences}, \textit{3}, 137–-141.

Snijders, T. A. B., and Bosker, R. J. (1999). \textit{Multilevel Analysis}. London: Sage.

Robinson, G. K. (1991). That BLUP is a good thing: the estimation
of random effects (with discussion). \textit{Statistical Science}, \textit{6}, 15–-51.

\end{document}
