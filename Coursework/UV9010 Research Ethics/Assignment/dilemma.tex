\documentclass[
    a4paper,            % Paper size
    11pt,               % Font size
    stu,                % Format as assignment
%   donotrepeattitle,   % Start body text without repeating title
    noextraspace,       % Reduce spaces between section header and text
    floatsintext,       % Insert tables and figures with texts
    biblatex,           % Use BibLaTeX for references
    colorlinks=true,        % Colour all links
    linkcolor=red,          % Cross-references in red
    anchorcolor=black,      % Keep anchors black
    citecolor=blue,         % In-text-referencs in blue
    urlcolor=blue,          % DOIs and URLs are in blue
    bookmarks=true,         % Generate bookmarks for PDF readers
    bookmarksopen=false,    % Expand all bookmarks as default
    bookmarksnumbered=true  % Keep section number in bookmarks
]{apa7}

% Avoid breaking a word into two lines
\usepackage[none]{hyphenat}

% Activate AMS packages for typing maths
\usepackage{amsmath,amssymb}

\newcommand{\p}[1]{\mathbb{P}\left(#1\right)}
\newcommand{\E}[1]{\mathbb{E}\left(#1\right)}
\renewcommand{\exp}[1]{\mathrm{exp}\left\{#1\right\}}

% Insert images
\usepackage{graphicx}

% Change line spacing
\usepackage{setspace}
\setstretch{2}

\usepackage[nameinlink,noabbrev,capitalise]{cleveref}

% Package biblatex has already been loaded by apa7.
% Only need to specify the bib library
\addbibresource{../../../Bibliography/Master.bib}
\newcommand{\poscite}[1]{\citeauthor{#1}'s (\citeyear{#1})}

\title{A Short Project Presentation Including Research Ethics Dilemma}
\author{Tony C. A. Tan}
\affiliation{Centre for Educational Measurement, University of Oslo}
\course{UV9010 Research Ethics}
\professor{Associtate Prof Anett Kaale}
\duedate{27 September 2021}

\begin{document}
\maketitle

Universities in Norway use applicants' grade point averages (GPA) as the sole admission criterion. The fairness of the GPA system, however, has been repeatedly called into question by studies overseas \parencite{he:2018,korobko:2008}. Since not all grades mean the same thing \parencite{caulkins:1996}, subjects differ in their difficulties and lower GPAs could reflect \emph{either} lower competency \emph{or} candidates' decisions of taking subjects with more stringent grading standards. This study directly tests the inter-subject difficulties using Norway's GPA archival record in order to verify the existence of significant misalignment in difficulty parameters and proposes a statistical procedure for restoring cross-subject comparability.

For this study, students' GPA records will be extracted from the Norwegian registry covering the period between 2009 (the year ``clean data'' became available after the 2006 reform) and 2019 (the last ``normal year'' before COVID). GDPR registration is lodged through the NSD Portal and the UiO ethics approval is also obtained. All data import, storage, and analyses are to be conducted within the secured infrastructure TSD provided by the UiO Central IT Division. TSD logs all activities and no data or results can be copied out of the restricted system without prior approval from project leaders.

This project experiences the ethic dilemma concerning the re-identifiability of test subjects. Although immediate identifiers such as names, dates of birth and addresses have been recoded by the data provider \textit{Statistisk sentralbyr{\aa}}, it is entirely feasible for researchers to narrow the database down to a manageable number of entries by applying filters such as ``females, from private schools, have taken both physics and advanced mathematics for GPA, in the year 2019''. Privacy and trust to public institutions may be undermined should researchers release analysis results to the public domain without further de-identification.

\printbibliography

\end{document}