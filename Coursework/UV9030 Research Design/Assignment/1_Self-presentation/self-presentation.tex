\documentclass[
    a4paper,            % Paper size
    11pt,               % Font size
    stu,                % Format as assignment
    donotrepeattitle,   % Start body text without repeating title
    noextraspace,       % Reduce spaces between section header and text
    floatsintext,       % Insert tables and figures with texts
    biblatex,           % Use BibLaTeX for references
    colorlinks=true,        % Colour all links
    linkcolor=red,          % Cross-references in red
    anchorcolor=black,      % Keep anchors black
    citecolor=blue,         % In-text-referencs in blue
    urlcolor=blue,          % DOIs and URLs are in blue
    bookmarks=true,         % Generate bookmarks for PDF readers
    bookmarksopen=false,    % Expand all bookmarks as default
    bookmarksnumbered=true  % Keep section number in bookmarks
]{apa7}

% Avoid breaking a word into two lines
\usepackage[none]{hyphenat}

% Activate AMS packages for typing maths
\usepackage{amsmath,amssymb}

\newcommand{\p}[1]{\mathbb{P}\left(#1\right)}
\newcommand{\E}[1]{\mathbb{E}\left(#1\right)}
\renewcommand{\exp}[1]{\mathrm{exp}\left\{#1\right\}}

% Insert images
\usepackage{graphicx}

% Change line spacing
\usepackage{setspace}
\setstretch{1.25}

\usepackage[nameinlink,noabbrev,capitalise]{cleveref}

% Package biblatex has already been loaded by apa7.
% Only need to specify the bib library
\addbibresource{../../../../Bibliography/Master.bib}
\newcommand{\poscite}[1]{\citeauthor{#1}'s (\citeyear{#1})}

\title{A Self-presentation on Research Design}
\author{Tony C. A. Tan}
\affiliation{Centre for Educational Measurement, University of Oslo}
\course{UV9030 Research Design}
\professor{Prof {\O}istein Anmarkrud \& Prof Marte Blikstad-Balas}
\duedate{10 September 2021}

\begin{document}
\maketitle

%//mark Make a self-presentation (not more than 1 page). This presentation should comprise information about: your area of investigation, one or two purpose statements describing the broader aims of your research, main research questions, research design (primarily qualitative, quantitative or mixed methods), and why you have chosen the design (see Creswell & Creswell, chapters 1–5). This presentation should be submitted in Canvas by 10th of September. The self-presentations will be used in collaborative groups (to receive and offer feedback) during the course. Please bring 6 copies of your self-presentation to the course.

\section{Area of Investigation}

My project investigates the fairness of Norwegian universities' admission processes. Unlike their counterpart in the USA who assign considerable weights to extra-curriculum activities and reference letters in addition to academic performance, universities in Norway use applicants' grade point averages (GPA) as the sole selection criterion. Studies from the UK \parencite{coe:2008} and the Netherlands \parencite{korobko:2008}, however, have demonstrated that GPA in their unadjusted form can be highly variable in difficulties across subjects, causing considerable fairness concerns. Numerous adjustment procedures have been proposed and adopted in many jurisdictions \parencite{lamprianou:2009} to varying degrees of success.

\section{Purpose Statement}

My study wishes to examine the viability of using one particular statistical technique called partial credit model \parencite{masters:1982} for the purpose of re-aligning subject difficulties during GPA calculations. Using students' GPA records from the Norwegian registry data between 2006 (the year of education reform) and 2020 (most recent  year with available data), I wish to answer these research questions: (1) To what extent can a partial credit model approximate Norway's GPA scores? (2) Does the difficulty parameter differ significantly across subjects? If the answer to (2) is ``yes'', I then would like to further enquire (3a) Which are the easiest and hardest subjects? (3b) Are there any systematic differences by social-demographic variables such as sex, immigration history or social-economic spectrum? (3c) Do such differences remain stable over time?

\section{Research Design}

This study takes a pooled cross sectional quantitative research design by repeating the same measurement over years but on different test candidates. According to the item response theory, students' academic performance may be considered as the outcome of their underlying ``latent traits''. By examining the covariation among GPA scores, candidates' general academic ability can therefore be ascertained.

\section{Motivation}

This project primarily takes on the postpositivism worldview while lending empirical support to the transformative worldview. It assumes determination between one's ability and their performance and reduces one's complex academic performance process into a few numeric scores. By conducting empirical observation and measurement using archival data, this study attempts to verify the suitability of the item response theory for educational assessment purposes. Three factors influenced my decision of taking this quantitative research approach: (1) my research questions demand a \emph{numerical measure} of goodness-of-fit between a mathematical model and the actual data; (2) I am trained in econometrics, educational measurement and I am familiar with statistical procedures and quantitative paper structure; and (3) the audience of my research, including my thesis advisors, expect an unambiguous answer to the fairness, or the lackthereof, of the university admission criterion currently practised in Norway.
\printbibliography

\end{document}