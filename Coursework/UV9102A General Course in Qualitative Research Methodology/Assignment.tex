%//mark [PowerPoint] Upon completion of part A, students are to have acquired an understanding of different qualitative methodologies and their relative strengths and weaknesses. They will be able to evaluate qualitative research on relevant criteria, and the quality of the results of such research. The course is intended to strengthen the students’ competence in scientific reasoning and argumentation.

%//mark In the paper, you should

%//mark Select one of the empirical papers for a critical analysis
%//mark Give a descriptive summary of the study
%//mark Discuss the credibility of the study, by assessing
%//mark     Internal coherence (relations between research questions and design)
%//mark     Transparency of the research process: How data was collected description of the analytical strategy, empirical grounding of claims
%//mark     Reported validation strategies, scope and form of conclusions
%//mark     Include both strengths and limitations in your discussion
%//mark Relate your discussion to selected texts from the course literature--but you may also draw on other sources
%//mark Your text should also meet criteria for transparent and grounded claims
%//mark 2000--3000 words length
%//mark To be submitted in Canvas by 31 March 2022

\documentclass[
    % Specify apa7 options
        a4paper, % Use A4 paper
        12pt, % APA7 Rule 2.19 10pt Computer Modern too small. Change to 11pt.
        stu, % All CEMO thesis must be compiled using student paper mode.
        donotrepeattitle, % Will generate heading in Section 1 myself.
%        noextraspace, % Make section headings and text closer and nicer.
        floatsintext, % Insert tables and figures in the body of the thesis.
        biblatex, % Use BibLaTeX as the engine.
        twoside, % Prepare LaTeX for open-right
    % Specify hyperref options
        colorlinks=true,        % Colour all links
        linkcolor=red,          % Cross-references in red
        anchorcolor=red,      % Keep anchors black
        citecolor=blue,         % In-text-referencs in blue
        urlcolor=blue,          % DOIs and URLs are in blue
        bookmarks=true,         % Generate bookmarks for PDF readers
        bookmarksopen=false,    % Expand all bookmarks as default
        bookmarksnumbered=true,  % Keep section number in bookmarks
    % Specify xcolor options
        dvipsnames
]{apa7}

\title{Main title goes here}
\author{Tony Tan}
\affiliation{Centre for Educational Measurement, University of Oslo}
\course{UV9102A General Course in Qualitative Research Methodology}
\professor{Prof Marte {Blikstad-Balas} and Prof Monika B{\ae}r{\o}e Nerland}
\duedate{31 March 2022}

\usepackage{/home/tony/uio/pc/Dokumenter/tt}
% Must not use ~ as a shorthand for home directory. Spell the path in full.

\begin{document}
\maketitle

%//mark Select one of the empirical papers for a critical analysis

\section{Paper Summary} %//mark 250 words
%//mark Give a descriptive summary of the study

\textcite{hennessy:2021} presented a qualitative research project aimed at supporting practitioner-led inquiry into classroom dialogue. Using a design-based approach involving 74 participants raging from early education to the tertiary levels, the authors examined the effectiveness of the Teacher Scheme for Educational Dialogue Analysis (T-SEDA) resource pack for promoting teachers' implementation of classroom dialogue. By examining data derived from surveys, inquiry reports and interviews, Hennessy and colleagues reported significant effect sizes attributable to T-SEDA in participating classrooms in terms of classroom dialogue effectiveness.

\section{Credibility Examination} %//mark 200 words
%//mark Discuss the credibility of the study, by assessing:

\subsection{Internal Coherence} %//mark 200 words
%//mark Internal coherence (relations between research questions and design)

\subsection{Transparency of Research Process} %//mark 200 words
%//mark Transparency of the research process: How data was collected description of the analytical strategy, empirical grounding of claims

\subsection{Validation Strategies} %//mark 200 words
%//mark Reported validation strategies, scope and form of conclusions

\section{Discussion} %//mark 1000 words
%//mark Include both strengths and limitations in your discussion

\printbibliography

\end{document}