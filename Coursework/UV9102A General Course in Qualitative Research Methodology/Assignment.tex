%//mark [PowerPoint] Upon completion of part A, students are to have acquired an understanding of different qualitative methodologies and their relative strengths and weaknesses. They will be able to evaluate qualitative research on relevant criteria, and the quality of the results of such research. The course is intended to strengthen the students’ competence in scientific reasoning and argumentation.

%//mark In the paper, you should

%//mark Select one of the empirical papers for a critical analysis
%//mark Give a descriptive summary of the study
%//mark Discuss the credibility of the study, by assessing
%//mark     Internal coherence (relations between research questions and design)
%//mark     Transparency of the research process: How data was collected description of the analytical strategy, empirical grounding of claims
%//mark     Reported validation strategies, scope and form of conclusions
%//mark     Include both strengths and limitations in your discussion
%//mark Relate your discussion to selected texts from the course literature--but you may also draw on other sources
%//mark Your text should also meet criteria for transparent and grounded claims
%//mark 2000--3000 words length
%//mark To be submitted in Canvas by 31 March 2022

\documentclass[
    % Specify apa7 options
        a4paper, % Use A4 paper
        12pt, % APA7 Rule 2.19 10pt Computer Modern too small. Change to 11pt.
        stu, % All CEMO thesis must be compiled using student paper mode.
        donotrepeattitle, % Will generate heading in Section 1 myself.
%        noextraspace, % Make section headings and text closer and nicer.
        floatsintext, % Insert tables and figures in the body of the thesis.
        biblatex, % Use BibLaTeX as the engine.
        twoside, % Prepare LaTeX for open-right
    % Specify hyperref options
        colorlinks=true,        % Colour all links
        linkcolor=red,          % Cross-references in red
        anchorcolor=red,      % Keep anchors black
        citecolor=blue,         % In-text-referencs in blue
        urlcolor=blue,          % DOIs and URLs are in blue
        bookmarks=true,         % Generate bookmarks for PDF readers
        bookmarksopen=false,    % Expand all bookmarks as default
        bookmarksnumbered=true,  % Keep section number in bookmarks
    % Specify xcolor options
        dvipsnames
]{apa7}

\title{A Critique Paper on Hennessy et al. (2021)}
\author{Tony Tan}
\affiliation{Centre for Educational Measurement, University of Oslo}
\course{UV9102A General Course in Qualitative Research Methodology}
\professor{Prof Marte {Blikstad-Balas} and Prof Monika B{\ae}r{\o}e Nerland}
\duedate{31 March 2022}

\usepackage{/home/tony/uio/pc/Dokumenter/tt}
% Must not use ~ as a shorthand for home directory. Spell the path in full.

\begin{document}
\maketitle

\section{Paper Selected}
%//mark Select one of the empirical papers for a critical analysis
\longcite{hennessy:2021}

\section{Paper Summary} %//mark 500 words
%//mark Give a descriptive summary of the study

\textcite{hennessy:2021} presented a qualitative research project aimed at supporting practitioner-led inquiry into classroom dialogue. Using a design-based approach involving 74 participants raging from early education to the tertiary levels, the authors examined the effectiveness of the Teacher Scheme for Educational Dialogue Analysis (T-SEDA) resource pack for promoting teachers' implementation of dialogic pedagogy. By examining data derived from surveys, inquiry reports and interviews, Hennessy and colleagues reported significant effect sizes attributable to the T-SEDA in participating classrooms in terms of classroom dialogue implementation.

This paper began with a clear and purposeful introduction. The authors motivated their study with research gaps and limitations from unidirectional intervention designs --- an argument naturally led to a design-based research (DBR) methodology. A brief description of DBR was then given followed by a list of its distinctive features. These paragraphs served non-specialists particularly well in terms of scaffolding and contextualising this project. Subsequently, Hennessy and colleagues linked T-SEDA's design principles to DBR by explicitly positioning practitioners as contributors to research knowledge. The introduction section ended with an outline, giving readers a clear structure and purpose for this paper.

Next, T-SEDA as a professional development (PD) tool were carefully derived in the context of classroom dialogue development. The authors firstly introduced the theoretical foundation of their key concept ``dialogue'' (Vygotsky's sociocultural theory), carefully distinguished dialogue from talks, and presented the challenges for both students and teachers during pedagogical practices. Prior publications related to classroom dialogue were then presented, with T-SEDA being anchored to the strand of research that focuses on the forms and functions of classroom discourse. After rejecting the ``recitation'' model, Hennessy and co-authors listed obstacles to wider implementations of learning-focused dialogue, particularly in secondary schools, and identified PD opportunities as effective remedies for promoting classroom dialogic practices. The success of various PD setup has been mixed, however, due to most PD program's reliance on external providers, small scales as well as short durations. Although compulsory participation and/or random assignment greatly promote the validity of study designs, such approach removes teachers' agency---a problem T-SEDA was specifically designed to address thanks to its two-way designs and being an open educational resource (OER).

The paper then described in detail the composition of a T-SEDA pack. The authors provided sufficient information regarding the length (70 pages), content (user's guide, core resources, and additional resources), as well as how to implement the ``reflective cycle'' through both texts and diagrams. T-SEDA and similar packages designed under the DBR approach were shown to be widely tested and well received, although the exact motives behind teachers' engagement with T-SEDA as well as the organisational structures and circumstances conducive to T-SEDA uptake were not well understood. Resultantly, the authors made answering these questions the mission of this research project.

\section{Credibility Examination} %//mark 200 words
%//mark Discuss the credibility of the study, by assessing:

%This study has a mix success in establishing its research credibility.

\subsection{Internal Coherence} %//mark 200 words
%//mark Internal coherence (relations between research questions and design)

Internal coherence refers to the connection between the research questions and the research design. The authors listed two research questions: a) ``Why might individual practitioners engage with and disengage from T-SEDA inquiry?'' and b) ``What are the organisational structures and circumstances supporting engagement with T-SEDA in different local settings?'' In answering the ``why'' question, Hennessy and colleagues examined participants' responses to two rounds of online surveys as well as inquiry reports; while the ``what'' question was studied through post-study semi-structured interviews with local facilitators. By approaching different sub-groups of participants for different research questions, this paper enhanced the connection between research questions and designs hence has strong internal coherence.

\subsection{Transparency and Validation} %//mark 400 words
%//mark Transparency of the research process: How data was collected description of the analytical strategy, empirical grounding of claims
%//mark Reported validation strategies, scope and form of conclusions

\poscite{hennessy:2021} study is highly transparent in its research process. Under the guidance of \textcite[][pp. 185--198]{creswell:2018}, the study authors have demonstrated considerations to the following qualitative research best practices:

\subsubsection{Data Collection}

The idea underpinning qualitative research is to purposefully select participants in order to best help the researchers understand the problem and to answer the research questions. In contrast to their quantitative counterpart, qualitative research projects do \emph{not} necessarily demand large sample sizes or random sampling procedures. Instead, Hennessy and colleagues approached their samples with the setting, actors, events and process in mind by disclosing the recruitment procedure (both general advertising such as Twitter and professional contacts), sample sizes (breaking down by education sector, ranging from early ears to higher education), and geographic coverage (e.g., England as well as global distribution from New Zealand to Pakistan). The authors then documented in detail the procedure they have followed for collecting surveys, reports and interview data (such as the face-to-face workshops and meetings and online session for remote participants).

Another highlight of the \textcite{hennessy:2021} paper is its attention to research ethics. Both UK's Ethical Guidelines for Educational Research and EU's GDPR were explicitly acknowledged and followed by the research team via written informed consent, with further safeguard in the form of additional ethical approval from the participants' end. Lastly, the paper disclosed any funding arrangement that may carry weights towards independent in appearance or in fact. In summary, the data collection section is highly transparent, and signalled an accountable, ethical and credible publication.

\subsubsection{Data Recording}

The data recording section of this paper is also strong and transparent. The authors first of all reported the detailed sample sizes of each round of surveys, reports and interviews. The employment of research assistants who were not involved in the data collection or program design was well justified on the ground of minimising bias. Both deductive and inductive coding practices were carried out and justified by the authors with examples. Interviews were then transcribed in ``intelligent verbatim'' form with hesitations, repetitions and ungrammatical or filler words removed in order to prepare a lightly edited record of the conversation. Lastly, the research team reported a procedure similar to the checklist by \textcite[][pp. 193--195]{creswell:2018} by familiarising themselves with the interview transcripts, categorising different local structures and deriving new insight that were not built into the original design. Respondent validation was conducted as a quality assurance final step by sharing early drafts amongst facilitators. Overall, sufficient information was given by the authors to make their data recording procedure trustworthy.

\subsection{Data Analysis}

The attrition section of \textcite[][pp. 17--20]{hennessy:2021} substantially enhanced this paper's transparency and credibility. A detailed description of the number of participants in each research stage was given in both text and diagrammatic forms. The authors explicitly acknowledged the high attrition rate and spent long paragraphs on documenting and conjecturing its causes and effects. Narrations from some facilitators provided nuanced insight into why participants failed complete the full course and compelling reasons have been identified as reasonable justification for dropouts. Statistical analyses were carried out, showing that participants' inabilities to (a) distinguish dialogue and other forms of talk and (b) identify features of a dialogic classroom were similar between dropouts and retained members ($\chi^2(3,154)=6.38,\ p=.094$), while significant differences was observed between dropouts and retained teachers when it comes to their initial familiarity with classroom dialogue in practice and their readiness to conduct systematic inquiry ($\chi^2(3,154)=15.28,\ p=.0016$). Such transparency with dropouts not only enhanced the credibility of the research team but also led the authors to the insight that the retaining sample of participants was tend to have (a) stronger initial understanding of dialogue, which in turn may have motivated them to persist with inquiries --- a significant discovery and answer to the first research question.

\section{Discussion} %//mark 700 words
%//mark Include both strengths and limitations in your discussion

The proceeding paragraphs painted a publication that is strong in credibility, transparency and validity. \textcite{hennessy:2021} successfully achieved their goal of interrogating the ``why'' and ``what'' of practitioner-led inquiry into classroom dialogue. By initialising their research project on strong ethical foundation and by carefully recruiting, measuring, documenting, coding and reflecting on possible dropout causes and effects, this paper was able to establish itself in delivering authoritative research conclusions such as the importance of securing institutional leadership buy-in in promoting classroom dialogue practices.

This paper also serves as a exemplar that highlights the differences between qualitative and quantitative research. It highlighted, amongst others, the different purposes hence procedure in data collection (purposefully selection vs random sampling, small vs large sample sizes) as during data analyses (data ``winnowing'' vs data preservation). This insight is particularly beneficial for quantitative researchers.

\printbibliography

\end{document}