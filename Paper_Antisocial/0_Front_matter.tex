\abstract{%
Parental mental distress can have a significant impact on adolescents' antisocial behavior (ASB), and this relationship can be exacerbated or alleviated by family dynamics such as conflict and cohesion. By analyzing responses to clinical questionnaires from 157 Norwegian adolescents and their primary caregivers, this study found a significant mediating role family conflict played in strengthening the relationship between parental mental distress and adolescents' ASB. Family cohesion, on the other hand, provided insufficient protective effect. Both results are consistent with the family stress model, and highlight the importance of addressing not only individuals' mental health but also family-level factors in preventing and treating adolescent ASB. This insight is robust against measurement errors through the use of Bayesian structural equation models.
}

\keywords{%
adolescent antisocial behavior, parental mental distress, family conflict, family cohesion, mediation effect, family stress model, Bayesian structural equation model\\

\bigskip

\noindent \textbf{Highlights:}
\begin{itemize}
    \item Parental mental distress has a significant direct effect on adolescent ASB, family conflict and cohesion.
    \item Family conflict has a significant mediating role in strengthening the relationship between parental mental distress and adolescent ASB.
    \item Family cohesion shows no mediating effect between parental mental distress and adolescent ASB.
\end{itemize}
}