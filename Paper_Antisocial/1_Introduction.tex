%\section{Introduction} % Do not include the word "Introduction" as a Level 1 heading. Just start intro as a normal paragraph immediately after the title.

Adolescence is a time known for change in relationships and family structure. Normative for this period is increased family conflict and reduced family cohesion. However, in families with parental psychopathology and adolescent antisocial behaviors, these changes may be more severe. Interpersonal relationships and interactions between family members are highlighted as certain aspects that may exacerbate this influence, and are important to consider as underlying factors and triggers for adolescent outcomes \parencite{vanloon:2014, xu:2017}.

\subsection{Adolescent Antisocial Behavior}

Antisocial behavior (ASB) is characterized as behaviors that violate norms and rules about how persons and property should be treated \parencite{scott:2015}. These behaviors are destructive and insensitive to other people's rights, it can be criminal and noncriminal, overt and covert, and may include aggression, substance use, bullying, sexual precocity, vandalism, and delinquency \parencite{dishion:2006}. Persistent ASB may have major long-term consequences, both for the individual and society (e.g., academic failure, drug abuse, violence, and economic struggle) \parencite{lobraico:2020, moffitt:2018}. Exhibition of ASB is both a common and heterogeneous form of problem behavior among youth  \parencite{frick:2009}. Early emerging ASB have a higher chance of persisting into adulthood, due to more severe individual and environmental risk factors. Research indicates that maternal psychopathology, harsh and neglectful parenting, and elevated family conflict are some of such risk factors \parencite{dishion:2006, moffitt:2015}.

Growing research advocates for distinguishing between different subtypes for adolescent ASB \parencite{burt:2012, burt:2009, kornienko:2019}. The main distinction is between aggressive (e.g., threats, physical aggression and violence) and non-aggressive behaviors (e.g., theft, vandalism, and relational aggression) \parencite{burt:2016, kornienko:2019, little:2003}. Some also include risk-taking behaviors \parencite{mishra:2008}, defined as engagement in actions that are associated with potentially adverse consequences \parencite{boyer:2006}. Risk-taking behaviors are thought of as more normative in adolescence \parencite{moffitt:2018, sundell:2019}, they are not necessarily illegal or dangerous, but include actions where the outcome is uncertain \parencite{ciranka:2021}. \textcite{steinberg:2004} points out that adolescents are susceptible to peer pressure, making them more likely to engage in similar activities and behaviors as their peers \parencite{ciranka:2021}.

\subsection{Parental Mental Distress}

The connection between parental mental distress, symptoms of depression and anxiety, are well established risk factors for negative child and adolescent outcomes \parencite[e.g.,][]{cummings:1994, goodman:2006, hails:2018, hawes:2005}, indicating that mental distress may reduce parents' ability to engage in proactive and positive parenting \parencite{elgar:2007, joyner:2021}. Family environments with depressed caregivers are characterized by negative patterns of interpersonal interactions, lax monitoring, and inconsistent discipline and display of affection \parencite{elgar:2007, korhonen:2014}. However, for anxious parents, they tend to be more controlling and overprotective, parenting their offspring closely, expecting disclosure of information, and allowing less autonomy \parencite{jones:2021, vera:2012}.

Cummings and colleagues (\citeyear{cummings:2005}) found that parental depressive symptoms were linked to poor child adjustment, both internalizing and externalizing problems, peer rejection, and lack of prosocial behavior. Greater parental symptoms were associated with intrusiveness, control through guilt, and less parental warmth. Parental rejection and overprotection has been found to mediate the association between parental psychopathology and offspring ASB \parencite{vera:2012}. \textcite{korhonen:2014} investigated whether it is the timing, recurrence or chronicity of maternal depression that puts offspring's wellbeing at risk. Findings indicate that recurrent depressive symptoms were significantly associated with adolescents' poorer psychosocial health, including self-reported externalizing behaviors. Anxiety symptoms in mothers are associated with negative criticism \parencite{hirshfeld:1997}, and lower levels of affirmation towards their adolescent, which in turn predict higher levels of externalizing behaviors \parencite{bellina:2020}. Parental rejection and overprotection was found to mediate the association between parental psychopathology and offspring ASB \parencite{vera:2012}.

Research is somewhat conflicted on mothers and fathers separate influence on offspring maladjustment \parencite{cummings:2005, sweeney:2016}. Both \textcite{marmorstein:2004} and Vera and colleagues (\citeyear{vera:2012}) found that maternal psychopathology had greater influence and predicted higher levels of maladjustment, compared to fathers. Conversely, a meta-analysis by \textcite{connell:2002} did not find differences in mothers and fathers psychopathology on externalizing behavior. However, they found that parents' gender may predict internalizing behavior, with mothers having greater influence.

\subsection{Family Conflict and Cohesion as Mediators}

PMD may function as a risk factor for increased conflict levels and lower levels of cohesion within families. Family conflict involves frequent expression of anger, hostility, and resentment \parencite{lobraico:2020}. Adolescents' desire for autonomy and liberation from parental control may be a source for frustration, friction, and conflict \parencite{buehler:2006, saxbe:2014}, as they attempt to adjust boundaries, renegotiate parental authority, and increase their independence \parencite{weymouth:2016}. High family conflict is associated with emotional and behavioral problems (e.g. mental distress, aggression, delinquency, and school problems) \parencite{fosco:2020, sun:2021, xu:2017}. Similarly, a meta-analysis by Weymouth and colleagues (\citeyear{weymouth:2016}) found positive associations between parent-adolescent conflict and youth maladjustment. Family conflict is also connected to heightened engagement in risky behavior \parencite{skinner:2016}. Elevated levels of conflict may increase the use of coercive strategies in parent-adolescent interactions \parencite{lobraico:2020}, where ASB emerges and stabilizes over time \parencite{granic:2006}.

Family cohesion is characterized by warmth, openness, emotional connection, and flexibility. Offspring in such families are found to have better psychological and behavioral adjustment \parencite{coe:2018, richmond:2006, sun:2021}. High and stable levels of cohesion make family members less adversely affected by PMD, adolescent ASB, or other life challenges \parencite{coe:2018}. Adolescents who feel connected to their family, are more likely to seek guidance, disclose information, and spend time with their families, leaving them with less opportunity to affiliate with delinquent and deviant peers \parencite{fosco:2019, vieno:2009}. Family cohesion tends to decrease through adolescent development and liberation process \parencite{baer:2002, dekovic:2003}. Decrease in family cohesion was lower and had less impact on youth who initially reported high connectedness to their family, while low levels in early adolescence predicted more delinquent behavior in later adolescence \parencite{lin:2019}. Other studies also found that low family cohesion predicted externalizing behavior as Conduct Disorder, Oppositional Defiant Disorder and hostility \parencite{coe:2018, richmond:2006}.

Depressed mothers report that their family environments more often are less cohesive and more conflict-filled, compared to non-affected mothers \parencite{slee:1996}. Higher levels of maternal depression is associated with lower levels of family cohesion, reported by both mothers and adolescents \parencite{perez:2018}. \textcite{fosco:2020} found that adolescents within families with high levels of cohesion, reported feeling more positive, more satisfied with life, and less angry, depressed, and anxious. Reflecting that family cohesion can function as a protective factor against life difficulties.

\subsection{SES (Agathe)}

\subsection{The Current Study}

The current study aims to investigate whether family conflict and cohesion mediate the effect of PMD on ASB. We hypothesize that higher symptoms of PMD will increase family conflict, and decrease family cohesion. Further, we expect that elevated levels of family conflict is related to increased adolescent ASB, while elevated levels of cohesion is associated with lower adolescent ASB. We also expect the mediators to covary, with high levels in one resulting in low levels in the other.
