\section{Methods}

\subsection{Participants}

The current study utilized clinical data from Norway's Functional Family Therapy (FFT) initiative \parencite{bjornebekk:2013}. Adolescents and caregivers from 159 families participated in the combined randomized control- and process-outcome-design which sought to treat moderate to severe antisocial behavior. We included in our sample
\begin{seriate}
    \item adolescents between 11 and 19 years of age, with
    \item manifested or being at risk of developing at least one of the following behavioral problems: aggressive (both verbal and physical) and violent behavior, delinquency with severe risk of future offences, vandalism, severe rule-breaking behavior at home, school or in the local community, and substance use.%
\end{seriate}
Our exclusion criteria applied to
\begin{seriate}
    \item adolescents with autism spectrum disorder,
    \item who possessed imminent risk of suicide or had recently experienced an acute psychotic episode. Additionally,
    \item home environments that were unsafe for the therapist,
    \item cases with ongoing investigation by the local child welfare service, and
    \item cases that already participated in interventions or treatments incompatible with FFT were also excluded.%
\end{seriate}

Two observations were removed due to complete missings, leading to a sample size of 157 adolescents (age $M = 14.74$, $SD = 1.47$, range = $[10.80, 17.88]$) and their primary caregivers (age $M = 43.93$, $SD = 6.90$, range = $[29, 78]$). There was a slightly higher proportion of male adolescents ($n = 85$, 52.1\%) than females ($n = 72$, 45.9\%) while percentages were in reverse for caregivers, with 89.8\% mothers and 10.2\% fathers ($n = 141$, $n = 16$) respectively. Most adolescents lived with single parents ($n=59$, 37.6\%), while the remaining lived with both parents, adoptive parents, or in foster care (see \cref{tab:demographic} for participants' socio-demographic attributes).

\subsection{Procedures}

Participants were surveyed at three time points: before being assigned to different groups (T1), after intervention/treatment (T2), and at a follow-up session one year later (T3). This paper studies retrospectively the pre-treatment data at T1. This archival cross-sectional design implies that relationships revealed by this study did not interact with subsequent intervention/treatment.

Parents and adolescents completed their respective questionnaires on portable computers installed with \textsf{Ci3} software \parencite{sawtooth:2013}. The participants completed the questionnaires in their homes, or at a municipality office. A research assistant was available for questions and gave general instructions on how to use the \textsf{Ci3} system. Families received a minor compensation for participation equivalent to 50 US dollars \parencite{thogersen:2020}.

\subsection{Measures}

% \subsubsection{Asolescent Antisocial Behavior (ASB)}

% Child behavior Checklist 6--18 \parencite[CBCL, ][]{achenbach:2001} was used to assess adolescent ASB. This questionnaire is one of the most used parental measures of emotional and behavioral problems among youth between 6 and 18 years old. Based on their adolescent's behavior over the past six months, primary caregivers filled out 113 CBCL items in 3-point Likert scales: 0 (not true), 1 (true or sometimes true), and 2 (very true or often true) \parencite{achenbach:2001}. Historically, CBCL has shown acceptable reliability and validity \parencite{achenbach:2001, naarking:2004, pandolfi:2014}, also in Norwegian samples \parencite{lurie:2006}. To operationalise the outcome variable ASB, we selected 35 items from the subscale ``externalising behavior'', consisting of two syndrome measures: ``aggressive behavior'' (e.g., Attacks other people physically) and ``rule-breaking behavior” (e.g., Breaks rules at home, at school, or other places) \parencite{achenbach:2001}. Parent-reported ASB has satisfactory reliability with externalising behavior (35 items; $\alpha = .92$), aggressive behavior (18 items; $\alpha = .92$), and rule-breaking behavior (17 items, $\alpha = .81$), respectively.

\subsubsection{Parental Mental Distress}

The 8-item version of the \textit{Hopkins Symptom Checklist} \parencite[originally SCL-90,][]{derogatis:1974} was used to measure parental mental distress \parencite[Norwegian translation of SCL-8,][]{fink:2004a} due to its brevity and high reliability \parencite{siqveland:2016}. Parents recalled their anxiety and depression symptoms over the past 14 days (e.g., ``Sudden fear without any clear reason'') on a 4-point Likert scale: 1 (Not bothered), 2 (Somewhat bothered), 3 (Very bothered) and 4 (Very much bothered). The SCL-8 scale contains only emotional symptoms, and is suggested to be a valid and robust screening tool \parencite[$\alpha = .91$,][]{fink:2004a, fink:2004b}. This study uses both the sum-score (PMH\_SUM, \cref{tab:descriptive}) and the latent construct of PMH (see \cref{tab:lat_PMH}).

% \subsubsection{Family Conflict and Cohesion}

% Family conflict and cohesion were measured using parental self-report of the Norwegian version of the Family Environment Scale (FES), which assesses the social environment of families along ten salient dimensions \parencite{moos:1976}. FES consists of 90 true/false items distributed onto ten subscales, with conflict and cohesion consisting of nine items each. Conflict is conceptualised as the amount of openly expressed anger and aggression, and how conflicted interactions are characteristics of the family (e.g., ``Family members often criticize each other''). The cohesion subscale is operationalized as the extent family members are concerned and committed to the family and the degree of support and helpfulness between family members (``Family members really help and support one another'') \parencite{moos:1976, lucia:2006}. Results are somewhat conflicted on the acceptable validity and reliability of FES \parencite{moos:1990, moos:2009, roosa:1990}. Our analysis found acceptable reliability for both the conflict and cohesion subscales ($\alpha = .76$ and $\alpha = .73$, respectively).

% \subsubsection{Control Variables}

% This study considered participants' socio-demographic characteristics as control variables. They include adolescents' age and gender, as well as parents' educational levels and perceived economic hardship. Lastly, the number of additional children was included to account for family size effect (see \cref{tab:demographic}).

% \subsubsection{Ethical Considerations}

% The current study received approval from the Regional Committees for Medical and Health Research Ethics (REK) to utilize data gathered by the study of Evaluation of FFT in Norway \parencite{bjornebekk:2013}. Both parents and adolescents participants gave their informed written consent. Consent forms included information about participants' right to withdraw from the study at any given time without any adverse consequences, and safeguard of participants' confidentiality. Participants consent forms were presented for Norwegian Center for Research Data (NSD) and Norwegian Data Protection Authority (Datatilsynet) \parencite{bjornebekk:2013}. All data were collected, stored, and processed within a certified secure IT environment \parencite[TSD, ][]{tsd:2020}.

% \subsubsection{Data Analysis}

% A mediation analysis is suitable to examine how or if one variable is related to another variable through some other variable. We proposed a structural equation model (SEM) with two mediators \parencite{mackinnon:2008, rucker:2011}. We firstly conducted series of preliminary analyses using \textsf{SPSS} 28, including descriptive statistics, missing values, and correlation studies. All variables appeared to be non-normal based on Shapiro-Wilks test: parental mental distress ($W = .92$, $p < .001$), adolescent ASB ($W = .98$, $p < .05$), family conflict ($W = .94$, $p < .001$), family cohesion ($W = .92$, $p < .001$), and economic hardship ($W = .88$, $p < .001$). Resultantly, we reported correlations in Spearman's $\rho$. Two observations were removed before analysis due to whole-row missings. Next, we carried out SEM analyses in \textsf{Mplus} \parencite[Version 8.3, ][]{muthen:2017} to examine direct and indirect effects among parental mental distress, adolescent ASB, family conflict, and cohesion. In order to account for missing data, ten imputed datasets were generated using \textsf{Mplus}'s unrestricted variance-covariance model \parencite[``JM-AM H1'',][]{asparouhov:2022}. The path between parental mental distress, family conflict, and adolescent ASB was controlled for by economic hardship (see \cref{tab:results}). We employed robust maximum likelihood (MLR) estimator due to its ability to handle non-normal data, as well as Bayes estimators for corroboration. Model fit was evaluated using SRMR. Standardized parameter estimates were used to assess the direct and indirect effects between variables.