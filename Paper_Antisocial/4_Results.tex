\section{Results}

% \subsection{Descriptive Statistics}

% Means, standard deviations, and correlations between all study variables are presented in Table 2. Due to non-significant correlations with the proposed control variables, adolescent age, gender, and parental educational level, they are not reported in text. However, economic hardship correlated with both parental mental distress ($\rho = .17$, $p = .031$, $95\% \text{ CI} = [.012, .326]$), and family conflict ($\rho = .20$, $p = .013$, $95\% \text{ CI} = [.037, .352]$). We therefore included economic hardship as a relevant variable. Correlations show that parental mental distress were significantly associated with adolescent ASB ($\rho = .42$, $p < .001$, $95\% \text{ CI} = [0.27, 0.55]$). Parental mental distress was significant with family conflict ($\rho = .46$, $p < .001$, $95\% \text{ CI} = [0.32, 0.48]$), and family cohesion ($\rho = .28$, $p < .001$, $95\% \text{ CI} = [-0.43, -0.12]$). Adolescent ASB was significant with family conflict ($\rho = .38$, $p < .001$, $95\% \text{ CI} = [0.23, 0.52]$), and cohesion ($\rho = -.24$, $p < .001$, $95\% \text{ CI} = [-0.39, -0.08]$). The two mediating variables correlated strongly ($\rho = -.45$, $p < .001$, $95\% \text{ CI} = [-0.57, -0.31]$).

% \subsection{Mediation Analysis}

% To investigate the effect of family conflict and cohesion on the relationship between parental mental distress and adolescent ASB, a multiple mediation analysis was performed using \textsf{Mplus}. The outcome variable for the analysis was adolescent ASB, while the predictor variable was parental mental distress. The two mediating variables were family conflict and cohesion. Due to sample size constraint, manifest rather than latent variables were utilized in the model. In this analysis we explicitly allow the two mediators to covary to account for their oriented dependence. Family conflict and cohesion had a significant negative covariance ($\hat{\rho} = -.37$, $SE = 0.07$, $p < .001$, $95\% \text{ CI} = [-0.51, -0.24]$). Model fit indices suggest good fit, considering the small sample size (RMSEA = 0.00, $p$ = .863, CFI = 1.00, TLI = 1.10).


% \subsection{Direct Effects}

Parental mental distress was significantly related to adolescent ASB ($\hat{\beta} = .29$, $SE = 0.08$, $p < .001$, $95\% \text{ CI} = [0.14, 0.44]$). As shown in \cref{fig:bayes}, the path between parental mental distress and family conflict was significant ($\hat{\beta} = .39$, $SE = 0.07$, $p < .001$, $95\% \text{ CI} = [0.24, 0.53]$), so was that from parental mental distress to family cohesion ($\hat{\beta} = -.30$, $SE = 0.07$, $p < .001$, $95\% \text{ CI} = [-0.44, -0.16]$). The path between family conflict and adolescent ASB was significant ($\hat{\beta} = .23$, $SE = 0.08$, $p < .001$, $95\% \text{ CI} = [0.13, 0.44]$), while to family cohesion was not ($\hat{\beta} = -.05$, $SE = 0.07$, $p = .459$, $95\% \text{ CI} = [-0.19, 0.09]$). We controlled for economic hardship, which was significant on parental mental distress ($\hat{\beta} = .18$, $SE = 0.09$, $p < .05$, $95\% \text{ CI} = [0.01, 0.35]$), family conflict ($\hat{\beta} = .13$, $SE = 0.06$, $p < .05$, $95\% \text{ CI} = [0.02, 0.24]$), but not on adolescent ASB ($\hat{\beta} = -.14$, $SE = 0.08$, $p = .061$, $95\% \text{ CI} = [-0.29, 0.01]$).

% \subsection{Indirect Effects}

% The indirect mediation of family conflict on parental mental distress and adolescent ASB was significant and positive ($\hat{\beta} = .11$, $SE = 0.04$, $p < .05$, $95\% \text{ CI} = [0.09, 0.43]$). In contrast, the indirect mediation path of family cohesion between parental mental distress and adolescent ASB failed to reach significance ($\hat{\beta} = .02$, $SE = 0.02$, $p = .461$, $95\% \text{ CI} = [-0.06, 0.14]$). The total indirect mediation, including both conflict and cohesion, showed a significant total indirect effect ($\hat{\beta} = .12$, $SE = 0.04$, $p < .001$, $95\% \text{ CI} = [0.05, 0.20]$).
