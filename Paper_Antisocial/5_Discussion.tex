\section{Discussion}

The purpose of this current study was to investigate whether family conflict and cohesion have a mediating role on the relationship between parental mental distress (PMD) and adolescent antisocial behavior (ASB). Results revealed a direct association between PMD and adolescent ASB, and is consistent with previous research. Indicating that PMD with all the possible behaviors or attitudes this measure includes, directly affects adolescents development and exhibition of ASB. Some mechanisms that might influence, but are not controlled for in the current study include parenting styles \parencite{hautmann:2015, vera:2012}, parental hostility and overprotection \parencite{sellers:2014}, and coping strategies \parencite{francisco:2015}, as well as environmental factors outside the family (KILDE). Further, our overall results indicate that family conflict has a mediating effect on adolescent ASB, while cohesion does not. Unsurprisingly, elevated and chronic patterns of family conflict, and within specific dyads, e.g. parent-adolescent, results in deteriorated family cohesion.

As to our first hypothesis, results indicate that elevated levels of PMD is associated with increased levels of family conflict, and reduction in family cohesion. These results are consistent with previous findings \parencite[e.g.,][]{garber:2005, perez:2018, xu:2017}. We assume that PMD negatively affects their ability to choose proactive and effective parenting strategies, as previous research has found that depressed caregivers use inconsistent discipline, initiate negative patterns of interactions, and lack monitoring \parencite{korhonen:2014,perez:2018}. Factors like these are possible explanations for why family environments with distressed caregivers may function as catalysts for adverse interaction patterns, resulting in chronic conflict-filled communication between family members \parencite{garber:2005}.  Hostile and conflict-filled interpersonal relationships can result in withdrawal by family members \parencite{romm:2022}. Hence, explaining the reduction in family cohesion when PMD is high. These results are also in line with previous research \parencite{li:2021, vanloon:2014}. It is reasonable to assume that within a clinical sample, with interactions characterized by higher conflict and PMD, any deterioration in family cohesion will escalate the situation.

As expected, results indicate that family conflict has a mediating role on the relationship between PMD and adolescent ASB, while cohesion does not. There are several explanations for why and how family conflict has an impact on the path to adolescent ASB. Parents with increased mental distress usually have reduced capacity and ability to engage in positive and favorable parenting \parencite{joyner:2021}. As depression and anxiety influence parenting styles characterized by control through guilt and overprotection, hostility, criticism, and inconsistent discipline \parencite{cummings:2005,korhonen:2014}. This may result in family environments characterized by coercive and hostile attitudes and behaviors. \textcite{lobraico:2020} found that adolescents in coercive families experienced the most robust risk across ASB outcomes. Families that engage in more hostile behaviors, in the form of fighting and aggression, may damage both trust and secure attachments between parent and adolescent \parencite{buehler:2006,weymouth:2016}. When this pattern of communication becomes normative within the family, offsprings may adopt and stabilize these attitudes to other social relations, encouraging affiliation with antisocial groups and peers \parencite{carroll:2009,ciranka:2021,moffitt:2015}. Conversely, research shows that living with antisocial or delinquent adolescents have transactional adverse consequences on PMD and family conflict \parencite{gross:2009}. For instance, having a teen not complying to rules, expressing hostile and aggressive behaviors, and parents' awareness that they engage in antisocial activities may create significant stress \parencite{allen:2010}.

Adolescence brings normative shifts in family relations, resulting in increased conflict and reduced cohesion between parent and youth, as they attempt to adjust boundaries, renegotiate parental authority, and increase their autonomy and independence \parencite{baer:2002,lin:2019,weymouth:2016}. Youth also tend to become more oppositional during adolescence \parencite{steinberg:2011}, which may exacerbate adverse patterns of communication and interaction. Also, parents modeling role on behaviors and attitudes are gradually replaced by peers. Developmental trends like these become more problematic for mentally distressed parents, compared to non-distressed. We assume that PMD might exacerbate their ability to meet and adjust to adolescent autonomy seeking, resulting in even more friction and conflict. Connectedness and youth self-disclosure are found to significantly enhance youths' prosperity to seek guidance when navigating difficulties, value parental input, and spend time with their families. Hence, leaving them with less opportunity to engage in ASB \parencite{ackard:2006,crawford:2008,vieno:2009}. Therefore, we assume that high conflict and lack of cohesion in our sample, contribute to the youth seeking affiliation with deviant peers and not their parents. Especially among mentally distressed parents, where rejection and love withdrawal are prominent. This may exacerbate the distance between parent and adolescent.

When controlling for economic hardship, we found that this had an influence on PMD and family conflict, but not on adolescent ASB. These findings suggest that economic hardship directly impacts parents. Previous research has found socioeconomic disadvantage to be a strong indicator of depressive symptoms in parents \parencite{conger:2010,sturgeapple:2014,vreeland:2019}. Therefore, we assume that living in economic disadvantage might place the parents under elevated stress, which further impair their parental practices and family climate. Further, PMD may also be a contributing factor to poorer employability, therefore more economic hardship. Resultantly, this stressor may be a reason for increased levels of family conflict within the family system, and have an indirect influence on adolescent ASB.

\subsection{Limitations}

This study has several limitations. Firstly, a consequence of small sample size is lack of power to detect statistical significance for the observed associations. Secondly, we only used parent-reported measures. This is problematic due to well documented discrepancies between parental and adolescents' reports on family environment and ASB \parencite[e.g.,][]{delosreyes:2011, robinson:2019}, introducing a potential reporting bias (Allen et al., 2010). Parents and adolescents may interpret and observe each other's behaviors differently, therefore, research should attempt to include the offspring's perspectives. Further, cross-sectional data prevents us from drawing any causal conclusions. Our findings are an artifact of our modelling choices, reflecting that results could be different using other methods and samples. For example, compared to the general population, a clinical sample usually has higher levels of symptoms, with in turn affects the generalization of our findings. The current study provides a small ``snapshot'' of a bigger picture. However, this still contributes to research, as many small ``snapshots'' jointly inform the full picture.

\subsection{Implications and Future Research}

Findings from the current study have various practical implications. This study provides insight and confirmation of previous research on the association between family mechanisms, PMD and adolescent ASB. This is important when establishing holistic interventions, targeting environmental factors and parents' psychopathology. Results suggest that family interaction patterns, such as conflict and cohesion, have significant and distinct influences on interpersonal relationships, feelings and behaviors among family members. Further research should seek to use multi-informants, youths' perspectives, and differentiate by gender when examining relations between interpersonal and environmental constructs.
