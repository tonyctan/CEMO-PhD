\documentclass[
% Options for apa7
    a4paper,                % Paper size
    jou,                    % Journal format
    twoside,                % Two-sided printing
    % donotrepeattitle,       % Start body text without repeating title
    floatsintext,           % Insert tables and figures with texts
    biblatex,               % Use BibLaTeX for references
% Options for hyperref
    colorlinks=true,        % Colour all links
    linkcolor=red,          % Cross-references in red
    anchorcolor=black,      % Keep anchors black
    citecolor=blue,         % In-text-referencs in blue
    urlcolor=blue,          % DOIs and URLs are in blue
    bookmarks=true,         % Generate bookmarks for PDF readers
    bookmarksopen=false,    % Expand all bookmarks as default
    bookmarksnumbered=true, % Keep section number in bookmarks
    % Options for xcolor
    dvipsnames              % Use colour BrickRed and PineGreen
]{apa7}

\usepackage{adjustbox}

% Specify absolute path to the tt.sty file, depending on the operating system
\usepackage{ifplatform}
\ifwindows
    \usepackage{M:/pc/Dokumenter/tt}
\fi
\iflinux
    \usepackage{/home/tony/uio/pc/Dokumenter/tt} % Must not use ~ for home directory. Spell in full.
\fi
\ifmacosx
    \usepackage{/Users/tctan/uio/pc/Dokumenter/tt}
\fi

% Specify project's bib library
\addbibresource{./Bibliography/Antisocial.bib}

\setlength\parindent{0.4cm}

\title{Parental Mental Distress and Adolescent Antisocial Behavior:\\
\vspace*{1mm}
The Mediating Role of Family Conflict and Cohesion}
\shorttitle{Family Dynamics and Adolescent Antisocial Behavior}
\leftheader{Skancke, Mausethagen, Tan, {Backer-Gr{\o}ndahl}, \& Bj{\o}rnebekk (2024)}
\journal{\copyright American Psychological Association}
\volume{ISSN: 0033-2909}
\ccoppy{Psychological Bulletin}
\copnum{2024, Vol. 150, No. 2, 91--117\\
\href{https://doi.org/10.1037/bul0000447}{http://dx.doi.org/10.1037/bul0000447}}

\authorsnames[1,1,2,3,{1,3}]{
    Frida T. Skancke,
    Thea Fahle Mausethagen,
    \correspondingauthor{Tony C. A. Tan},\\
    Agathe {Backer-Gr{\o}ndahl},
    Gunnar Bj{\o}rnebekk%
}

\authorsaffiliations{
    {Department of Special Education, University of Oslo},
    {Centre for Educational Measurement, University of Oslo},
    {Norwegian Center for Child Behavioral Development, Oslo, Norway}
}

\authornote{
    \addORCIDlink{Frida T. Skancke}{0000-0002-4374-7038}
    \addORCIDlink{Thea Fahle Mausethagen}{0000-0002-4532-4370}
    \addORCIDlink{Tony C. A. Tan}{0000-0001-6632-3791}
    \addORCIDlink{Agathe {Backer-Gr{\o}ndahl}}{0000-0003-0933-6531}
    \addORCIDlink{Gunnar Bj{\o}rnebekk}{0000-0003-2176-7393}

Correspondence concerning this article should be addressed to Tony Tan, Centre for Educational Measurement, University of Oslo, P.O. Box 1161 Forskningsparken, 0318 Oslo, Norway. E-mail: \href{mailto:tctan@uio.no}{tctan@uio.no}.
}

\abstract{%
Parental mental distress can have a significant impact on adolescents' antisocial behavior (ASB), and this relationship can be exacerbated or alleviated by family dynamics such as conflict and cohesion. By analyzing responses to clinical questionnaires from 157 Norwegian adolescents and their primary caregivers, this study found a significant mediating role family conflict played in strengthening the relationship between parental mental distress and adolescents' ASB. Family cohesion, on the other hand, asserted insufficient protective effect. Both results are consistent with the family stress model, and highlight the importance of addressing not only individuals' mental health but also family-level factors in preventing and treating adolescent ASB. This insight is robust against measurement errors through the use of Bayesian structural equation models.
}

\keywords{%
adolescent antisocial behavior, parental mental distress, family conflict, family cohesion, mediation effect, family stress model, Bayesian structural equation model

\bigskip

\textit{Supplemental materials:} \href{https://doi.org/10.1037/bul0000447.supp}{http://dx.doi.org/10.1037/bul0000447.supp}
}
\begin{document}
\maketitle

\setcounter{page}{91}

%\section{Introduction} % Do not include the word "Introduction" as a Level 1 heading. Just start intro as a normal paragraph immediately after the title.

The grade point average (GPA, \textit{skolepoeng} in Norwegian) plays a key role in Norway's educational assessment process. From Year 8 onwards, Norwegian high schoolers receive formal grades from both their teachers (\textit{standpunktkarakter}) and year-end exams \parencite{raeder:2020}, which are used for high-stake decisions such as graduate certifications (Year 10) and university admissions (VG3). Since different subjects are treated \emph{equally} in its calculation \parencite{gpa:2021}, GPA implicitly assumes that grades across different specialities are \emph{equivalent} indicators of students' preparedness for the next stage of education --- an assumption that remains untested and questioned by descriptive statistics \parencite{udir:2022}.

Concerns for the comparability of subject difficulties are further deepened by prior studies from education systems similar to that of Norway's. \textcite{he:2018} in the UK and \textcite{korobko:2008} in the Netherlands both reported persistent disagreements among subject difficulties, which may lead to differential treatment of students with different specialisations. Besides fairness concerns, the lack of difficulty comparability also leads to a lack of construct validity \parencite{messick:1989} in GPA, as the construct-irrelevant variance related to subject characteristics, in addition to candidates' competencies, have been included in the GPA calculation. Understanding the presence, directions, and magnitudes of inter-subject difficulties therefore becomes a key issue for assessing the validity of GPA. By analysing Norway's education record data, this study aims to test GPA's validity as a measurement scale in mapping candidate competence into numeric scores, as well as the fairness consequences subsequent to its use in high-stake situations. More specifically, this study will address the following research questions:
\begin{ttitemize}
    \item[\textbf{RQ1:}] Do Norwegian Year 10 subjects differ in their difficulty levels?
    \item[\textbf{RQ2:}] Do subject difficulties differ by source such as between teachers and external examiners?
    \item[\textbf{RQ3:}] Do subject difficulties differ across achievement levels?
    \item[\textbf{RQ4:}] Do subject difficulties differ across demographic attributes such as socioeconomic status, gender and immigration background?
\end{ttitemize}  % Use \input{} for Intro.tex so that the title and the first paragraph stay on the same page.

% \section{Conceptual Framework}

\subsection{The Norwegian Education and Assessment System}

The Norwegian education system is organised into three levels: primary school (Year 1--7) where formal grading is not practised, lower secondary school (Year 8--10) and upper secondary school (Year 11--13). During the first ten years of schooling (\textit{grunnskole}), students follow centralised national curricula with largely compulsory subjects plus some electives. Upon successful completion of Year 10, students may choose between vocational and academic tracks for their upper secondary schools. The former is a two-year program that prepares students for employment in a specific field, whereas the latter is a three-year program (\textit{videreg{\aa}ende oppl{\ae}ring}, VG1--3) that prepares students for university studies.

The grade point average (\textsc{gpa}) aims to provide a sum-score measure of a student's overall competency. For \textit{grunnskole} graduation purposes, the \textsc{gpa} is calculated as the unweighted average of students' grades from all Year 10 subjects. Both teacher-assigned grades and exam grades are included in the GPA calculation, with each subject ranging from 1 (low competency) to 6 (outstanding). While every compulsory subject receives a teacher-assigned grade, Year 10 students are randomly assigned into participating in \emph{one} of the three written exams (mathematics, Norwegian, and English), as well as \emph{one} oral exam (same as written exams, plus many electives). A candidate's GPA is then computed by averaging the grades they have obtained, multiplying by 10, and rounding to two decimal places.

\subsection{The \textit{manu}--\textit{mente} Clusters}

Not all \textsc{gpa} subjects target the same cognitive domain. While all subject demand students' cognitive input, some courses are undoubtedly more hand-on and practice-based such as physical education and food and health. We label the more hand-on subject in Latin as ``\textit{manu} subjects'' while the cognitive-demanding ones as ``\textit{mente} subjects''. [insert references suggested by Jose].

\section{Methods}

\subsection{Population}

This study selects the entire cohort of Norwegian Year 10 students graduating in 2019 as its targeted population. Students' GPA (\textit{grunnskolepoeng}), teacher-assigned grades (\textit{standpunktkarakter}), as well as written (\textit{SKR}) and oral (\textit{MUN}) exam grades were extracted from the Norwegian register. This data source is unique because it is the \emph{population}, not a sample, that is the subject of our study.

Attainment records were subsequently re-formatted with each row representing one candidate and each column being one subject, leading to 64,918 students and 200 subjects. Students without valid GPA records were removed from the data set (data loss $n^-= 4,300$ cases, loss rate $r^- = 6.62\%$).

Under the Norwegian education system, students shall complete 13 compulsory subjects as well as electives. For teacher-assigned grades, this study focuses on these compulsory subjects with one exclusion ``Sidem{\aa}l''.\footnote{Norway has two official written language forms: Bokm{\aa}l and Nynorsk, with the former being the dominant form in national media. Students growing up in one written form must enroll the other as their Sidem{\aa}l, unless Norwegian is not their native language. Nynorsk users tend to have easier time in Sidem{\aa}l due to existing exposure to Bokm{\aa}l. Bokm{\aa}l users, on the other hand, find Nynorsk more challenging while fulfilling Sidem{\aa}l. Since this subject contains two sub-cohorts with very different difficulty profiles, we opt not to include Sidem{\aa}l in our analyses.} We applied equal treatment to courses instructed in Norwegian and in Sami language by merging these records.\footnote{If academic results from both instruction languages were available, we retain the higher grades during merging.} Twelve teacher-assigned grades were retained for our analysis: Written Norwegian (NORW), Oral Norwegian (NORO), Written English (ENGW), Oral English (ENGO), Mathematics (MATH), Natural Sciences (NATS), Social Sciences (SOCS), Religion (RELI), Music (MUSI), Arts and Handcraft (HAND), Food and Health (FOOD), and Physical Education (PHED).

At this stage, the existence of missing data no longer poses any problems for our analyses thanks to sufficient overlap across subjects in the score matrix. The ability to deal with incomplete data is one major advantage of using the Rasch model for studying inter-subject comparability \parencite{he:2018}.

\subsubsection{Generalised Partial Credit Model (GPCM)}

A unidimensional generalised partial credit model \parencite{muraki:1992} with the probability that Candidate $n$'s score in Subject $i$ ($x_{ni}$) being Grade $j$ ($j=0, \dots, m$) is given by
\begin{equation}\label{eqn:gpcm}
    p \left( x_{ni}=j | d_{ni} = 1; \theta_n \right) = \frac{\exp{j \alpha_i \theta_n - \sum_{h=1}^j \beta_{ih}} }{ 1 + \sum_{h=1}^m \exp{h \alpha_i \theta_n - \sum_{k=1}^h \beta_{ik}} },
\end{equation}
where $\theta_n$ is the unidimensional proficiency parameter that represents the overall proficiency of Candidate $n$.

\subsubsection{Log-likelihood}

In MML, a likelihood function ($\ell$) is maximised where the candidates' proficiency parameters ($\theta$) are integrated out of the likelihood. The marginal log-likelihood for a unidimensional GPCM is given by
\begin{equation}\label{eqn:ll}
    \ell_\text{unidimensional} = \sum_p \sum_{n | p} \log \int \prod_i p( x_{ni} | d_{ni}; \theta ) g(\theta; \mu_p, \sigma^2) \dd \theta,
\end{equation}
where $x_{ni}$ is the observed grade, $p( \cdot )$ is equal to \cref{eqn:gpcm} evaluated at $x_{ni}$ if $d_{ni}=1$, and $p( \cdot ) = 1$ if $d_{ni} = 0$. In addition, $g(\theta; \mu_p, \sigma^2)$ is the normal pdf with mean $\mu_p$ and variance $\sigma^2$. The model can be identified by choosing a standard normal $\mathcal{N}(0,1)$ \parencite{korobko:2008}.

\subsubsection{Multidimensionality}

There exists strong believes among educational scientists that learners' proficiency is multidimensional, such as one proficiency factor for STEM subjects, for example, and another one for languages. If $F$ proficiency dimensions are required to model the grades, the proficiency can be represented by a vector of proficiency parameters $\m{\theta}_n=\left(\theta_{n1}, \cdots, \theta_{nF}\right)\Ts$ with the corresponding GPCM:
\begin{equation}\label{eqn:mgpcm}
    p \left( x_{ni} = j | d_{ni} = 1; \m{\theta}_n \right) =
    \frac{ \exp{ j \left( \sum_{f=1}^F \alpha_{if} \theta_{nf} \right) - \sum_{h=1}^j \beta_{ih} } }{ 1 + \sum_{h=1}^m \exp{ h \left( \sum_{f=1}^F \alpha_{if} \theta_{nf} \right) - \sum_{k=1}^h \beta_{ik} } }.
\end{equation}
with $\m{\theta}_n$ following a multivariate normal distribution with mean $\m{\mu}_p$ and variance-covariance matrix $\m{\Sigma}$. Similar to the unidimensional case, \cref{eqn:mgpcm} is identified by setting $\m{\mu}_p=\m{0}$ and $\m{\Sigma} = \m{I}$ the identity matrix. The log-likelihood of a multidimensional GPCM then becomes:
\begin{equation}\label{eqn:mll}
    \ell_\text{multidimensional} = \sum_{p} \sum_{n | p} \log \int \cdots \int \prod_i p( x_{ni} | d_{ni}; \m{\theta} ) g( \m{\theta}; \m{\mu}_p, \m{\Sigma} ) \dd \m{\theta},
\end{equation}
with each component sharing similar interpretations to the unidimensional counterpart in \cref{eqn:ll}.

\subsection{Interaction between Subject Choice and Proficiency}

Under the advisory of \textcite{korobko:2008}, a latent variable $\theta^+$ is introduced to reflect student's propensity of choosing a particular subject. Augmenting $\theta^+$ to $\m{\theta}=\left(\theta_1, \cdots, \theta_F\right)\Ts$ yields $\m{\theta}^+=\left(\theta_1, \cdots, \theta_F, \theta^+\right)\Ts$, with a corresponding marginal likelihood:
\begin{equation}\label{eqn:lli}
    \ell_\text{interaction} = \sum_p \sum_{ n | p} \log \int \cdots \int \prod_i \left[ p \left( x_{ni} | d_{ni}; \m{\theta} \right) p \left( d_{ni}; \theta^+ \right) \right] g(\m{\theta}^+; \m{\mu}_p, \m{\Sigma}) \dd \m{\theta}^+ .
\end{equation}



\section{Results}

% \subsection{Descriptive Statistics}

% Means, standard deviations, and correlations between all study variables are presented in Table 2. Due to non-significant correlations with the proposed control variables, adolescent age, gender, and parental educational level, they are not reported in text. However, economic hardship correlated with both parental mental distress ($\rho = .17$, $p = .031$, $95\% \text{ CI} = [.012, .326]$), and family conflict ($\rho = .20$, $p = .013$, $95\% \text{ CI} = [.037, .352]$). We therefore included economic hardship as a relevant variable. Correlations show that parental mental distress were significantly associated with adolescent ASB ($\rho = .42$, $p < .001$, $95\% \text{ CI} = [0.27, 0.55]$). Parental mental distress was significant with family conflict ($\rho = .46$, $p < .001$, $95\% \text{ CI} = [0.32, 0.48]$), and family cohesion ($\rho = .28$, $p < .001$, $95\% \text{ CI} = [-0.43, -0.12]$). Adolescent ASB was significant with family conflict ($\rho = .38$, $p < .001$, $95\% \text{ CI} = [0.23, 0.52]$), and cohesion ($\rho = -.24$, $p < .001$, $95\% \text{ CI} = [-0.39, -0.08]$). The two mediating variables correlated strongly ($\rho = -.45$, $p < .001$, $95\% \text{ CI} = [-0.57, -0.31]$).

% \subsection{Mediation Analysis}

% To investigate the effect of family conflict and cohesion on the relationship between parental mental distress and adolescent ASB, a multiple mediation analysis was performed using \textsf{Mplus}. The outcome variable for the analysis was adolescent ASB, while the predictor variable was parental mental distress. The two mediating variables were family conflict and cohesion. Due to sample size constraint, manifest rather than latent variables were utilized in the model. In this analysis we explicitly allow the two mediators to covary to account for their oriented dependence. Family conflict and cohesion had a significant negative covariance ($\hat{\rho} = -.37$, $SE = 0.07$, $p < .001$, $95\% \text{ CI} = [-0.51, -0.24]$). Model fit indices suggest good fit, considering the small sample size (RMSEA = 0.00, $p$ = .863, CFI = 1.00, TLI = 1.10).


% \subsection{Direct Effects}

% Parental mental distress was significantly related to adolescent ASB ($\hat{\beta} = .29$, $SE = 0.08$, $p < .001$, $95\% \text{ CI} = [0.14, 0.44]$). As shown in \cref{fig:bayes}, the path between parental mental distress and family conflict was significant ($\hat{\beta} = .39$, $SE = 0.07$, $p < .001$, $95\% \text{ CI} = [0.24, 0.53]$), so was that from parental mental distress to family cohesion ($\hat{\beta} = -.30$, $SE = 0.07$, $p < .001$, $95\% \text{ CI} = [-0.44, -0.16]$). The path between family conflict and adolescent ASB was significant ($\hat{\beta} = .23$, $SE = 0.08$, $p < .001$, $95\% \text{ CI} = [0.13, 0.44]$), while to family cohesion was not ($\hat{\beta} = -.05$, $SE = 0.07$, $p = .459$, $95\% \text{ CI} = [-0.19, 0.09]$). We controlled for economic hardship, which was significant on parental mental distress ($\hat{\beta} = .18$, $SE = 0.09$, $p < .05$, $95\% \text{ CI} = [0.01, 0.35]$), family conflict ($\hat{\beta} = .13$, $SE = 0.06$, $p < .05$, $95\% \text{ CI} = [0.02, 0.24]$), but not on adolescent ASB ($\hat{\beta} = -.14$, $SE = 0.08$, $p = .061$, $95\% \text{ CI} = [-0.29, 0.01]$).

% \subsection{Indirect Effects}

% The indirect mediation of family conflict on parental mental distress and adolescent ASB was significant and positive ($\hat{\beta} = .11$, $SE = 0.04$, $p < .05$, $95\% \text{ CI} = [0.09, 0.43]$). In contrast, the indirect mediation path of family cohesion between parental mental distress and adolescent ASB failed to reach significance ($\hat{\beta} = .02$, $SE = 0.02$, $p = .461$, $95\% \text{ CI} = [-0.06, 0.14]$). The total indirect mediation, including both conflict and cohesion, showed a significant total indirect effect ($\hat{\beta} = .12$, $SE = 0.04$, $p < .001$, $95\% \text{ CI} = [0.05, 0.20]$).


\section{Discussion}

The purpose of this current study was to investigate whether family conflict and cohesion have a mediating role on the relationship between parental mental distress (PMD) and adolescent antisocial behavior (ASB). Results revealed a direct association between PMD and adolescent ASB, and is consistent with previous research. Indicating that PMD with all the possible behaviors or attitudes this measure includes, directly affects adolescents development and exhibition of ASB. Some mechanisms that might influence, but are not controlled for in the current study include parenting styles \parencite{hautmann:2015, vera:2012}, parental hostility and overprotection \parencite{sellers:2014}, and coping strategies \parencite{francisco:2015}, as well as environmental factors outside the family (KILDE). Further, our overall results indicate that family conflict has a mediating effect on adolescent ASB, while cohesion does not. Unsurprisingly, elevated and chronic patterns of family conflict, and within specific dyads, e.g. parent-adolescent, results in deteriorated family cohesion.

As to our first hypothesis, results indicate that elevated levels of PMD is associated with increased levels of family conflict, and reduction in family cohesion. These results are consistent with previous findings \parencite[e.g.,][]{garber:2005, perez:2018, xu:2017}. We assume that PMD negatively affects their ability to choose proactive and effective parenting strategies, as previous research has found that depressed caregivers use inconsistent discipline, initiate negative patterns of interactions, and lack monitoring \parencite{korhonen:2014,perez:2018}. Factors like these are possible explanations for why family environments with distressed caregivers may function as catalysts for adverse interaction patterns, resulting in chronic conflict-filled communication between family members \parencite{garber:2005}.  Hostile and conflict-filled interpersonal relationships can result in withdrawal by family members \parencite{romm:2022}. Hence, explaining the reduction in family cohesion when PMD is high. These results are also in line with previous research \parencite{li:2021, vanloon:2014}. It is reasonable to assume that within a clinical sample, with interactions characterized by higher conflict and PMD, any deterioration in family cohesion will escalate the situation.

As expected, results indicate that family conflict has a mediating role on the relationship between PMD and adolescent ASB, while cohesion does not. There are several explanations for why and how family conflict has an impact on the path to adolescent ASB. Parents with increased mental distress usually have reduced capacity and ability to engage in positive and favorable parenting \parencite{joyner:2021}. As depression and anxiety influence parenting styles characterized by control through guilt and overprotection, hostility, criticism, and inconsistent discipline \parencite{cummings:2005,korhonen:2014}. This may result in family environments characterized by coercive and hostile attitudes and behaviors. \textcite{lobraico:2020} found that adolescents in coercive families experienced the most robust risk across ASB outcomes. Families that engage in more hostile behaviors, in the form of fighting and aggression, may damage both trust and secure attachments between parent and adolescent \parencite{buehler:2006,weymouth:2016}. When this pattern of communication becomes normative within the family, offsprings may adopt and stabilize these attitudes to other social relations, encouraging affiliation with antisocial groups and peers \parencite{carroll:2009,ciranka:2021,moffitt:2015}. Conversely, research shows that living with antisocial or delinquent adolescents have transactional adverse consequences on PMD and family conflict \parencite{gross:2009}. For instance, having a teen not complying to rules, expressing hostile and aggressive behaviors, and parents' awareness that they engage in antisocial activities may create significant stress \parencite{allen:2010}.

Adolescence brings normative shifts in family relations, resulting in increased conflict and reduced cohesion between parent and youth, as they attempt to adjust boundaries, renegotiate parental authority, and increase their autonomy and independence \parencite{baer:2002,lin:2019,weymouth:2016}. Youth also tend to become more oppositional during adolescence \parencite{steinberg:2011}, which may exacerbate adverse patterns of communication and interaction. Also, parents modeling role on behaviors and attitudes are gradually replaced by peers. Developmental trends like these become more problematic for mentally distressed parents, compared to non-distressed. We assume that PMD might exacerbate their ability to meet and adjust to adolescent autonomy seeking, resulting in even more friction and conflict. Connectedness and youth self-disclosure are found to significantly enhance youths' prosperity to seek guidance when navigating difficulties, value parental input, and spend time with their families. Hence, leaving them with less opportunity to engage in ASB \parencite{ackard:2006,crawford:2008,vieno:2009}. Therefore, we assume that high conflict and lack of cohesion in our sample, contribute to the youth seeking affiliation with deviant peers and not their parents. Especially among mentally distressed parents, where rejection and love withdrawal are prominent. This may exacerbate the distance between parent and adolescent.

When controlling for economic hardship, we found that this had an influence on PMD and family conflict, but not on adolescent ASB. These findings suggest that economic hardship directly impacts parents. Previous research has found socioeconomic disadvantage to be a strong indicator of depressive symptoms in parents \parencite{conger:2010,sturgeapple:2014,vreeland:2019}. Therefore, we assume that living in economic disadvantage might place the parents under elevated stress, which further impair their parental practices and family climate. Further, PMD may also be a contributing factor to poorer employability, therefore more economic hardship. Resultantly, this stressor may be a reason for increased levels of family conflict within the family system, and have an indirect influence on adolescent ASB.

\subsection{Limitations}

This study has several limitations. Firstly, a consequence of small sample size is lack of power to detect statistical significance for the observed associations. Secondly, we only used parent-reported measures. This is problematic due to well documented discrepancies between parental and adolescents' reports on family environment and ASB \parencite[e.g.,][]{delosreyes:2011, robinson:2019}, introducing a potential reporting bias (Allen et al., 2010). Parents and adolescents may interpret and observe each other's behaviors differently, therefore, research should attempt to include the offspring's perspectives. Further, cross-sectional data prevents us from drawing any causal conclusions. Our findings are an artifact of our modelling choices, reflecting that results could be different using other methods and samples. For example, compared to the general population, a clinical sample usually has higher levels of symptoms, with in turn affects the generalization of our findings. The current study provides a small ``snapshot'' of a bigger picture. However, this still contributes to research, as many small ``snapshots'' jointly inform the full picture.

\subsection{Implications and Future Research}

Findings from the current study have various practical implications. This study provides insight and confirmation of previous research on the association between family mechanisms, PMD and adolescent ASB. This is important when establishing holistic interventions, targeting environmental factors and parents' psychopathology. Results suggest that family interaction patterns, such as conflict and cohesion, have significant and distinct influences on interpersonal relationships, feelings and behaviors among family members. Further research should seek to use multi-informants, youths' perspectives, and differentiate by gender when examining relations between interpersonal and environmental constructs.


\subsection{Data Availability}

\subsection{Acknowledgments}

\subsection{Author Contributions}

\subsection{Ethical Considerations}

To ensure acceptable principles of ethical and professional conduct, the current study received approval from Regional Committees for Medical and Health Research Ethics (REK) to utilize data gathered by the study of Evaluation of Functional Family Therapy in Norway \parencite{bjornebekk:2013}. All participants, both parents and adolescents gave written informed consent. Consent forms included information about participants' right to withdraw from the study at any given time, and ensured participants confidentiality. Participants consent forms were presented for Norwegian Center for Research Data (NSD) and Norwegian Data Protection Authority [Datatilsynet] \parencite{bjornebekk:2013}. All data were collected, stored, and processed within a certified secure IT environment \parencite[TSD,][]{tsd:2020}.

\printbibliography

% Figures
\onecolumn
\ttptikz{fig:bayes}{Structural Equation Model Predicting Youth's Antisocial Behavior}{
\begin{tikzpicture}[
    manvar/.style={rectangle,draw=black,minimum width=1.5cm},
    latvar/.style={ellipse,draw=black,minimum width=1.5cm,minimum height=1cm},
    convar/.style={rectangle,draw=black!25!white,minimum width=1.5cm},
    mean/.style={fill=black!10!white,regular polygon,regular polygon sides=3},
    ->,>=stealth',semithick,
    bend angle=-45,
    decoration={
        zigzag,
        amplitude=1pt,
        segment length=1mm,
        post=lineto,
        post length=4pt
    }
]

% Set parental mental health (input variable, X)
\node[latvar] (X) at (0,0) {PMH};

% Set family conflict and family cohesion (mediators, M)
\node[latvar] (M1) at (5,2.5) {CON};
\node[latvar] (M2) at (5,-2.5) {COH};

% Set antisocial behaviour (outcome variable, Y)
\node[latvar] (Y) at (13,0) {ASB};

% Link X to M
\draw[->,line width=2*0.500mm] (X.east) to node[above,sloped] {\press{0.502}{***}{0.074}} (M1.west);
\draw[->,line width=2*0.434mm] (X.east) to node[above,sloped] {\press{-0.434}{***}{0.080}} (M2.west);

% Lind M to Y
\draw[->,line width=2*0.512mm] (M1.east) to node[above,sloped] {\press{0.512}{**}{0.143}} (Y.west);
\draw[->,dashed] (M2.east) to (Y.west);

% Link X to Y
\draw[->,line width=2*0.347mm] (X.east) to node[above,pos=0.72] {\press{0.347}{***}{0.108}} (Y.west);

% Covariance between M1 and M2
\draw[black!25!white,<->,line width=2*0.530mm] (M1.south) to [bend right] (M2.north);

% Control variable (HAR)
\node[manvar] (S) at (13,-2.5) {HAR};

% Link SES to Y
\draw[->,line width=2*0.210mm] (S.north) to node[above,sloped] {\press{-0.210}{**}{0.083}} (Y.south);

\end{tikzpicture}
}{This structural equation model predicts youth's antisocial/externalizing behavior (ASB) from parental mental health (PMH), with mediating effects from family conflict (CON) and family cohesion (COH). Variables in ellipses are latent constructs (see \cref{app:latent}). Standardized regression coefficients are computed using Bayes estimator \parencite{depaoli:2021} and averaged over ten imputed datasets \parencite{little:2020, vanbuuren:2018}. Solid lines are visualized in proportion to their estimates while dashed lines represent nonsignificant relations at $\alpha=.05$ level. Parental perception of economic hardship (HAR) is only significantly related to the outcome variable ASB. All control variables are nonsignificant and are omitted from the diagram.\\
    $^* p < .05$. $^{**} p < .01$. $^{***} p < .001$.
}

% Tables

\ttptable{tab:demographic}{Socio-demographic Characteristics of Participants}{
    \begin{tabular}{lrcc}
        \toprule
        \multicolumn{1}{c}{Sample Characteristics} & \multicolumn{1}{c}{$n$} & \multicolumn{1}{c}{$M$ ($SD$)} & \multicolumn{1}{c}{Missing} \\
        \midrule
        \rowcolor[rgb]{ .949,  .949,  .949} Adolescents' Age & 157   & 14.74 (1.47) &  \\
        \rowcolor[rgb]{ .949,  .949,  .949} Adolescents' Gender & 157   &       &  \\
        \hspace{0.75cm} Female & 72    &       &  \\
        \hspace{0.75cm} Male & 85    &       &  \\
        \rowcolor[rgb]{ .949,  .949,  .949} Adolescents' Living Condition & 153   &       & 2.50\% \\
        \hspace{0.75cm} Category 1: Living at home with their parents & 40    &       &  \\
        \hspace{0.75cm} Category 2: Living partly with both parents & 8     &       &  \\
        \hspace{0.75cm} Category 3: Living mainly with one parent who have no new partner & 59    &       &  \\
        \hspace{0.75cm} Category 4: Living mainly with one parent who has a new partner & 36    &       &  \\
        \hspace{0.75cm} Category 5: Being adopted or living in foster care & 10    &       &  \\
        &&&\\
        \rowcolor[rgb]{ .949,  .949,  .949} Parental Age & 157   & 43.9 (6.90) &  \\
        \rowcolor[rgb]{ .949,  .949,  .949} Parental Gender & 157   &       &  \\
        \hspace{0.75cm} Mother & 141   &       &  \\
        \hspace{0.75cm} Father & 16    &       &  \\
        \rowcolor[rgb]{ .949,  .949,  .949} Parental Education Level & 156   &       & 0.60\% \\
        \hspace{0.75cm} Level 1: Primary and secondary school (< 10 years) & 23    &       &  \\
        \hspace{0.75cm} Level 2: Upper secondary school (11--14 years) & 67    &       &  \\
        \hspace{0.75cm} Level 3: Higher education (> 14  year) & 66    &       &  \\
        \rowcolor[rgb]{ .949,  .949,  .949} Perceived Economic Hardship & 156   &       & 0.60\% \\
        \hspace{0.75cm} Level 1: Living comfortably & 12    &       &  \\
        \hspace{0.75cm} Level 2: Doing alright & 43    &       &  \\
        \hspace{0.75cm} Level 3: Just about getting it & 76    &       &  \\
        \hspace{0.75cm} Level 4: Finding it quite difficult & 15    &       &  \\
        \hspace{0.75cm} Level 5: Finding it very difficult & 10    &       &  \\
        &&&\\
        \rowcolor[rgb]{ .949,  .949,  .949} Additional Children in the Family & 157   & 1.25 (0.99) &  \\
        \bottomrule
    \end{tabular}%
}{Categorical variables are coded using their category/level numbers. The means ($M$) and standard deviations ($SD$) of non-categorical variables are measured in their natural units (e.g., years of age).}


\ttltable{tab:descriptive}{Descriptive Statistics for the 18 GPA Subjects}{
    \begin{tabular}{clrrrrrrrrrl}
        \toprule
    Subject & \multicolumn{1}{c}{Subject} & \multicolumn{1}{c}{Valid} & \multicolumn{1}{c}{Missing} & \multicolumn{1}{c}{Missing} & \multicolumn{6}{c}{Grade Frequency}           & \multicolumn{1}{c}{\textsc{udir}} \\
    \cmidrule{6-11}    Code  & \multicolumn{1}{c}{Name} & \multicolumn{1}{c}{Entries} & \multicolumn{1}{c}{($n$)} & \multicolumn{1}{c}{($\%$)} & \multicolumn{1}{c}{1} & \multicolumn{1}{c}{2} & \multicolumn{1}{c}{3} & \multicolumn{1}{c}{4} & \multicolumn{1}{c}{5} & \multicolumn{1}{c}{6} & \multicolumn{1}{c}{Course Code} \\
        \midrule
        \multicolumn{12}{l}{\textbf{Teacher-assigned Grades} (12 subjects)}\\
    \textsc{math}  & Mathematics & 59,184 & 1,434 & 2.37  & 1,165 & 10,086 & 15,447 & 15,443 & 12,520 & 4,523 & \href{https://www.udir.no/lk20/fagkoder/mat0010}{\textsc{mat0010}} \\
    \textsc{norw}  & Written Norwegian & 58,946 & 1,672 & 2.76  & 498   & 5,504 & 15,503 & 20,482 & 13,888 & 3,071 & \href{https://www.udir.no/lk20/fagkoder/nor0214}{\textsc{nor0214}} \& \href{https://www.udir.no/lk20/fagkoder/nor0041}{\textsc{nor0041}} \\
    \textsc{engw}  & Written English & 59,047 & 1,571 & 2.59  & 850   & 5,315 & 13,671 & 19,934 & 14,937 & 4,440 & \href{https://www.udir.no/lk20/fagkoder/eng0012}{\textsc{eng0012}} \\
    \textsc{nats}  & Natural Sciences & 59,642 & 976   & 1.61  & 490   & 4,801 & 12,134 & 17,179 & 17,448 & 7,590 & \href{https://www.udir.no/lk20/fagkoder/nat0010}{\textsc{nat0010}} \& \href{https://www.udir.no/lk20/fagkoder/nat0020}{\textsc{nat0020}} \\
    \textsc{noro}  & Oral Norwegian & 58,982 & 1,636 & 2.70  & 210   & 2,994 & 10,855 & 18,764 & 19,254 & 6,905 & \href{https://www.udir.no/lk20/fagkoder/nor0216}{\textsc{nor0216}} \& \href{https://www.udir.no/lk20/fagkoder/nor0042}{\textsc{nor0042}} \\
    \textsc{engo}  & Oral English & 59,148 & 1,470 & 2.43  & 441   & 3,200 & 9,938 & 19,725 & 18,947 & 6,897 & \href{https://www.udir.no/lk20/fagkoder/eng0013}{\textsc{eng0013}} \\
    \textsc{reli}  & Religion & 56,993 & 3,625 & 5.98  & 316   & 3,386 & 9,928 & 16,966 & 18,293 & 8,104 & \href{https://www.udir.no/lk20/fagkoder/RLE0030}{\textsc{rle0030}} \& \href{https://www.udir.no/lk20/fagkoder/rel0040}{\textsc{rel0040}} \\
    \textsc{socs}  & Social Sciences & 59,715 & 903   & 1.49  & 307   & 3,466 & 10,489 & 17,527 & 19,538 & 8,388 & \href{https://www.udir.no/lk20/fagkoder/saf0010}{\textsc{saf0010}} \& \href{https://www.udir.no/lk20/fagkoder/saf0020}{\textsc{saf0020}} \\
    \textsc{musi}  & Music & 57,716 & 2,902 & 4.79  & 120   & 1,450 & 6,771 & 18,830 & 22,940 & 7,605 & \href{https://www.udir.no/lk20/fagkoder/mus0010}{\textsc{mus0010}} \& \href{https://www.udir.no/lk20/fagkoder/mus0020}{\textsc{mus0020}} \\
    \textsc{hand}  & Arts and Handcraft & 58,001 & 2,617 & 4.32  & 93    & 1,142 & 6,876 & 19,744 & 23,140 & 7,006 & \href{https://www.udir.no/lk20/fagkoder/khv0010}{\textsc{khv0010}} \& \href{https://www.udir.no/lk20/fagkoder/khv0020}{\textsc{khv0020}} \\
    \textsc{phed}  & Physical Education & 57,731 & 2,887 & 4.76  & 102   & 913   & 4,612 & 16,606 & 26,037 & 9,461 & \href{https://www.udir.no/lk20/fagkoder/kro0020}{\textsc{kro0020}} \\
    \textsc{food}  & Food and Health & 57,683 & 2,935 & 4.84  & 16    & 561   & 6,060 & 19,098 & 23,907 & 8,041 & \href{https://www.udir.no/lk20/fagkoder/mhe0010}{\textsc{mhe0010}} \& \href{https://www.udir.no/lk20/fagkoder/mhe0020}{\textsc{mhe0020}} \\
        \multicolumn{12}{l}{\textbf{Written Exam Grades} (3 subjects)}\\
    \textsc{mat\_w} & Written Mathematics & 15,252 & 45,366 & 74.84 & 235   & 2,482 & 4,261 & 4,529 & 2,902 & 843   & \href{https://www.udir.no/lk20/fagkoder/mat0010}{\textsc{mat0010}} \\
    \textsc{nor\_w} & Written Norwegian & 13,851 & 46,767 & 77.15 & 237   & 2,345 & 5,245 & 4,106 & 1,616 & 302   & \href{https://www.udir.no/lk20/fagkoder/nor0214}{\textsc{nor0214}} \\
    \textsc{eng\_w} & Written English & 14,723 & 45,895 & 75.71 & 225   & 1,573 & 4,224 & 4,940 & 2,913 & 848   & \href{https://www.udir.no/lk20/fagkoder/eng0012}{\textsc{eng0012}} \\
        \multicolumn{12}{l}{\textbf{Oral Exam Grades} (3 subjects)}\\
    \textsc{mat\_o} & Oral Mathematics & 8,838 & 51,780 & 85.42 & 16    & 770   & 2,041 & 2,503 & 1,958 & 1,550 & \href{https://www.udir.no/lk20/fagkoder/mat0011}{\textsc{mat0011}} \\
    \textsc{nor\_o} & Oral Norwegian & 9,310 & 51,308 & 84.64 & 34    & 459   & 1,651 & 2,520 & 2,342 & 2,304 & \href{https://www.udir.no/lk20/fagkoder/nor0216}{\textsc{nor0216}} \\
    \textsc{eng\_o} & Oral English & 9,207 & 51,411 & 84.81 & 36    & 330   & 1,405 & 2,651 & 2,452 & 2,333 & \href{https://www.udir.no/lk20/fagkoder/eng0013}{\textsc{eng0013}} \\
        \bottomrule
    \end{tabular}%
}{
    Missing counts ($n$) and percentages ($\%$) were computed relative to the population size $N = 60,618$. Official documentation about each subject is available from the Norwegian Ministry of Education (\textsc{udir}) database by clicking each hyperlink. Subjects offered in both Norwegian and Sami as instruction languages are merged, with both codes given in the \textsc{udir} column.
}

\ttltable{tab:descriptive}{Variance-covariance Matrix and Correlation Table of Key Variables}{
      \begin{tabular}{clrrrrrrrrr}
      \toprule
      Seq   & \multicolumn{1}{c}{Variable} & \multicolumn{1}{c}{1} & \multicolumn{1}{c}{2} & \multicolumn{1}{c}{3} & \multicolumn{1}{c}{4} & \multicolumn{1}{c}{5} & \multicolumn{1}{c}{6} & \multicolumn{1}{c}{7} & \multicolumn{1}{c}{8} & \multicolumn{1}{c}{9} \\
      \midrule
      1     & YAGE  & \textbf{2.152} & \cellcolor[rgb]{ .745,  .745,  1}0.257 & \cellcolor[rgb]{ 1,  .851,  .851}$-$0.146 & \cellcolor[rgb]{ .902,  .902,  1}0.101 & \cellcolor[rgb]{ 1,  .984,  .984}$-$0.013 & \cellcolor[rgb]{ 1,  .906,  .906}$-$0.091 & \cellcolor[rgb]{ .863,  .863,  1}0.140 & \cellcolor[rgb]{ .965,  .965,  1}0.038 & \cellcolor[rgb]{ 1,  .996,  .996}$-$0.003 \\
      2     & PAGE  & 2.599 & \textbf{47.365} & \cellcolor[rgb]{ 1,  .686,  .686}$-$0.310 & \cellcolor[rgb]{ .776,  .776,  1}0.226 & \cellcolor[rgb]{ 1,  .898,  .898}$-$0.101 & \cellcolor[rgb]{ .949,  .949,  1}0.054 & \cellcolor[rgb]{ .918,  .918,  1}0.083 & \cellcolor[rgb]{ 1,  .918,  .918}$-$0.079 & \cellcolor[rgb]{ 1,  .949,  .949}$-$0.050 \\
      3     & SIB   & $-$0.214 & $-$2.122 & \textbf{0.992} & \cellcolor[rgb]{ 1,  .945,  .945}$-$0.052 & \cellcolor[rgb]{ .992,  .992,  1}0.009 & \cellcolor[rgb]{ .996,  .996,  1}0.005 & \cellcolor[rgb]{ 1,  .996,  .996}$-$0.002 & \cellcolor[rgb]{ .898,  .898,  1}0.105 & \cellcolor[rgb]{ 1,  .992,  .992}$-$0.004 \\
      4     & INC   & 33.367 & 348.795 & $-$11.532 & \textbf{50,467.613} & \cellcolor[rgb]{ 1,  .933,  .933}$-$0.065 & \cellcolor[rgb]{ .992,  .992,  1}0.010 & \cellcolor[rgb]{ 1,  .914,  .914}$-$0.083 & \cellcolor[rgb]{ .973,  .973,  1}0.030 & \cellcolor[rgb]{ 1,  .588,  .588}$-$0.411 \\
      5     & PMH\_SUM & $-$0.095 & $-$3.593 & 0.046 & $-$75.449 & \textbf{26.738} & \cellcolor[rgb]{ .588,  .588,  1}0.414 & \cellcolor[rgb]{ 1,  .694,  .694}$-$0.302 & \cellcolor[rgb]{ .6,  .6,  1}0.403 & \cellcolor[rgb]{ .831,  .831,  1}0.171 \\
      6     & CON\_SUM & $-$0.319 & 0.886 & 0.011 & 5.111 & 5.110  & \textbf{5.697} & \cellcolor[rgb]{ 1,  .541,  .541}$-$0.455 & \cellcolor[rgb]{ .608,  .608,  1}0.396 & \cellcolor[rgb]{ .804,  .804,  1}0.199 \\
      7     & COH\_SUM & 0.476 & 1.322 & $-$0.004 & $-$43.410 & $-$3.614 & $-$2.515 & \textbf{5.367} & \cellcolor[rgb]{ 1,  .741,  .741}$-$0.258 & \cellcolor[rgb]{ 1,  .937,  .937}$-$0.062 \\
      8     & ASB\_SUM & 0.698 & $-$6.793 & 1.307 & 84.089 & 25.908 & 11.768 & $-$7.447 & \textbf{154.805} & \cellcolor[rgb]{ 1,  .957,  .957}$-$0.041 \\
      9     & HAR   & $-$0.004 & $-$0.327 & $-$0.003 & $-$87.556 & 0.839 & 0.449 & $-$0.135 & $-$0.481 & \textbf{0.898} \\
      \phantom{9} & \phantom{PMH\_SUM} & \phantom{50,467.613} & \phantom{50,467.613} & \phantom{50,467.613} & \phantom{50,467.613} & \phantom{50,467.613} & \phantom{50,467.613} & \phantom{50,467.613} & \phantom{50,467.613} & \phantom{50,467.613}\\
      \bottomrule
      \end{tabular}
}{This table summarizes variables' variances (diagonal elements, bold), their covariances (lower triangle) and correlations (upper triangle, with heat map).}


\ttptable{tab:lat_PMH}{Confirmatory Factor Analysis for Latent Construct Parental Mental Health (PMH)}{
    \begin{tabular}{cc c rrrr c rr@{\hskip -0.08mm}l c c}
    \toprule
    \multicolumn{2}{c}{Item} &       & \multicolumn{4}{c}{Category Count} &       & \multicolumn{3}{c}{Factor Loading}       &       & \multicolumn{1}{c}{Residual} \\
\cmidrule{1-2}\cmidrule{4-7}\cmidrule{9-11}    Seq   & Code  &       & \multicolumn{1}{c}{1} & \multicolumn{1}{c}{2} & \multicolumn{1}{c}{3} & \multicolumn{1}{c}{4} &       & \multicolumn{1}{c}{$\lambda$} & \multicolumn{2}{c}{$SE(\lambda$)}       &       & \multicolumn{1}{c}{Variance} \\
    \midrule
    1     & SCL8\_01 &       & 70    & 60    & 19    & 7     &       & 0.872 & 0.026 & $^{***}$   &       & 0.046 \\
    2     & SCL8\_02 &       & 69    & 60    & 22    & 5     &       & 0.891 & 0.028 & $^{***}$   &       & 0.049 \\
    3     & SCL8\_03 &       & 48    & 67    & 29    & 12    &       & 0.849 & 0.029 & $^{***}$   &       & 0.049 \\
    4     & SCL8\_04 &       & 68    & 58    & 24    & 6     &       & 0.862 & 0.030 & $^{***}$   &       & 0.053 \\
    5     & SCL8\_05 &       & 32    & 77    & 35    & 12    &       & 0.825 & 0.029 & $^{***}$   &       & 0.047 \\
    6     & SCL8\_06 &       & 45    & 67    & 33    & 11    &       & 0.773 & 0.037 & $^{***}$   &       & 0.053 \\
    7     & SCL8\_07 &       & 66    & 57    & 26    & 7     &       & 0.716 & 0.043 & $^{***}$   &       & 0.062 \\
    8     & SCL8\_08 &       & 130   & 21    & 2     & 3     &       & 0.515 & 0.070 & $^{***}$   &       & 0.072 \\
    \bottomrule
    \end{tabular}%

    \bigskip

    \begin{tabular}{cccrrrrrrrr}
    \toprule
    \multicolumn{2}{c}{Item} &       & \multicolumn{8}{c}{MI (lower triangle) and EPC (upper triangle)} \\
\cmidrule{1-2}\cmidrule{4-11}    Seq   & Code  &       & \multicolumn{1}{c}{1} & \multicolumn{1}{c}{2} & \multicolumn{1}{c}{3} & \multicolumn{1}{c}{4} & \multicolumn{1}{c}{5} & \multicolumn{1}{c}{6} & \multicolumn{1}{c}{7} & 8 \\
    \midrule
    1     & SCL8\_01 &       & \multicolumn{1}{c}{---} & 0.291 & $-$0.233 & $-$0.105 & $-$0.088 & $-$0.139 & $-$0.053 & 0.014 \\
    2     & SCL8\_02 &       & 58.735 & \multicolumn{1}{c}{---} & $-$0.164 & $-$0.159 & $-$0.096 & $-$0.110 & $-$0.027 & 0.017 \\
    3     & SCL8\_03 &       & 19.023 & 10.699 & \multicolumn{1}{c}{---} & 0.135 & 0.079 & 0.082 & 0.007 & $-$0.010 \\
    4     & SCL8\_04 &       & 4.813 & 11.385 & 11.516 & \multicolumn{1}{c}{---} & 0.053 & 0.047 & $-$0.003 & $-$0.022 \\
    5     & SCL8\_05 &       & 3.983 & 4.282 & 3.862 & 1.780 & \multicolumn{1}{c}{---} & 0.061 & $-$0.013 & $-$0.006 \\
    6     & SCL8\_06 &       & 5.446 & 3.267 & 1.938 & 0.631 & 1.163 & \multicolumn{1}{c}{---} & 0.120 & $-$0.013 \\
    7     & SCL8\_07 &       & 0.811 & 0.221 & 0.015 & 0.002 & 0.053 & 3.124 & \multicolumn{1}{c}{---} & 0.023 \\
    8     & SCL8\_08 &       & 0.123 & 0.189 & 0.058 & 0.337 & 0.028 & 0.126 & 0.406 & \multicolumn{1}{c}{---} \\
    \bottomrule
    \end{tabular}%
}{Parental mental health (PMH) was measured by the 8-item \textit{Hopkins Symptom Checklist} (SCL-8). The unidimensionality assumption was associated with the following fit statistics: RMSEA $= 0.139$ (90\% CI: $[0.108, 0.171]$), CFI $= 0.955$, TLI $= 0.937$, and SRMR $= 0.069$. The minimum (oblique geomin) rotation function value was $5.08$. The upper panel contains the frequency table, factor loadings and residual variances. The lower panel reports modification indices (MI) and expected parameter changes (EPC).\\
$^* p < .05$. $^{**} p < .01$. $^{***} p < .001$.
}


% Table generated by Excel2LaTeX from sheet 'Sheet1'
\ttltableX{tab:results}{Structural Equation Model Parameters and Fit Indices}{
      \begin{tabular}{llll r@{\hskip -0.1mm}l r@{\hskip -0.1mm}l c r@{\hskip -0.1mm}l r@{\hskip -0.1mm}l c r@{\hskip -0.1mm}l r@{\hskip -0.1mm}l}

      \toprule

      \multicolumn{4}{c}{\multirow{2}[4]{*}{Variable}} & \multicolumn{4}{c}{Model 1}   &       & \multicolumn{4}{c}{Model 2}   &       & \multicolumn{4}{c}{Model 3} \\
      \cmidrule{5-8}\cmidrule{10-13}\cmidrule{15-18}    \multicolumn{4}{l}{}          & \multicolumn{2}{c}{MLR} & \multicolumn{2}{c}{Bayes} &       & \multicolumn{2}{c}{MLR} & \multicolumn{2}{c}{Bayes} &       & \multicolumn{2}{c}{MLR} & \multicolumn{2}{c}{Bayes} \\

      \midrule

      \multicolumn{4}{l}{\textbf{FIXED EFFECTS}} &       &       &       &       &       &       &       &       &       &       &       &       &       &  \\
      & \multicolumn{3}{l}{\textbf{Intercept}} &       &       &       &       &       &       &       &       &       &       &       &       &       &  \\
      &       & \multicolumn{2}{l}{of antisocial behavior (ASB)} & 0.054 &       &       &       &       & $-$0.024 &       &       &       &       & 0.410 &       &       &  \\
%      &       &       & rule-breaking behavior (RBB) &       &       &       &       &       &       &       &       &       &       &       &       &       &  \\
%      &       &       & aggression (AGG) &       &       &       &       &       &       &       &       &       &       &       &       &       &  \\
      &       & \multicolumn{2}{l}{of family conflict (CON)} &       &       &       &       &       & 0.436 &       &       &       &       & 0.108 &       &       &  \\
      &       & \multicolumn{2}{l}{of family cohesion (COH)} &       &       &       &       &       & 3.438 & ***   &       &       &       & 3.465 & ***   &       &  \\
      &       & \multicolumn{2}{l}{of parental mental health (PHM)} &       &       &       &       &       &       &       &       &       &       & 2.320 & ***   &       &  \\
      & \multicolumn{3}{l}{\textbf{Direct effect}} &       &       &       &       &       &       &       &       &       &       &       &       &       &  \\
      &       & \multicolumn{2}{l}{PMH $\longrightarrow$ ASB} & 0.402 & ***   & 0.566 & **   &       & 0.268 & ***   & 0.260 & *     &       & 0.280 & ***   & 0.280 & * \\
      & \multicolumn{3}{l}{\textbf{Mediating effect}} &       &       &       &       &       &       &       &       &       &       &       &       &       &  \\
      &       & \multicolumn{2}{l}{PMH $\longrightarrow$ CON $\longrightarrow$ ASB} &       &       &       &       &       &       &       &       &       &       &       &       &       &  \\
      &       &       & PMH $\longrightarrow$ CON &       &       &       &       &       & 0.413 & ***   & 0.485 & ***   &       & 0.392 & ***   & 0.448 & *** \\
      &       &       & CON $\longrightarrow$ ASB &       &       &       &       &       & 0.270 & ***   & 0.553 & **   &       & 0.298 & ***   & 0.590 & ** \\
      &       & \multicolumn{2}{l}{PMH $\longrightarrow$ COH $\longrightarrow$ ASB} &       &       &       &       &       &       &       &       &       &       &       &       &       &  \\
      &       &       & PMH $\longrightarrow$ COH &       &       &       &       &       & $-$0.301 & ***   & $-$0.416 & ***   &       & $-$0.299 & ***   & $-$0.417 & *** \\
      &       &       & COH $\longrightarrow$ ASB &       &       &       &       &       & $-$0.056 &       & 0.097 &       &       & $-$0.060 &       & 0.100 &  \\
      & \multicolumn{3}{l}{\textbf{Serial mediating effect}} &       &       &       &       &       &       &       &       &       &       &       &       &       &  \\
      &       & \multicolumn{2}{l}{Socioeconomic status (SES)} &       &       &       &       &       &       &       &       &       &       &       &       &       &  \\
      &       &       & SES $\longrightarrow$ ASB &       &       &       &       &       &       &       &       &       &       & $-$0.160 & *     & $-$0.214 & * \\
      &       &       & SES $\longrightarrow$ CON &       &       &       &       &       &       &       &       &       &       & 0.132 & *     & 0.171 & * \\
      &       &       & SES $\longrightarrow$ COH &       &       &       &       &       &       &       &       &       &       & $-$0.011 &       & 0.004 &  \\
      &       &       & SES $\longrightarrow$ PMH &       &       &       &       &       &       &       &       &       &       & 0.171 & *     & 0.193 & * \\
      & \multicolumn{3}{l}{\textbf{Control variables}} &       &       &       &       &       &       &       &       &       &       &       &       &       &  \\
      &       & \multicolumn{2}{l}{Adolescent's age (YAGE)} & 0.063 &       & 0.090 &       &       & 0.102 &       & $-$0.015 &       &       & 0.109 &       & $-$0.004 &  \\
      &       & \multicolumn{2}{l}{Parents' age (PAGE)} & $-$0.037 &       & $-$0.015 &       &       & $-$0.068 &       & 0.039 &       &       & $-$0.064 &       & 0.033 &  \\
      &       & \multicolumn{2}{l}{Number of siblings (SIB)} & 0.102 &       & 0.148 &       &       & 0.097 &       & 0.195 & *     &       & 0.095 &       & 0.181 &  \\
      &       & \multicolumn{2}{l}{Household income (INC)} & 0.062 &       & 0.038 &       &       & 0.048 &       & 0.077 &       &       & $-$0.018 &       & 0.002 &  \\
      & \multicolumn{3}{l}{\textbf{Covariance}} &       &       &       &       &       &       &       &       &       &       &       &       &       &  \\
      &       & \multicolumn{2}{l}{RRB $\longleftrightarrow$ AGG} &       &       & 0.610 & *    &       &       &       & 0.638 & *    &       &       &       & 0.620 & \\
      &       & \multicolumn{2}{l}{CON $\longleftrightarrow$ COH} &       &       &       &       &       & $-$0.377 & ***   & $-$0.484 & ***   &       & $-$0.378 & ***   & $-$0.491 & *** \\
      & & \multicolumn{2}{l}{\phantom{of parental mental health (PHM)}} & \phantom{$-$10,110.014} & \phantom{***} & \phantom{$-$0.065} & \phantom{***} & & \phantom{$-$10,068.675} & \phantom{***} & \phantom{$-$0.418} & \phantom{***} & & \phantom{$-$10,329.913} & \phantom{***} & \phantom{$-$0.413} & \phantom{***}\\
      \bottomrule
      \end{tabular}
}



\ttltableCont{
      \begin{tabular}{llll r@{\hskip -0.1mm}l r@{\hskip -0.1mm}l c r@{\hskip -0.1mm}l r@{\hskip -0.1mm}l c r@{\hskip -0.1mm}l r@{\hskip -0.1mm}l}

      \toprule

      \multicolumn{4}{c}{\multirow{2}[4]{*}{Variable}} & \multicolumn{4}{c}{Model 1}   &       & \multicolumn{4}{c}{Model 2}   &       & \multicolumn{4}{c}{Model 3} \\
      \cmidrule{5-8}\cmidrule{10-13}\cmidrule{15-18}    \multicolumn{4}{c}{}          & \multicolumn{2}{c}{MLR} & \multicolumn{2}{c}{Bayes} &       & \multicolumn{2}{c}{MLR} & \multicolumn{2}{c}{Bayes} &       & \multicolumn{2}{c}{MLR} & \multicolumn{2}{c}{Bayes} \\

      \midrule

      \multicolumn{4}{l}{\textbf{RANDOM EFFECTS}} &       &       &       &       &       &       &       &       &       &       &       &       &       &  \\
            & \multicolumn{3}{l}{Residual variance} &       &       &       &       &       &       &       &       &       &       &       &       &       &  \\
            &       & \multicolumn{2}{l}{of ASB} & 0.821 & *** & 0.548 & *** &       & 0.743 & *** & 0.458 & *** &       & 0.722 & *** & 0.440 & *** \\
            &       &       & of RRB &       &       & 0.638 & *** &       &       &       & 0.682 & *** &       &       &       & 0.639 & *** \\
            &       &       & of AGG &       &       & 0.548 & *** &       &       &       & 0.368 & *** &       &       &       & 0.353 & *** \\
            &       & \multicolumn{2}{l}{of CON} &       &       &       &       &       & 0.829 & *** & 0.765 & *** &       & 0.811 & *** & 0.735 & *** \\
            &       & \multicolumn{2}{l}{of COH} &       &       &       &       &       & 0.910 & *** & 0.827 & *** &       & 0.909 & *** & 0.820 & *** \\
            &       & \multicolumn{2}{l}{of PMH} &       &       &       &       &       &       &       &       &       &       & 0.971 & *** & 0.963 & *** \\
      \midrule
      \multicolumn{4}{l}{\textbf{MODEL FIT INDICES}} &       &       &       &       &       &       &       &       &       &       &       &       &       &  \\
            & \multicolumn{3}{l}{Log-likelihood} & $-$10,113.419 &       &       &       &       & $-$10,072.121 &       &       &       &       & $-$10,333.277 &       &       &  \\
            & \multicolumn{3}{l}{Number of free parameters} & 135   &       & 183   &       &       & 140   &       & 206   &       &       & 144   &       & 208   &  \\
            & \multicolumn{3}{l}{AIC} & 20,496.838 &       &       &       &       & 20,424.242 &       &       &       &       & 20,954.554 &       &       &  \\
            & \multicolumn{3}{l}{BIC} & 20,909.431 &       &       &       &       & 20,852.117 &       &       &       &       & 21,394.653 &       &       &  \\
            & \multicolumn{3}{l}{SRMR} & 0.230 &       &       &       &       & 0.229 &       &       &       &       & 0.229 &       &       &  \\
            & \multicolumn{3}{l}{$R^2$} &       &       &       &       &       &       &       &       &       &       &       &       &       &  \\
            &       & \multicolumn{2}{l}{of ASB} & 0.179 & ** & 0.452 & *** &       & 0.257 & *** & 0.542 &*** &       & 0.278 & *** & 0.560 & *** \\
            &       &       & of RBB &       &       & 0.365 & *** &       &       &       & 0.318 & *** &       &       &       & 0.361 & *** \\
            &       &       & of AGG &       &       & 0.362 & *** &       &       &       & 0.632 & *** &       &       &       & 0.647 & *** \\
            &       & \multicolumn{2}{l}{of CON} &       &       &       &       &       & 0.171 & ** & 0.235 & *** &       & 0.189 & ** & 0.265 & *** \\
            &       & \multicolumn{2}{l}{of COH} &       &       &       &       &       & 0.090 & * & 0.173 & *** &       & 0.091 & * & 0.180 & *** \\
            &       & \multicolumn{2}{l}{of PMH} &       &       &       &       &       &       &       &       &       &       & 0.029 &       & 0.037 & *** \\
            & & \multicolumn{2}{l}{\phantom{of parental mental health (PHM)}} & \phantom{$-$10,110.014} & \phantom{***} & \phantom{$-$0.065} & \phantom{***} & & \phantom{$-$10,068.675} & \phantom{***} & \phantom{$-$0.418} & \phantom{***} & & \phantom{$-$10,329.913} & \phantom{***} & \phantom{$-$0.413} & \phantom{***}\\
      \bottomrule
      \end{tabular}
}{This table summarizes the model building process. Model 1 only proposes a direct relationship between parental mental health (PMH) and adolescent's antisocial behavior (ASB). Mediators are then introduced in Model 2 to account for the effects of family conflict (CON) and family conhesion (COH) on ASB. Finally, socioeconomic status (SES) is hypothesised to influence every existing variable in Model 3. All statistics are pooled results over ten imputed datasets. Significance tests are two-tailed.\\
$^* p < .05$. $^{**} p < .01$. $^{***} p < .001$.}


\appendix

\section{Ethics Approval}\label{app:ethics}

This study was approved by the Regional Committee for Medical and Health Research Ethics (REK Sør-Øst, Norway) (REK reference number 2019/1589). All participants gave written informed consent. The study was conducted in accordance with the \textit{Declaration of Helsinki}.


\section{Latent Variables}\label{app:latent}

\subsection{Overview of the Measurement Models}

\begin{figure}[htbp]
    \caption{Correlogram of the Questionnaire Items}
    \label{fig:heat}
    \includegraphics[width=\textwidth]{./Figures/heat_map.pdf}
\end{figure}

\subsection{Measurement Models of Parental Mental Health (PMH)}

\subsection{Measurement Models of Family Conflict (CON)}

\subsection{Measurement Models of Family Cohesion (COH)}

\subsection{Measurement Models of Adolescent Antisocial Behavior (ASB)}

\subsubsection{Rule-breaking Behavior (RBB)}

\subsubsection{Aggression (AGG)}


%\section{Analysis Code}\label{app:code}

\subsection{\textsf{Mplus} Script for Descriptive Statistics}

\begin{singlespacing}
    \lstinputlisting[language=Mplus,style=vscodeMplus]{Mplus/des_0.out}
\end{singlespacing}
\newpage

\subsection{\textsf{Mplus} Script for Exploratory Factor Analysis}

\begin{singlespacing}
    \lstinputlisting[language=Mplus,style=vscodeMplus]{Mplus/efa_1_pmh.out}
\end{singlespacing}

\begin{singlespacing}
    \lstinputlisting[language=Mplus,style=vscodeMplus]{Mplus/efa_2_rbb.out}
\end{singlespacing}

\newpage

\begin{singlespacing}
    \lstinputlisting[language=Mplus,style=vscodeMplus]{Mplus/efa_3_agg.out}
\end{singlespacing}

\newpage

\begin{singlespacing}
    \lstinputlisting[language=Mplus,style=vscodeMplus]{Mplus/efa_4_coh.out}
\end{singlespacing}

\newpage

\begin{singlespacing}
    \lstinputlisting[language=Mplus,style=vscodeMplus]{Mplus/efa_5_con.out}
\end{singlespacing}

\newpage

\begin{singlespacing}
    \lstinputlisting[language=Mplus,style=vscodeMplus]{Mplus/efa_6_asb.out}
\end{singlespacing}

\newpage

\subsection{\textsf{Mplus} Script for Path Analyses and SEM}

\begin{singlespacing}
    \lstinputlisting[language=Mplus,style=vscodeMplus]{Mplus/pat_0_mlr_dir.out}
\end{singlespacing}

\newpage

\begin{singlespacing}
    \lstinputlisting[language=Mplus,style=vscodeMplus]{Mplus/pat_1_mlr_med.out}
\end{singlespacing}

\newpage

\begin{singlespacing}
    \lstinputlisting[language=Mplus,style=vscodeMplus]{Mplus/sem_2_bayes_2_theory.out}
\end{singlespacing}

\newpage

\begin{singlespacing}
    \lstinputlisting[language=Mplus,style=vscodeMplus]{Mplus/sem_3_bayes_2_free.out}
\end{singlespacing}

\newpage

\begin{singlespacing}
    \lstinputlisting[language=Mplus,style=vscodeMplus]{Mplus/sem_4_bayes_6_free.out}
\end{singlespacing}

\end{document}
