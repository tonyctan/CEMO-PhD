\section{Preregistration}

\subsection{Title}
Differential Effects of COVID-19 School Closures on Students' Learning

\subsection{Authors}
Tim F{\"u}tterer, Tony C. A. Tan, Rolf V. Olsen, Sigrid Bl{\"o}meke

\subsection{Keywords}
COVID-19, student achievement, social inequality, Norway, socioeconomic status, school closures

\subsection{Description}
\textit{Preface:} This study uses Norway's national registers to investigate the associations between COVID-19 school closures and students' learning outcomes. Register datasets overcome sampling issues by preserving information about the entire Norwegian population. Due to our source data's large sizes, we expect statistical significance for most inferential parameters. In addition to preventing questionable practices such as $p$-hacking, this preregistration serves to enhance research transparency by declaring our research questions and methodological approaches before key variables become available to the authors.

School closures resultant from the COVID-19 pandemic in 2020 and 2021 represented a sudden and unexpected disruption of students' learning in schools. ``School systems had to rapidly improvise to ensure some continuity in the education of children and adapt their teaching methods to a situation in which, in the space of a day, the setting in which education took place moved from the school to the home for most children and the mode of instruction shifted from face-to-face contact between pupils and their teachers/instructors to some form of remote or distance learning, often supervised by parents'' \parencite[][p. 13]{thorn:2021}. Findings from previous studies suggest that school closures had a negative effect on student achievement ($d=-0.005\ SD$ to $d=-0.05\ SD$ per week), especially for students with low socioeconomic status \parencite{hammerstein:2021}. In their meta-analysis, \textcite{betthauser:2022} found an overall negative effect of school closures on student learning early in the pandemic across 34 studies (Cohen's $d=-0.17$). In summary, the results indicate that measures to maintain learning during school closures that began in March 2020 were not effective. Although these findings appeared robust as many studies examined large samples, used administrative or test data, and often employed methods enabling causal inferences (e.g., difference-in-difference approach), most of the prior studies contained weaknesses in their underlying data \parencite{thorn:2021}. For example, convenience samples were often used \parencite[e.g., ][]{clark:2021}, data were not representative of the underlying population \parencite[e.g., ][]{kuhfeld:2020}, or data were obtained from survey methods such as web-based surveys, mobile apps, or telephone interviews result in distorted samples and/or low response rates \parencite[e.g., ][]{vandervelde:2021}. Moreover, although some consensus has emerged (e.g., greater learning loss among students with lower SES), uncertainties remain among mixed findings across different subjects. Lastly, school closures had prevented achievement assessments from taking place. The lack of data presents educational researchers with additional challenge of generating evidence on the impact of school closures. Against this background, there is a need for both studies that se the impact of school closures on student achievement using enhanced methodology based on high quality data (e.g., representative data allowing us to draw conclusions about the entire population including family circumstances that are important for learning at home) and studies on how to accommodate systemic missing data.

We aim to conduct two studies using Norway's register data between 2009 and 2020. In Study 1 (number of archival entries $\approx$ 12.3 million), we will present a Bayesian approach to estimating missing exam data in 2020 using Year 10 students' teacher-assigned grades and exam grades from the previous 10 years, with particular focus on mathematics, Norwegian and English. In Study 2 (number of archival entries $\approx$ 5.6 million), we will use Year 8 and 9 students' national test data in reading and mathematics as measures for students' learning. Since identical tests are used in Year 8 and 9, learning growths can be operationalised using difference scores. We will use difference-in-difference (DiD) approaches to compare learning progression (i.e., gains or losses) between the cohorts effected by school closures due to COVID-19 and previous cohorts. We therefore aim to provide robust findings that allow causal inferences on the effects of school closures (e.g., duration) on students' learning progression. This study focuses on the school closure elasticity of student achievement (i.e., percentage change in student academic achievement in response to percentage change in school closure). This study aims to gain insights into the importance of teaching and learning in schools. In addition, we aim to shed light into the differential effects of school closures by looking at important background variables of learning at home (e.g., parental income and education status, and housing conditions in terms of floor areas per person). Findings from this study will assist future policy-formation by quantifying educational costs resultant from major social measures.

The full preregistration is available in PDF on our \href{https://osf.io/t6myh/?view_only=85ac0580daf54c44979de1b9ffe0c011}{OSF project page}.

\section{Study Information}

\subsection{Research questions}
Our overall research question is: How did school closures affect students' learning? We plan on approaching this overarching theme through two studies. The first study explores how statistical methods can address the systemic missing of important information through this research question (RQ):

\textbf{RQ1} How can missing data in Year 10 exam grades in 2020 be estimated using the Bayesian inference approach? \label{rq:1}

In the second study, we examine differential effects of school closures on the estimated student achievement (Year 10, see RQ1) and national test data (Year 8 and 9) in reading and mathematics before and during the school closures. Specifically, we wish to investigate the following research questions:

\textbf{RQ2} What impact had school closures had on students' learning outcomes?

\textbf{RQ3} How were students' learning outcomes related to students' socioeconomic status, household, and family characteristics?

\textbf{RQ4} How did learning outcomes during closure differ between low and high achievers?

\subsection{Hypothesis}
We will consider all effects to be meaningful for both the period that includes the first school closures in 2020 (Period 1) and the period that includes the second school closures in 2021 (Period 2).

\textbf{H1} Regarding RQ2: We expect small effect sizes $d=-{0.2}/{52}=-0.0038\ SD$ per week or smaller in learning loss in Period 1.

\textit{Notes}: For a justification of H1 see the Effect Size section. For Period 2, we will conduct exploratory analyses (see the Exploratory Analyses section). The term learning loss does not suggest a decline in total amount of learning, but smaller growth sizes compared to equivalent cohorts before COVID school closures. Since identical national tests are implemented in Year 8 and Year 9 (see the Variables section), we expect cohorts prior to 2020 to have advantages in learning growth. In addition, we expect to see differential effects of school closures on students' learning growth.

\textbf{H2} Regarding RQ3: Effects of school closures on students' learning are moderated by students' SES in Period 1, with lower SES students showing more learning loss from school closures than their higher SES peers.

\textit{Justification for H2}: Previous studies on Period 1 highlighted the critical role SES played for the extent of learning loss. Although Norwegian residents usually enjoy high standards of living, such privilege is distributed unevenly when comparing the mean of the 5th against that of the 95th percentile \parencite{oecd:2019}, suggesting unequal relationship between school closures and SES. More specifically, we conjecture that the less favourable students' SES, the greater their learning loss.

For Period 2, we will conduct exploratory analyses (see the Exploratory Analysis section), but we expect similar differential effects as for Period 1 such that SES significantly moderates the relationship between school closures and learning loss.

\textbf{H3} Regarding RQ4: Effects of school closures on students' learning are moderated by students' previous achievement in Period 1. Low achievers show more learning loss due to school closures than high achievers.

\textit{Justification for H3}: It is well known that low and high achievers share dissimilar metacognitive capabilities. As high achievers are usually better at self-regulated learning, we expect high achievers to be affected less by distance learning than low achievers. For Period 2, we will conduct exploratory analyses (see the Exploratory Analysis section), with similar differential effects from Period 1. We expect low achievers to show more learning loss due to school closures in Period 2 than high achievers.
