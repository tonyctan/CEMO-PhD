\section{Data Description}

\subsection{Datasets}
This project sources its data from the Norwegian national register. This data source captures information about the entire Norwegian population dating back to the early 1900s through each individual's unique national ID number. Under a secured IT environment, we obtain national statistics on Norwegian residents' education (e.g., person's highest level of education, academic attainment record), employment (e.g., working hours per week), income (e.g., after-tax income), COVID-19 specific information (e.g., COVID infection rates), housing conditions (e.g., floor areas, number of persons per household), as well as family relations and composition (e.g., kinship, number of siblings). Most importantly, entries across different datasets can be linked through person IDs, enabling us to match students with their parents' education and income data as well as their housing conditions such as floor space. Furthermore, municipality-level (e.g., duration of school closures) and school-level data (e.g., school resources, student composition) can be linked to student-level outcomes (e.g., teacher-assigned grades, exam grades, and national test grades).

\subsection{Data Availability}
The datasets underlying this project were provided by Statistics Norway (SSB) and the Norwegian Ministry of Education (UDIR) by permission. Researchers can gain access to these datasets by submitting written applications to \href{https://www.ssb.no/en/omssb/tjenester-og-verktoy/data-til-forskning}{SSB} and by following instructions on \href{https://login.udir.no/LoggInn/logginn}{UDIR website} respectively. The Norwegian national register contains large amount of private and sensitive information. Research institutes must provide sufficient justification and undergo a rigorous application process for data access. The Norwegian government requires data access to be granted only to registered users and within a secured IT environment that fully logs every operation.

\subsection{Data Access}
The dataset can be accessed using \href{https://www.uio.no/english/services/it/research/sensitive-data/}{secured IT infrastructure} only.

\subsection{Data Identifiers}
No persistent, unique identifier of the datasets is available.

\subsection{Access Date}
The second author received access credential to the register data in April 2022. However, we do not have access to key independent (e.g., durations of school closures, housing condition data) and dependent variables (e.g., 2021 grades and national test data) at the time of preregistration lodgement. Retrieval applications will be submitted in August 2022, with expected delivery in autumn 2022.

\subsection{Data Collection Procedures}
Norwegian national register is maintained by Statistics Norway (\href{https://www.ssb.no/en}{SSB}). SSB is the national statistical institute of Norway and the main producer of official statistics. SSB is responsible for collecting statistics related to the economy, population, and society at national, regional and local levels. Information related to school characteristics was managed by the Ministry of Education and Training (UDIR) through the School Information System (\href{https://gsi.udir.no}{GSI}). Our project team received access to both data sources in April 2022. No further data collection is conducted.

\subsection{Codebook}
Table 1 describes key variables used in our study. A full codebook can be obtained from our \href{https://osf.io/t6myh/?view_only=85ac0580daf54c44979de1b9ffe0c011}{OSF page}.