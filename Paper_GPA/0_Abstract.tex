\section{Abstract}

%//mark A NERA proposal for presentation must be written in English. The abstract should be 300--500 words (including references, if any) and should include the following:
%//mark 1. Research topic/aim
%//mark 2. Theoretical framework
%//mark 3. Methodology/research design
%//mark 4. Expected results/findings
%//mark 5. Relevance to Nordic educational research

\subsection{Research Topic}

%//mark [Sigrid]:
%//mark Research objective: examine the consequences of treating all grades equally through, say, estimating a new score?

%//mark [Astrid]:
%//mark Use GPA and university entry as a motivator
%//mark Narrow research target to grades, not GPA

The Grade Point Average (GPA, \textit{skolepoeng} in Norwegian) plays a determining role in Norway's tertiary admission process. The academic track in Norwegian upper secondary education offers students a set of compulsory joint core subjects as well as a wide range of elective subjects for different specialisations. Since different elective subjects are treated \emph{equally} in its calculation, GPA implicitly assumes that grades across different specialised subjects are \emph{equivalent} indicators of students' preparedness for higher education---an assumption that remains untested and questioned by descriptive statistics \parencite{udir:2022}. This paper focuses on the comparability of difficulty levels across subjects to provide a test of the hidden assumption in the current procedure for producing the GPA.

\subsection{Theoretical Framework}

%//mark [Rolf]:
%//mark selection as an object //done
%//mark introduce fairness as part of a validity framework //done
%//mark grading as a process //done

Fairness is both an essential and an elusive integral of educational assessment. Following \poscite{gipps:2009} social-cultural framing of assessment fairness and \poscite{tierney:2017} democratic--measurement--pedagogical construction, the current study models GPA as a selection device \parencite{kane:2013} for accessing privileged social resources \parencite{bourdieu:1973}. It addresses the construct validity of GPAs by examining any construct-irrelevant variance \parencite{messick:1989} related to students' subject choices.

\subsection{Methodology}

Item response theory is particularly suitable for extracting item difficulty information in order to study assessment's selection fairness. This study considers each GPA subject as an item and each candidate as a person. Using marginal maximum likelihood (MML) estimation, the analyses will ascertain difficulty parameters for all major subjects in Norwegian upper secondary schools. Registry data containing Norwegian students' GPA performance in 2019 are first regularised by removing subjects with fewer than 1,000 candidates and by only including candidates who have received valid GPAs through upper secondary school completions. Next, subject difficulty parameters will be extracted using generalised partial credit models \parencite[GPCM,][]{muraki:1992}. Lastly, group invariance tests are applied to assess the extent to which selection bias had impacted on subject difficulty parameter estimates.

\subsection{Expected Results}

The registry data set will be available for analysis in short time and the described analyses will be presented and discussed at the conference. We expect Norway's GPA subjects to differ in difficulties \parencite{he:2018} and to exhibit significant selection effects \parencite{korobko:2008}.

\subsection{Relevance to Nordic Educational Research}

Given that university entries in Europe is largely based on the final grades from secondary schooling, the presented analysis is likely to be relevant to other countries using grades as the selection criteria into tertiary education. The issue of potential unequal treatment of students with different specialisation in upper secondary school applies beyond the Norwegian context. By testing the assumption that grades from different specialities support GPA's selection purpose equally well, this study lends statistical support to evidence-based policy formation process commonly practised in the Nordic community and serves to strengthen the fairness of our merit-based university admission decisions.
