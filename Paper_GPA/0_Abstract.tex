\section{Abstract}

%//mark A NERA proposal for presentation must be written in English. The abstract should be 300--500 words (including references, if any) and should include the following:
%//mark 1. Research topic/aim
%//mark 2. Theoretical framework
%//mark 3. Methodology/research design
%//mark 4. Expected results/findings
%//mark 5. Relevance to Nordic educational research

\subsection{Research Topic}

The Grade Point Average (GPA, \textit{skolepoeng} in Norwegian) plays a determining role in Norway's tertiary admission process. The academic track in Norwegian upper secondary education offers students a set of compulsory joint core subjects as well as a wide range of elective subjects for different specialisations, e.g., a specialisation in the sciences or language arts. Each subject awards students a grade ranging from $0$ to $6$ for low- and high-competence respectively \parencite[][\S 3-5]{lovdata:2006}. Students' GPAs are best characterised as sum scores of their subject grades. For the majority of tertiary studies, the different elective subjects are treated equally when producing the GPA. In other words, the GPA implicitly assumes that grades across different specialised subjects are \emph{equivalent} indicators of students' preparedness for higher education, an assumption that remains untested. Descriptive statistics suggests that there are substantial differences in grades across subjects \parencite{udir:2022}. The current study is part of a larger project to examine Norwegian administrative grade data by Item Response Theory (IRT). Specifically, this paper focuses on \textit{the comparability of difficulty levels across subjects}, and thus provides a test of the hidden assumption in the current procedure for producing the GPA.

\subsection{Theoretical Framework}

IRT is particularly suitable for extracting item difficulty information in order to study assessment's selection fairness. This study considers each GPA subject as an item and each candidate as a person. Using marginal maximum likelihood (MML) estimation, the analyses will ascertain difficulty parameters for all major subjects in Norwegian upper secondary schools. A second theoretical consideration relates to self-selection bias. Freedom in subject choices in Norway's upper secondary academic track inevitably produces rather sparse data matrix once all subjects and students are included. Since the presence or absence of observations was not resulted from randomisation but self-selection, and the missing likelihood is reasonably expected to covary with the subject difficulties, the observed GPA datasets shall be considered missing not at random \parencite[MNAR,][]{rubin:1976}. Leaving untreated, such non-ignorable missingness would cause over- and under-estimates of person and item parameters, respectively \parencite{rose:2013}. In order to assess the impact of non-random missings on difficulty parameter estimates, IRT analyses will be repeated on three groups: the whole population, medical school applicants (low subject choice freedom) and language arts stream students (high freedom).

\subsection{Methodology}

Registry data containing Norwegian students' GPA performance in 2019 are first regularised by removing subjects with fewer than 1,000 candidate and candidates taking fewer than two subjects following the practices in \textcite{he:2018}. Candidates' grades are then recoded into a polytomous scale between $0$ and $5$ representing the low- and high-ends of the competency spectrum. Next, subject difficulty parameters will be extracted using generalised partial credit models \parencite[GPCM,][]{muraki:1992} over three sub-groups (whole population, medicine, and language arts). Lastly, the sensitivity analysis section will contain group invariance tests to assess the extend to which selection bias had impacted on subject difficulty parameter estimates.

\subsection{Expected Results}

The registry data set will be available for analysis in short time and the described analyses will be presented and discussed at the conference. Given that university entries in Europe is largely based on the final grades from secondary education, Norway's GPA system is expected to be comparable to the A Levels in the UK and the Central Examinations in Secondary Education in the Netherlands. More specifically, we expect Norway's GPA subjects to differ in difficulties \parencite[per report by][]{he:2018} and to exhibit significant selection effect \parencite[as demonstrated in][]{korobko:2008} represented by different difficulty parameters among the whole sample, medical school applicants, and language arts candidates.

\subsection{Relevance to Nordic Educational Research}

Fairness and equal treatment represent the guiding principles of Nordic societies. In examining assessment fairness, researchers in Nordic countries are privileged to have access to national registry data, a gateway to nuanced information about individual-level phenomena. Consensus on a standard procedure for analysing registry data for educational research purposes, however, are yet to emerge that safeguards methodological accuracy as well as promotes social welfare at large. This study pays particular attention to the non-ignorable missing data issues during IRT modelling. Establishing and verifying the analytical procedures and properties of resultant estimates would directly benefit Nordic educational scientists communities using registry data.

\printbibliography