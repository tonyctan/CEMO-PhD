\section{Abstract}

%//mark A NERA proposal for presentation must be written in English. The abstract should be 300--500 words (including references, if any) and should include the following:
%//mark 1. Research topic/aim
%//mark 2. Theoretical framework
%//mark 3. Methodology/research design
%//mark 4. Expected results/findings
%//mark 5. Relevance to Nordic educational research

\subsection{Research Topic}

Grade point averages (GPA) play monopolistic roles in Norway's tertiary admission processes. Earlier studies from the UK \parencite{he:2018} and the Netherlands \parencite{korobko:2008}, however, raised methodological and fairness concerns over GPA as an appropriate measure for graduates' academic competency. Violations of the unidimensionality assumption arose when different subjects contribute to the final GPA scores at different weights, causing invalid statistical inferences under the item response theory (IRT) framework. Additionally, misaligned subject difficulties distort candidates' incentives, leading to material misallocation of youth's time and effort at a critical point in their studies. This paper aims to examine whether Norway's GPA subjects exhibit comparable difficulty levels, both across candidate cohorts (e.g., medical school vs general tertiary applicants) and across time. It further investigates covariates that associated strongly with any discrepancies in subject difficulties for policy considerations.

\subsection{Theoretical Framework}

IRT is particularly suitable in the educational measurement literature for extracting item difficulty parameters. This study considers each GPA subject as an IRT item and each candidate as an IRT person. It primarily focuses on the item parameters $\m{\beta}$ while treating person competencies $\m{\theta}$ as ``nuisance parameters'' by integrating them out using marginal maximum likelihood estimates. Additionally, since students had self-selected into GPA subjects with highest expected payoffs, the observed GPA datasets are reasonably expected to involve the missing-not-at-random (MNAR) mechanism. Leaving untreated, such non-ignorable missingness would cause over- and under-estimates of person and item parameters, respectively \parencite{rose:2013}. This study addresses MNAR using a multiple imputation procedure prior to IRT analyses.

\subsection{Methodology}



\subsection{Expected Results}

\subsection{Relevance to Nordic Educational Research}