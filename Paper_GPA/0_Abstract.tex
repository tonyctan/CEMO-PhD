\section{Abstract}

%//mark A NERA proposal for presentation must be written in English. The abstract should be 300--500 words (including references, if any) and should include the following:
%//mark 1. Research topic/aim
%//mark 2. Theoretical framework
%//mark 3. Methodology/research design
%//mark 4. Expected results/findings
%//mark 5. Relevance to Nordic educational research

\subsection{Research Topic}

Grade point averages (GPA) play a determining role in Norway's tertiary admission processes. Earlier studies from the UK \parencite{he:2018} and the Netherlands \parencite{korobko:2008}, however, raised methodological and fairness concerns over GPA as an appropriate measure for graduates' academic competency. Violations of the unidimensionality assumption arose when different subjects contribute to the final GPA scores at different weights, undermining inference validity under the item response theory (IRT) framework. Additionally, misaligned subject difficulties distort candidates' incentives, leading to material misallocation of youth's time and effort at a critical point in their studies. This paper aims to examine \emph{whether Norway's GPA subjects differ in difficulty levels}, both across candidate cohorts (e.g., medical school vs general tertiary applicants) and across time. It further investigates covariates that associated strongly with any discrepancies in subject difficulties for policy considerations.

\subsection{Theoretical Framework}

IRT is particularly suitable in the educational measurement literature for extracting item difficulty parameters. This study considers each GPA subject as an IRT item and each candidate as an IRT person. It primarily focuses on the item parameters while integrating person competencies out of the equations using marginal maximum likelihood (MML) estimates. In addition, the observed GPA datasets were expected to result from missing-not-at-random (MNAR) processes since students had self-selected into GPA subjects with highest expected payoffs. Leaving untreated, such non-ignorable missingness would cause over- and under-estimates of person and item parameters, respectively \parencite{rose:2013}. This study addresses MNAR using a multiple imputation procedure prior to IRT analyses.

\subsection{Methodology}

This study adopts a cleaning--imputing--analysing modular design. Registry data containing Norwegian students' GPA performance between 2009 and 2019 are first regularised year-by-year by removing subjects with fewer than 1,000 candidate and candidates taking fewer than two subjects following the practices in \textcite{he:2018}. Candidates' grades are then recoded into a polytomous scale with $0$ and $5$ representing the low- and high-ends of competency spectrum. Multiple imputation by chained equations are then applied to the observed data matrix to obtain ten imputed versions for the subsequent MML estimations involving generalised partial credit models \parencite[GPCM,][]{muraki:1992} for subject- and grade-difficulty parameter extractions. Final estimates are pooled together using Rubin's Rule \parencite{rubin:1987} in order to obtain correct means and standard error statistics. Identical procedures are applied to each year to obtain a pooled cross sectional output and special cohorts such as medical school applicants are highlighted for sensitivity analyses.

\subsection{Expected Results}

University entry exams are invariably high-stake endeavours. Although differ in operational details, Norway's GPA system is expected to be comparable to the A Levels in the UK and the Central Examinations in Secondary Education in the Netherlands. More specifically, we expect Norway's GPA subjects to differ in difficulties \parencite[per report by][]{he:2018} and to exhibit significant selection effect \parencite[as demonstrated in][]{korobko:2008}. We further expect subject difficulty parameters to increase once the missing data mechanisms had been taken into account.

\subsection{Relevance to Nordic Educational Research}

Researchers in Nordic countries are privileged to have access to national registry data, a gateway to nuanced information about individual-level phenomena. Consensus on a standard procedure for analysing registry data for educational research purposes, however, are yet to emerge that ensures methodological validity and reliability as well as promotes social welfare at large. This study show cases a modular analysis design by explicitly addressing non-ignorable missing data issues before submitting the datasets to IRT modelling. Establishing and verifying the analytical procedures and properties of resultant estimates would directly benefit Nordic research communities using registry data.

\printbibliography