%\section{Introduction} % Do not include the word "Introduction" as a Level 1 heading. Just start intro as a normal paragraph immediately after the title.

The grade point average (GPA, \textit{skolepoeng} in Norwegian) plays a key role in Norway's educational assessment process. From Year 8 onwards, Norwegian high schoolers receive formal grades from both their teachers (\textit{standpunktkarakter}) and year-end exams \parencite{raeder:2020}, which are used for high-stake decisions such as graduate certifications (Year 10) and university admissions (VG3). Since different subjects are treated \emph{equally} in its calculation \parencite{gpa:2021}, GPA implicitly assumes that grades across different specialities are \emph{equivalent} indicators of students' preparedness for the next stage of education --- an assumption that remains untested and questioned by descriptive statistics \parencite{udir:2022}.

Concerns for the comparability of subject difficulties are further deepened by prior studies from education systems similar to that of Norway's. \textcite{he:2018} in the UK and \textcite{korobko:2008} in the Netherlands both reported persistent disagreements among subject difficulties, which may lead to differential treatment of students with different specialisations. Besides fairness concerns, the lack of difficulty comparability also leads to a lack of construct validity \parencite{messick:1989} in GPA, as the construct-irrelevant variance related to subject characteristics, in addition to candidates' competencies, have been included in the GPA calculation. Understanding the presence, directions, and magnitudes of inter-subject difficulties therefore becomes a key issue for assessing the validity of GPA. By analysing Norway's education record data, this study aims to test GPA's validity as a measurement scale in mapping candidate competence into numeric scores, as well as the fairness consequences subsequent to its use in high-stake situations. More specifically, this study will address the following research questions:
\begin{ttitemize}
    \item[\textbf{RQ1:}] Do Norwegian Year 10 subjects differ in their difficulty levels?
    \item[\textbf{RQ2:}] Do subject difficulties differ by source such as between teachers and external examiners?
    \item[\textbf{RQ3:}] Do subject difficulties differ across achievement levels?
    \item[\textbf{RQ4:}] Do subject difficulties differ across demographic attributes such as socioeconomic status, gender and immigration background?
\end{ttitemize}