\section{Theoretical Framework}

%\subsection{The Norwegian GPA System}

\subsection{Missing Data Treatment}

IRT item parameter estimations demand strong assumptions on the data missing mechanism. Joint, conditional and marginal maximum likelihood procedures are only valid under ``ignorable non-response'' conditions where missing propensities are related to neither item nor person parameters \parencite{molenaar:1995}. When test takers skip items after seeing their content, for example, the ignorablity condition is unlikely to hold \parencite{mislevy:1987}, neither are tests with items not reached due to time constraint \parencite{lord:1974, lord:1983}. In the current study involving Norway's GPA archive, missing records are not the result of randomly assigning candidates to subjects (not MCAR), nor are each candidate's missing GPAs independent of observed ones (not MAR). In fact, missing patterns are likely to be related to both personal capabilities and subject difficulties with low ability candidates self-selecting into easy subjects while difficult subjects attracting only high capability students. Resultantly, MML estimates of subject difficulties after marginalising personal parameters are no longer unbiased \parencite[][Table 2]{mislevy:1988}.

Literature has congregated into three main approaches for the purpose of addressing missing values. In the \emph{classical approaches}, missing responses can be (a) ignored and treated as non-administered, (b) coded as incorrect, or (c) assigned fractional correct values. This procedure is widely practised amongst international large-scale assessment analysts \parencite{pohl:2014}. Secondly, \emph{imputation-based approaches} encompass corrected mean substitution, response function imputation, EM algorithm and multiple imputation (MI). \textcite{finch:2008} compared the performance of competing imputation-based methods and found MI to be the optimal procedure. MI considers (a) candidates' valid responses, (b) the responses of similar participants, and (c) observed information on covariates if a background model is available, in imputing the missing responses. This Bayesian approach generates multiple draws from parameters' posterior distributions to form correct standard errors \parencite{carpenter:2013}. MI also reallocates the missing-data burden from the analysis stage to the data preparation stage \parencite{reiter:2007}, therefore re-enabling subsequent inferences whose validity depends on complete-data statistical methods and software \parencite[][Chapter 4]{rubin:1987}. Lastly, the recently developed \emph{model-based approaches} include missing tendency in the IRT model when estimating item and person parameters via either (a) latent missing propensity \parencite{holman:2005, glas:2008, glas:2015,korobko:2008} or (b) manifest approach \parencite{rose:2010}.

Both MI and model-based approaches carry their corresponding costs. Studies interested in person parameters shall employ plausible values as the appropriate strategy \parencite{mislevy:1991,mislevy:1993}. But plausible values themselves are multiple imputations of (already multiple imputed) latent variables, causing a cascade of computation demand in the form of nested MI, limiting its wide use in practice such as large-scale assessments or competence tests \parencite{pohl:2014}. Model-based approaches, on the other hand, may generate biased parameter estimates should the missing propensity violate the unidimensionality assumption \parencite{rose:2013}. The current study focuses on \emph{item} parameters (i.e., subject difficulties) while treating person parameters as ``nuisance'' by integrating it out of the maximum likelihood, thereby avoids the MI cascade problem. The risk of committing unidimensionality violation at the missing propensity estimation stage, however, is material since it is not implausible to expect more than one behavioural patterns in young students' GPA choice decisions. It is based on these cost-benefit considerations that this project prefers the MI-based approach to missing data over the model-based procedure.