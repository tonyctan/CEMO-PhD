\section{Theoretical Framework}

\subsection{The Norwegian GPA System}

\subsection{Missing Data Treatment}

IRT item parameter estimations demand strong assumptions on the data missing mechanism. Joint, conditional and marginal maximum likelihood procedures are only valid under ``ignorable non-response'' conditions where missing propensities are related to neither item nor person parameters \parencite{molenaar:1995}. When test takers skip items after seeing their content, for example, the ignorablity condition is unlikely to hold \parencite{mislevy:1987}, neither are tests with items not reached due to time constraint \parencite{lord:1974, lord:1983}. In the current study involving Norway's GPA archive, missing records are not the result of randomly assigning candidates to subjects (not MCAR), nor are each candidate's missing GPAs independent of observed ones (not MAR). In fact, missing patterns are likely to be related to both personal capabilities and subject difficulties with low ability candidates self-selecting into easy subjects while difficult subjects attracting only high capability students. Resultantly, MML estimates of subject difficulties after marginalising personal parameters are no longer unbiased \parencite[][Table 2]{mislevy:1988}.

Literature has congregated into three main approaches for the purpose of addressing missing values. In the \emph{classical approaches}, missing responses can be (a) ignored and treated as non-administered, (b) coded as incorrect, or (c) assigned fractional correct values. This procedure is widely practised amongst international large-scale assessment analysts \parencite{pohl:2014}.

Valid inferences, fortunately, can be obtained based on complete-data statistical methods should ``proper'' imputation models are applied \parencite[][Chapter 4]{rubin:1987}.