\section{Conceptual Framework}

\subsection{The Norwegian Education and Assessment System}

The Norwegian education system is organised into three levels: primary school (Year 1--7) where formal grading is not practised, lower secondary school (Year 8--10) and upper secondary school (Year 11--13). During the first ten years of schooling (\textit{grunnskole}), students follow centralised national curricula with largely compulsory subjects plus a few electives. Upon successful completion of Year 10, students may choose between vocational and academic tracks for their upper secondary schools. The former is a two-year program (\textit{fagskole}) that prepares students for employment in a specific field, whereas the latter is a three-year program (\textit{videreg{\aa}ende oppl{\ae}ring}, VG1--3) that prepares students for university studies. In preparation for the merit-based tertiary entry requirements, academic track students carefully choose their VG subjects in order to maximise their final GPA scores. This study focuses on Year 10 GPA data where the impact of subject choice is minimal.

The GPA aims to provide a sum-score measure of a student's overall competency. For \textit{grunnskole} graduation purposes, the GPA is calculated as a weighted average of students' grades from all Year 10 subjects. The weights are determined by the number of hours spent on each subject, with the exception of Norwegian and English, which are weighted equally. Both teacher-assigned grades and exam grades are included in the GPA calculation, with each subject ranging from 1 (low competency) to 6 (outstanding). While every subject a student undertakes receives a teacher-assigned grade, Year 10 students are randomly assigned into participating in \emph{one} of the three written exams, Norwegian, English, and mathematics, hence receives only one written exam grade. Similarly, students are randomly assigned into participating in \emph{one} of the many oral exam options, including Norwegian, English, mathematics, and electives. A candidate's GPA is then computed by averaging the grades they have obtained, multiplying by 10, and rounding to two decimal places.