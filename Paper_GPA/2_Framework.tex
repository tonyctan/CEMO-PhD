\section{Conceptual Framework}

\subsection{The Norwegian Education and Assessment System}

The Norwegian education system is organised into three levels: primary school (Year 1--7) where formal grading is not practised, lower secondary school (Year 8--10) and upper secondary school (Year 11--13). During the first ten years of schooling (\textit{grunnskole}), students follow centralised national curricula with largely compulsory subjects plus some electives. Upon successful completion of Year 10, students may choose between vocational and academic tracks for their upper secondary schools. The former is a two-year program that prepares students for employment in a specific field, whereas the latter is a three-year program (\textit{videreg{\aa}ende oppl{\ae}ring}, VG1--3) that prepares students for university studies.

The grade point average (\textsc{gpa}) aims to provide a sum-score measure of a student's overall competency. For \textit{grunnskole} graduation purposes, the \textsc{gpa} is calculated as the unweighted average of students' grades from all Year 10 subjects. Both teacher-assigned grades and exam grades are included in the GPA calculation, with each subject ranging from 1 (low competency) to 6 (outstanding). While every compulsory subject receives a teacher-assigned grade, Year 10 students are randomly assigned into participating in \emph{one} of the three written exams (mathematics, Norwegian, and English), as well as \emph{one} oral exam (same as written exams, plus many electives). A candidate's GPA is then computed by averaging the grades they have obtained, multiplying by 10, and rounding to two decimal places.

\subsection{The \textit{manu}--\textit{mente} Clusters}

Not all \textsc{gpa} subjects target the same cognitive domain. While all subject demand students' cognitive input, some courses are undoubtedly more hand-on and practice-based such as physical education and food and health. We label the more hand-on subject in Latin as ``\textit{manu} subjects'' while the cognitive-demanding ones as ``\textit{mente} subjects''. [insert references suggested by Jose].