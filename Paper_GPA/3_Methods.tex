\section{Methods}

\subsection{Sample}

Say something about Norwegian registry data.

\subsection{Missing Value Treatment}

Missing patterns are not missing at random. If a candidate chose to do physics, he was also highly likely to have chosen advanced maths. So the presence and absence of data tend to group in clusters.

\subsection{Marginal Maximum Likelihood}

A unidemensional generalised partial credit model \parencite{muraki:1992} with the probability that Candidate $n$'s score in Subject $i$ ($x_{ni}$) is Grade $j$ ($j=0, \dots, m$) is given by
\begin{equation}\label{eqn:gpcm}
    p(x_{ni}=j | d_{ni} = 1; \theta_n) = \frac{\exp{j \alpha_i \theta_n - \sum_{h=1}^j \beta_{ih}} }{ 1+ \sum_{h=1}^m \exp{h \alpha_i \theta_h - \sum_{k=1}^h \beta_{ik}} },
\end{equation}
where $\theta_n$ is the unidemensional proficiency parameter that represents the overall proficiency of Candidate $n$.

In MML, a likelihood function ($\ell$) is maximised where the candidates' proficiency parameters ($\theta$) are integrated out of the likelihood. The marginal log-likelihood for a unidemensional GPCM is given by
\begin{equation}\label{eqn:ll}
    \ell = \sum_p \sum_{n | p} \log \int \prod_i p( x_{ni} = j | d_{ni}; \theta ) g(\theta; \mu_p, \sigma^2) \dd \theta,
\end{equation}
where $x_{ni}$ is the observed grade, $p( x_{ni} | d_{ni}; \theta )$ is equal to \cref{eqn:gpcm} evaluated at $x_{ni}$ if $d_{ni}=1$, and $p(x_{ni} | d_{ni}; \theta) = 1$ if $d_{ni} = 0$. In addition, $g(\theta; \mu_p, \sigma^2)$ is the normal pdf with mean $\mu_p$ and variance $\sigma^2$. The model can be identified by choosing standard normal $\mathcal{N}(0,1)$.