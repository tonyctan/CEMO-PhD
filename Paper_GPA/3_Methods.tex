\section{Methods}

\subsection{Sample}

Year 10 students' GPA (\textit{grunnskolepoeng}) and teacher-assigned grades (\textit{standpunktkarakter}) were extracted from the Norwegian register for the administrative year ending in June 2019 (\textit{avgangdato} = 201906). Attainment records were subsequently re-formatted with each row representing one candidate and each column being one subject. This process yielded 64,918 students and 200 subjects.

Under the Norwegian education system, students shall complete 13 compulsory subjects as well as electives. This study focuses on these compulsory subjects but excludes one course ``Norwegian as a Second Language'' due to large number of missings and its sensitivity to factors such as candidates' native languages. We apply equal treatment to courses instructed in Norwegian and in Sami language by merging these records. Twelve subjects are retained for our analysis: Written Norwegian (NORW), Oral Norwegian (NORO), Written English (ENGW), Oral English (ENGO), Mathematics (MATH), Natural Sciences (NATS), Social Sciences (SOCS), Religion (RELI), Music (MUSI), Arts and Handcraft (HAND), Food and Health (FOOD), and Physical Education (PHED). After dropping students without valid GPA records (data loss $n^-= 4,300$ cases, loss rate $r^- = 6.62\%$), we impose the selection criteria a) four or more records among NORW, NORO, ENGW, ENGO, MATH, NATS, SOCS ($n^- = 1,101$, $r^- = 1.82\%$), and b) three or more among RELI, MUSI, HAND, FOOD and PHED ($n^-= 1,787$, $r^- = 3\%$) in order to retain only cases with more observed information than missings. The final data set contains $n = 57,730$ students and $i = 12$ subjects. At this stage, the existence of missing data no longer poses any problems for our analyses thanks to sufficient overlap across subjects in the score matrix. The ability to deal with incomplete data is one major advantage of using the Rasch model for studying inter-subject comparability \parencite{he:2018}.



\subsubsection{Generalised Partial Credit Model (GPCM)}

A unidemensional generalised partial credit model \parencite{muraki:1992} with the probability that Candidate $n$'s score in Subject $i$ ($x_{ni}$) being Grade $j$ ($j=0, \dots, m$) is given by
\begin{equation}\label{eqn:gpcm}
    p \left( x_{ni}=j | d_{ni} = 1; \theta_n \right) = \frac{\exp{j \alpha_i \theta_n - \sum_{h=1}^j \beta_{ih}} }{ 1 + \sum_{h=1}^m \exp{h \alpha_i \theta_n - \sum_{k=1}^h \beta_{ik}} },
\end{equation}
where $\theta_n$ is the unidemensional proficiency parameter that represents the overall proficiency of Candidate $n$.

\subsubsection{Log-likelihood}

In MML, a likelihood function ($\ell$) is maximised where the candidates' proficiency parameters ($\theta$) are integrated out of the likelihood. The marginal log-likelihood for a unidemensional GPCM is given by
\begin{equation}\label{eqn:ll}
    \ell_\text{unidimensional} = \sum_p \sum_{n | p} \log \int \prod_i p( x_{ni} | d_{ni}; \theta ) g(\theta; \mu_p, \sigma^2) \dd \theta,
\end{equation}
where $x_{ni}$ is the observed grade, $p( \cdot )$ is equal to \cref{eqn:gpcm} evaluated at $x_{ni}$ if $d_{ni}=1$, and $p( \cdot ) = 1$ if $d_{ni} = 0$. In addition, $g(\theta; \mu_p, \sigma^2)$ is the normal pdf with mean $\mu_p$ and variance $\sigma^2$. The model can be identified by choosing a standard normal $\mathcal{N}(0,1)$ \parencite{korobko:2008}.

\subsubsection{Multidimensionality}

There exists strong believes among educational scientists that learners' proficiency is multidimensional, such as one proficiency factor for STEM subjects, for example, and another one for languages. If $F$ proficiency dimensions are required to model the grades, the proficiency can be represented by a vector of proficiency parameters $\m{\theta}_n=\left(\theta_{n1}, \cdots, \theta_{nF}\right)\Ts$ with the corresponding GPCM:
\begin{equation}\label{eqn:mgpcm}
    p \left( x_{ni} = j | d_{ni} = 1; \m{\theta}_n \right) =
    \frac{ \exp{ j \left( \sum_{f=1}^F \alpha_{if} \theta_{nf} \right) - \sum_{h=1}^j \beta_{ih} } }{ 1 + \sum_{h=1}^m \exp{ h \left( \sum_{f=1}^F \alpha_{if} \theta_{nf} \right) - \sum_{k=1}^h \beta_{ik} } }.
\end{equation}
with $\m{\theta}_n$ following a multivariate normal distribution with mean $\m{\mu}_p$ and variance-covariance matrix $\m{\Sigma}$. Similar to the unidemensional case, \cref{eqn:mgpcm} is identified by setting $\m{\mu}_p=\m{0}$ and $\m{\Sigma} = \m{I}$ the identity matrix. The log-likelihood of a multidimensional GPCM then becomes:
\begin{equation}\label{eqn:mll}
    \ell_\text{multidimensional} = \sum_{p} \sum_{n | p} \log \int \cdots \int \prod_i p( x_{ni} | d_{ni}; \m{\theta} ) g( \m{\theta}; \m{\mu}_p, \m{\Sigma} ) \dd \m{\theta},
\end{equation}
with each component sharing similar interpretations to the unidemensional counterpart in \cref{eqn:ll}.

\subsection{Interaction between Subject Choice and Proficiency}

Under the advisory of \textcite{korobko:2008}, a latent variable $\theta^+$ is introduced to reflect student's propensity of choosing a particular subject. Augmenting $\theta^+$ to $\m{\theta}=\left(\theta_1, \cdots, \theta_F\right)\Ts$ yields $\m{\theta}^+=\left(\theta_1, \cdots, \theta_F, \theta^+\right)\Ts$, with a corresponding marginal likelihood:
\begin{equation}\label{eqn:lli}
    \ell_\text{interaction} = \sum_p \sum_{ n | p} \log \int \cdots \int \prod_i \left[ p \left( x_{ni} | d_{ni}; \m{\theta} \right) p \left( d_{ni}; \theta^+ \right) \right] g(\m{\theta}^+; \m{\mu}_p, \m{\Sigma}) \dd \m{\theta}^+ .
\end{equation}