\section{Methods}

\subsection{Sample}

For this study, students' GPA records will be extracted from the Norwegian registry covering the period between 2009 (first year post-2006 reform data became available) and 2019 (last ``normal'' year before COVID). GDPR registration is lodged through the NSD Portal and the UiO ethics approval is also obtained. All data import, storage, and analyses are to be conducted within the secured infrastructure TSD provided by the UiO Central IT Division. TSD logs all activities and no data or results can be copied out of the restricted system without prior approval from project leaders.

Under the advisory of \textcite{he:2018}, subjects with fewer than 1,000 candidates and students taking fewer than two GPA subjects will be excluded from subsequent analyses. Each year's record (score matrix) will contain $N$ rows representing the number of valid candidates and $L$ columns reflecting the usable number of GPA subjects in that year. Since no student took all the GPA subjects, a large proportion of the score matrices will remain missing by design. The existence of missing data does not pose any problems for using the Rasch model as the model functions at the individual subject and as long as there is sufficient overlap across subjects in the score matrix. The ability to deal with incomplete data is one major advantage of using the Rasch model for studying inter-subject comparability.

\subsection{Missing Value Treatment}

Missing patterns are not missing at random. If a candidate chose to do physics, he was also highly likely to have chosen advanced maths. So the presence and absence of data tend to group in clusters.

\subsubsection{Subject Choice}

This study explicitly models candidates' subject choice decisions by introducing an indicator variable $d_{ni}$ such that
\begin{equation}\label{eqn:indicator}
    d_{ni}=
    \left\{
        \begin{aligned}
            &1\ \text{if Candidate $n$ chose Subject $i$}\\
            &0\ \text{if Candidate $n$ did not choose Subject $i$},
        \end{aligned}
    \right.
\end{equation}
for Candidate $n = 1, \dots, N$ and GPA Subject $i = 1, \dots, L$.

\subsubsection{Generalised Partial Credit Model (GPCM)}

A unidemensional generalised partial credit model \parencite{muraki:1992} with the probability that Candidate $n$'s score in Subject $i$ ($x_{ni}$) being Grade $j$ ($j=0, \dots, m$) is given by
\begin{equation}\label{eqn:gpcm}
    p \left( x_{ni}=j | d_{ni} = 1; \theta_n \right) = \frac{\exp{j \alpha_i \theta_n - \sum_{h=1}^j \beta_{ih}} }{ 1 + \sum_{h=1}^m \exp{h \alpha_i \theta_n - \sum_{k=1}^h \beta_{ik}} },
\end{equation}
where $\theta_n$ is the unidemensional proficiency parameter that represents the overall proficiency of Candidate $n$.

\subsubsection{Log-likelihood}

In MML, a likelihood function ($\ell$) is maximised where the candidates' proficiency parameters ($\theta$) are integrated out of the likelihood. The marginal log-likelihood for a unidemensional GPCM is given by
\begin{equation}\label{eqn:ll}
    \ell_\text{unidimensional} = \sum_p \sum_{n | p} \log \int \prod_i p( x_{ni} | d_{ni}; \theta ) g(\theta; \mu_p, \sigma^2) \dd \theta,
\end{equation}
where $x_{ni}$ is the observed grade, $p( \cdot )$ is equal to \cref{eqn:gpcm} evaluated at $x_{ni}$ if $d_{ni}=1$, and $p( \cdot ) = 1$ if $d_{ni} = 0$. In addition, $g(\theta; \mu_p, \sigma^2)$ is the normal pdf with mean $\mu_p$ and variance $\sigma^2$. The model can be identified by choosing a standard normal $\mathcal{N}(0,1)$ \parencite{korobko:2008}.

\subsubsection{Multidimensionality}

There exists strong believes among educational scientists that learners' proficiency is multidimensional, such as one proficiency factor for STEM subjects, for example, and another one for languages. If $F$ proficiency dimensions are required to model the grades, the proficiency can be represented by a vector of proficiency parameters $\m{\theta}_n=\left(\theta_{n1}, \cdots, \theta_{nF}\right)\Ts$ with the corresponding GPCM:
\begin{equation}\label{eqn:mgpcm}
    p \left( x_{ni} = j | d_{ni} = 1; \m{\theta}_n \right) =
    \frac{ \exp{ j \left( \sum_{f=1}^F \alpha_{if} \theta_{nf} \right) - \sum_{h=1}^j \beta_{ih} } }{ 1 + \sum_{h=1}^m \exp{ h \left( \sum_{f=1}^F \alpha_{if} \theta_{nf} \right) - \sum_{k=1}^h \beta_{ik} } }.
\end{equation}
with $\m{\theta}_n$ following a multivariate normal distribution with mean $\m{\mu}_p$ and variance-covariance matrix $\m{\Sigma}$. Similar to the unidemensional case, \cref{eqn:mgpcm} is identified by setting $\m{\mu}_p=\m{0}$ and $\m{\Sigma} = \m{I}$ the identity matrix. The log-likelihood of a multidimensional GPCM then becomes:
\begin{equation}\label{eqn:mll}
    \ell_\text{multidimensional} = \sum_{p} \sum_{n | p} \log \int \cdots \int \prod_i p( x_{ni} | d_{ni}; \m{\theta} ) g( \m{\theta}; \m{\mu}_p, \m{\Sigma} ) \dd \m{\theta},
\end{equation}
with each component sharing similar interpretations to the unidemensional counterpart in \cref{eqn:ll}.

\subsection{Interaction between Subject Choice and Proficiency}

Under the advisory of \textcite{korobko:2008}, a latent variable $\theta^+$ is introduced to reflect student's propensity of choosing a particular subject. Augmenting $\theta^+$ to $\m{\theta}=\left(\theta_1, \cdots, \theta_F\right)\Ts$ yields $\m{\theta}^+=\left(\theta_1, \cdots, \theta_F, \theta^+\right)\Ts$, with a corresponding marginal likelihood:
\begin{equation}\label{eqn:lli}
    \ell_\text{interaction} = \sum_p \sum_{ n | p} \log \int \cdots \int \prod_i \left[ p \left( x_{ni} | d_{ni}; \m{\theta} \right) p \left( d_{ni}; \theta^+ \right) \right] g(\m{\theta}^+; \m{\mu}_p, \m{\Sigma}) \dd \m{\theta}^+ .
\end{equation}