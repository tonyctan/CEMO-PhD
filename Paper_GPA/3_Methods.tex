\section{Methods}

\subsection{Population}

This study retains the entire cohort of Norway's Year 10 students graduating in 2019 as its targeted population. Students' GPA (\textit{grunnskolepoeng}), teacher-assigned grades (\textit{standpunktkarakter}), as well as written (\textit{SKR}) and oral (\textit{MUN}) exam grades were extracted from the national register. This data source is unique because it is the \emph{population}, rather than samples, that forms our unit of analysis. Attainment records were then re-shaped into the format that each candidate occupies one row and each subject is represented by one column. This process led to a preliminary data set of 64,918 students and 200 subjects. Next, $4,300$ students without valid GPA records were excluded from subsequent analyses, representing a loss rate of $6.62\%$. Seventeen subjects were retained based on these inclusion criteria:

\subsubsection{Teacher-assigned Grades (12 subjects)}

Under the Norwegian education system, Year 10 students shall complete 13 compulsory subjects as well as electives. This study included all compulsory subjects except for Sidem{\aa}l.\footnote{The Norwegian language has two official written forms: Bokm{\aa}l and Nynorsk, with the former being more prevalent in the media. Students growing up in one written form must enroll the other as their Sidem{\aa}l, unless Norwegian is not their native language. Nynorsk users tend to have easier time in Sidem{\aa}l due to existing exposure to Bokm{\aa}l. Bokm{\aa}l users, on the other hand, find Nynorsk more challenging while fulfilling Sidem{\aa}l. Since Sidem{\aa}l contains two sub-cohorts with distinct difficulty profiles, we opt not to include this subject in our analyses.} We applied equal treatment to courses instructed in Norwegian and in Sami language by merging these records.\footnote{For example, \href{https://www.udir.no/kl06/nat0010}{NAT0010 Naturfag 10. {\aa}rstrinn} and \href{https://www.udir.no/kl06/nat0020}{NAT0020 Naturfag, samisk plan, 10. {\aa}rstrinn} were merged into one subject Natural Sciences. If academic results were available from both instruction languages, we retained the higher grades during merging.} Twelve teacher-assigned grades were included for our analysis: Written Norwegian (NORW), Oral Norwegian (NORO), Written English (ENGW), Oral English (ENGO), Mathematics (MATH), Natural Sciences (NATS), Social Sciences (SOCS), Religion (RELI), Music (MUSI), Arts and Handcraft (HAND), Food and Health (FOOD), and Physical Education (PHED).

\subsubsection{Written Exam Grades (3 subjects)}

Norway uses a lottery draw to randomly assign Year 10 students to participate in \emph{one} of the following three written exams: Norwegian (E-NORW), English (E-ENGW) and mathematics (E-MATH). This ``planned missingness'' implies that although numeral in quantity, the unobserved exam grades can be safely modelled under the missing completely at random (MCAR) assumption \parencite{little:2019}.\footnote{Even if the lottery is less than perfectly random, Rasch models are still valid under the weaker assumption of missing at random (MAR), as long as one is satisfied that missing propensities are unrelated to item or person parameters \parencite{molenaar:1995}.} Rasch models have a major advantage for handling missing values thanks to the sufficient overlap across subjects in the score matrix \parencite{he:2018}.

\subsubsection{Oral Exam Grades (2 subjects)}

Year 10's oral exams consist of the same three subjects as in written exams, plus a wide selection such as natural and social sciences, with students being randomly assigned into \emph{one} oral exam by lottery. In order to better match teacher-assigned grades, only Oral Norwegian (E-NORO) and Oral English (E-ENGO) were included in this study. Since students are spread thinly across many oral exam subjects, E-NORO and E-ENGO appeared more sparse than their teacher-assigned counterparts, leading to larger confidence intervals in subsequent analyses.
