\newpage
\section{Methods}

\subsection{Student Population and Exam Subjects}

This study retained the entire cohort of Year 10 graduates from Norway's lower secondary education (\textit{grunnskole}) in 2019 ($N_0 = 64,918$). Students' teacher-assigned grades, written-, and oral-exam grades were extracted from the national registers. This data source is unique because it is the \emph{population}, rather than samples, that forms our bases of analysis. Next, $4,300$ students without valid \textsc{gpa} records were excluded from subsequent analyses, representing a loss rate of $6.62\%$.

Year 10 students in Norway should complete 13 compulsory subjects as well as electives. The compulsory subjects are: mathematics (\textsc{math}), written Norwegian \textit{hovedm{\aa}l} (main, \textsc{norw}), oral Norwegian (\textsc{noro}), written English (\textsc{engw}), oral English (\textsc{engo}), natural sciences (\textsc{nats}), social sciences (\textsc{socs}), religion (\textsc{reli}), music (\textsc{musi}), arts and handcraft (\textsc{hand}), physical education (\textsc{phed}), and food and health (\textsc{food}). This study included all compulsory subjects except the secondary written Norwegian language \textit{sidem{\aa}l} due to non-random missingness. Norwegian has two written forms and students can be exempt from the one that is not their main language based on considerations such as bilingualism [re-write the hovedm{\aa}l/sidem{\aa}l complexity using Astrid's text].

Written exams at the Year 10-level involve equal-probability sampling. A lottery system randomly assigns students into \emph{one} of the three written subjects: Norwegian (\textsc{nor\_w}),  English (\textsc{eng\_w}), and mathematics (\textsc{mat\_w}). This planned missingness enables written exam grades to be modelled under the missing completely at random (\textsc{mcar}) assumption \parencite{little:2019}. Even if the lottery is less than perfectly random, Rasch models are still valid under the weaker assumption of missing at random (\textsc{mar}), hence ``ignorable'' \parencite{molenaar:1995}, as long as the missing propensities are unrelated to either item- or person-parameters. This practice is in agreement with previous studies \parencite[e.g., ][]{he:2018} that utilised Rasch models for handling missing values for score matrices with sufficient subject overlaps.

Similar to written exams, oral exams consist of the same three subjects plus a wide selection of locally-specified electives (e.g., religion, social or natural sciences) with students being randomly assigned into \emph{one} oral exam by lottery. In order to form teacher assigned-, written-, and oral-exam grade comparisons, we selected oral Norwegian (\textsc{nor\_o}), oral English (\textsc{eng\_o}), and oral mathematics (\textsc{mat\_o}) for analyses.

In summary, this study contains a final population size of $N = 60,618$ students. Twelve teacher-assigned grades, three written- and three oral-exam grades formed 18 ``Rasch items'' for subsequent \textsc{irt} modelling. Detailed description about each subject is available in \cref{tab:descriptive}.

\subsection{Rasch Model and Difficulty Measures}

A Rasch model is a unidimensional \textsc{irt} model with the assumption that the probability of a student's correct response to an item is a function of the student's ability and the item's difficulty \parencite{rasch:1960}. Rasch models are powerful tools for analysing both dichotomous and polytomous ordinal data thanks to its ability to accommodate missing values and its capability to estimate person- and item-parameters simultaneously \parencite{deayala:2022}. Similar to \textcite{he:2018}, this study models the 18 Norwegian \textsc{gpa} subjects as Rasch items---manifest outcomes of each candidate's latent scholastic capability---and consider two difficulty measures. We operationalise each subject's \emph{overall difficulty} as the expected grade for a candidate possessing average competency ($\E{x \mid \theta = 0}$). We further decompose each \textsc{gpa}  subject's \emph{grade-level difficulties} by examining the competencies students need to transition onto the next grade level (difficulty thresholds, $\delta_k$).

Multiple specifications of the Rasch models have been proposed to address different analytical demands. \poscite{masters:1982} partial credit model (\textsc{pcm}) is particularly suitable for the current study given \textsc{gpa}s in Norway are \emph{unweighted} sum scores across all subjects. The PCM model states that for a polytomous item with a maximum available score of $m$, the probability $\p{\theta, x}$ of a candidate with latent ability $\theta$ scoring $x$ on a subject can be expressed as
\begin{equation}\label{eqn:rasch}
    \p{\theta, x} =
    \left\{
        \begin{aligned}
            & \frac{1}{1 + \sum_{j=1}^5 \exp{\sum_{k=1}^j (\theta - \delta_k)}}\ \text{for } x = 0, \\
            & \frac{\exp{\sum_{k=1}^x (\theta - \delta_k)}}{1 + \sum_{j=1}^5 \exp{\sum_{k=1}^j (\theta - \delta_k)}}\ \text{for } x = 1, \dots, 5,
        \end{aligned}
    \right.
\end{equation}
where $\theta$ is the latent competency of the candidate, and $\delta_k$ is the location of the $k$-th threshold on the latent ability continuum. Since Norwegian students receive grades between 1 and 6, $m = [0,5]$ in this study. In addition, when each two adjacent grade curves intersect, a subject with six grades would generate five thresholds ($\delta_1, \dots, \delta_5$).

\subsection{Estimation Procedures}

Although Rasch models accommodate missing values well, certain output such as infit and outfit statistics are only computable under full data matrices \parencite{chalmers:2022}. We therefore apply multiple imputations (\textsc{mi}) to the score matrix under the \textsc{mcar} assumption \parencite{little:2019}. Each of the ten \textsc{mi} datasets is then analysed separately using the \textsf{R} package \textsf{mirt} \parencite[Version 1.38.1,][]{chalmers:2022}, then pulled together following Rubin's rules \parencite{rubin:1987}.
