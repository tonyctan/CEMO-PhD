\documentclass[
    % Options for apa7
    a4paper,                % Paper size
    11pt,                   % Font size
    stu,                    % Format as assignment
    donotrepeattitle,       % Start body text without repeating title
    floatsintext,           % Insert tables and figures with texts
    biblatex,               % Use BibLaTeX for references
% Options for hyperref
    colorlinks=true,        % Colour all links
    linkcolor=red,          % Cross-references in red
    anchorcolor=black,      % Keep anchors black
    citecolor=blue,         % In-text-referencs in blue
    urlcolor=blue,          % DOIs and URLs are in blue
    bookmarks=true,         % Generate bookmarks for PDF readers
    bookmarksopen=false,    % Expand all bookmarks as default
    bookmarksnumbered=true, % Keep section number in bookmarks
    % Options for xcolor
    dvipsnames              % Use colour BrickRed and PineGreen
]{apa7}

\title{Identifying Inter-subject Difficulties in Norwegian GPA Data}
\authorsnames{Tony C. A. Tan}
\authorsaffiliations{{Centre for Educational Measurement, University of Oslo}}
\course{Continuous Draft}
\professor{Prof Rolf V. Olsen \& Dr Astrid M. J. Sands{\o}r}
\duedate{H{\o}st 2022}

% Load ttbeamer template depending on operating system
\usepackage{ifplatform}
\ifwindows
    \usepackage{M:/pc/Dokumenter/tt}
\fi
\iflinux
    \usepackage{/home/tony/uio/pc/Dokumenter/tt}
\fi
\ifmacosx
    \usepackage{/Users/tctan/uio/pc/Dokumenter/tt}
\fi

\begin{document}
\maketitle

Consider the original regression
\begin{equation}\label{eq1}
    y = \hat{\beta}_0 + \hat{\beta}_1 x_1 + \hat{\beta}_2 x_2 + e,
\end{equation}
where $e$ is the residual from this regression.

Johan wanted two regressions
\begin{equation}\label{eq2}
    x_1 = \hat{\gamma}_0 + \hat{\gamma}_1 x_2 + g,
\end{equation}

\begin{equation}\label{eq3}
    y = \hat{\delta}_0 + \hat{\delta}_1 g + h,
\end{equation}
and wanted you to show $\hat{\delta}_1 = \hat{\beta_1}$.

From \cref{eq2}, we can save its residual as
\begin{equation}\label{eq4}
    g = x_1 - \hat{\gamma}_0 - \hat{\gamma}_1 x_2.
\end{equation}

Substitute \cref{eq4} into \cref{eq3}:
\begin{equation}\label{eq5}
    \begin{aligned}
        y &= \hat{\delta}_0 + \hat{\delta}_1 (x_1 - \hat{\gamma}_0 - \hat{\gamma}_1  x_2) + h \\
        &= \hat{\delta}_0 + \hat{\delta}_1 x_1 - \hat{\delta}_1 \hat{\gamma}_0 - \hat{\delta}_1 \hat{\gamma}_1 x_2 + h \\
        &= \left( \hat{\delta}_0 - \hat{\delta}_1 \hat{\gamma}_0 \right) + \hat{\delta}_1 x_1 + \left( - \hat{\delta}_1 \hat{\gamma}_1 \right) x_2 + h.
    \end{aligned}
\end{equation}

Compare \cref{eq5} with \cref{eq1}, we see that $\hat{\delta}_1 = \hat{\beta_1}$, which is what Johan wanted.


\end{document}
