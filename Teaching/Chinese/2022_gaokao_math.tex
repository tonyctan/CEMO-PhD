% !TEX encoding = UTF-8
% !TEX program = lualatex
\documentclass[a4paper,12pt]{article}

\usepackage{luatexja-fontspec}
\setmainjfont{FandolSong}

\usepackage{xcolor}

\title{《2022年高考数学》}

\begin{document}
\maketitle

%在 Lua\TeX{} 中正常地使用中文。获得自动的\textbf{字体选择},标点“压缩”,以及正确的断行处理等特性。
\setcounter{section}{19}

\textbf{20. (12分)}

一医疗团队为研究某地的一种地方性疾病与当地居民的卫生习惯(卫生习惯分为良好和不够良好两类)的关系,在已患该病的病例中随机调查了100例(成为病例组),同时在未患该疾病的人群中随机调查了100人(称为对照组),得到如下数据:
\begin{center}
    \begin{tabular}{c c c}
    \hline
                & 不够良好 & 良好 \\
    \hline
        病例组  & 40    & 60 \\
        对照组  & 10    & 90 \\
    \hline
    \end{tabular}
\end{center}

(1)能否有99\%的把握认为患该疾病群体与未患该疾病群体的卫生习惯有差异?

\color{blue}
解:

本题的统计学问题是:“病例组”和“对照组”(横行)是否与“良好”和“不够良好”的卫生习惯(纵列)相互独立。

为简便起见,按第(2)小题的标记,将事件“选到的人卫生习惯不够良好”简记为$A$,该事件的互补事件$\overline{A}$则为“选到的人卫生习惯良好”。相应地,将事件“选到的人患有该疾病”简记为$B$,“选到的人未患有该疾病”为$\overline{B}$。

分布表因此可简记为:

\begin{tabular}{c c c}
    \hline
                & $A$ & $\overline{A}$ \\
    \hline
        $B$  & 40    & 60 \\
        $\overline{B}$  & 10    & 90 \\
    \hline
\end{tabular}
$\Rightarrow$
\begin{tabular}{c c c c}
    \hline
                & $A$ & $\overline{A}$ & \\
    \hline
        $B$  & 40    & 60 & \textcolor{red}{100}\\
        $\overline{B}$  & 10    & 90 & \textcolor{red}{100}\\
    \hline
        & \textcolor{red}{50} & \textcolor{red}{150} & \textcolor{red}{200}
\end{tabular}

\end{document}