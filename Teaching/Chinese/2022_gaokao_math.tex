% !TEX encoding = UTF-8
% !TEX program = lualatex
\documentclass[a4paper,12pt]{article}

% Typesetting Chinese
\usepackage{luatexja-fontspec}
\setmainjfont{FandolSong}

% Adjust space before and after tables
\usepackage{verbatimbox}
% Enable text colours
\usepackage{xcolor}
% Enable maths
\usepackage{amsmath,amssymb}

% Increase row height
\renewcommand{\arraystretch}{1.5}

% Enable slash fractions
\newcommand{\slfrac}[2]{\left.#1\middle/#2\right.}
% Print probability symbol as "empty P"
\newcommand{\p}[1]{\mathbb{P}\left(#1\right)}
\newcommand{\ph}[1]{\widehat{\mathbb{P}}\left(#1\right)}
% Override bar as overline
\renewcommand{\bar}[1]{\overline{#1}}

\title{《2022年高考数学》}
\author{}
\date{}

\begin{document}
\maketitle

\setcounter{section}{19}

\textbf{20. (12分)}

一医疗团队为研究某地的一种地方性疾病与当地居民的卫生习惯(卫生习惯分为良好和不够良好两类)的关系,在已患该病的病例中随机调查了100例(成为病例组),同时在未患该疾病的人群中随机调查了100人(称为对照组),得到如下数据:
\begin{center}
    \begin{tabular}{c c c c}
    \hline
                & $A=$不够良好 & $\bar{A}=$良好 & \\
    \hline
        $B=$病例组  & $a = 40$    & $b = 60$ & \\
        $\bar{B}=$对照组  & $c = 10$    & $d = 90$ & \\
    \hline
        & & & $n = 200$
    \end{tabular}
\end{center}

附:

\[ \chi^2 = \frac{n(ad - bc)^2}{(a+b)(c+d)(a+c)(b+d)} \]

\begin{center}
    \begin{tabular}{c | c c c}
        $\p{\chi^2_1 \geq k}$ & .05 & .01 & .001\\
    \hline
        $k$  & 3.841 & 6.635 & 10.828\\
    \end{tabular}
\end{center}

\newpage

(1)能否有99\%的把握认为患该疾病群体与未患该疾病群体的卫生习惯有差异?

\color{blue}
解:
本题的统计学问题是:

随机变量
$$\text{患病}=\{\text{“病例组”;}\text{“对照组”}\}(横行)$$
是否与随机变量
$$\text{卫生习惯}=\{\text{“不够良好”;}\text{“良好”}\}(纵列)$$
相互独立。

$\chi^2$独立性假设检验可以回答这个问题。

\textbf{第一步:声明原假设$H_0$与备择假设$H_A$}

\begin{equation*}
    \left\{
        \begin{aligned}
            H_0 &: \text{患病与卫生习惯互为独立。} \\
            H_A &: \text{患病与卫生习惯不独立。}
        \end{aligned}
    \right.
\end{equation*}

\textbf{第二步:计算检验统计量$\chi^2$(套公式)}

\begin{equation*}
    \begin{aligned}
        \chi^2 &= \frac{n(ad - bc)^2}{(a+b)(c+d)(a+c)(b+d)}\\
        &= \frac{200 \times (40 \times 90 - 60 \times 10)^2}{(40 + 60) \times (10 + 90) \times (40 + 10) \times (60 + 90)}\\
        &= 24。
    \end{aligned}
\end{equation*}

\textbf{第三步:计算拒绝域}

自由度$\nu = (\text{行数} - 1) \times (\text{列数} - 1) = (2-1)\times(2-1) = 1$。

查表得$\chi^2_\text{临界}(\nu=1,\alpha=.01) = 6.635$,即拒绝域为$(6.635,+\infty)$。

\textbf{第四步:判断假设}

由于检验统计量$\chi^2 = 24$落入拒绝域$(6.635,+\infty)$,故拒绝原假设,继而下结论:在99\%显著水平,有把握认定患该疾病群体与未患该疾病群体的卫生习惯有差异。

\newpage

\color{black}
(2)从该地的人群中任选一人,$A$表示事件“选到的人卫生习惯不够良好”,$B$表示事件“选到的人患有该疾病”,
\[ \frac{\p{B|A}}{\p{\bar{B}|A}} \text{ 与 } \frac{\p{B|\bar{A}}}{\p{\bar{B}|\bar{A}}} \]
的比值是卫生习惯不够良好对患该疾病风险程度的一项度量指标,记该指标为$R$。

(i)证明:
\begin{equation*}
    R = \frac{\p{A|B}}{\p{\bar{A}|B}} \cdot \frac{\p{\bar{A}|\bar{B}}}{\p{A|\bar{B}}}。
\end{equation*}

\color{blue}
证明:
依题意:
\[ R = \slfrac{\frac{\p{B|A}}{\p{\bar{B}|A}}}{\frac{\p{B|\bar{A}}}{\p{\bar{B}|\bar{A}}}} \]

$R$中的每个因子都可以根据Bayes公式展开,例如:
\[ \p{B|A} = \p{A|B} \cdot \frac{\p{B}}{\p{A}} \]

因此,$R$的表达式可被展开为:
\begin{equation*}
    \begin{aligned}
        R &= \slfrac{\frac{\p{B|A}}{\p{\bar{B}|A}}}{\frac{\p{B|\bar{A}}}{\p{\bar{B}|\bar{A}}}} \\
        &= \frac{\p{B|A}}{\p{\bar{B}|A}} \cdot \frac{\p{\bar{B}|\bar{A}}}{\p{B|\bar{A}}} \\
        &= \frac{\p{A|B} \cdot \frac{\p{B}}{\p{A}}}{\p{A|\bar{B}} \cdot \frac{\p{\bar{B}}}{\p{A}}} \cdot \frac{\p{\bar{A}|\bar{B}} \cdot \frac{\p{\bar{B}}}{\p{\bar{A}}}}{\p{\bar{A}|B} \cdot \frac{\p{B}}{\p{\bar{A}}}} \\
        &= \frac{\p{A|B}}{\p{\bar{A}|B}} \cdot \frac{\p{\bar{A}|\bar{B}}}{\p{A|\bar{B}}} \text{(繁分式相互约分,然后对调分母位置)}
    \end{aligned}
\end{equation*}

证毕。

\newpage

\color{black}
(ii)利用该调查数据,给出$\p{A|B}$、$\p{A|\bar{B}}$的估计值,并利用(i)的结果给出$R$的估计值。

\color{blue}
解:
根据条件概率定义式:
\[ \ph{A|B} = \frac{\ph{A \cap B}}{\ph{B}} = \frac{40/200}{(40+60)/200} = 0.4, \]

\[ \ph{A|\bar{B}} = \frac{\ph{A \cap \bar{B}}}{\ph{\bar{B}}} = \frac{10/200}{(10+90)/200} = 0.1, \]

因而:

\[ \widehat{R} = \frac{\ph{A|B}}{\ph{\bar{A}|B}} \cdot \frac{\ph{\bar{A}|\bar{B}}}{\ph{A|\bar{B}}} = \frac{0.4}{1-0.4} \cdot \frac{1-0.1}{0.1} = 6。 \]

\end{document}