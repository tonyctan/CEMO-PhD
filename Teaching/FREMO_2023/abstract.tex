Identifying PISA test disengagement using timing data and decoy items

Not all students respond to their PISA questionnaires with equal diligence. Disengaged students produce data that appear valid but are not representative of their true abilities. Accepting all response data at their face values undermines the validity of subsequent analyses, which are further used to compare countries and to evaluate the effectiveness of educational policies. In this study, we propose a two-step identification method to highlight disengaged participants. Using 2018 financial literacy's response data and timing data, we firstly identify students with disengaged rapid response patterns. We then apply a second round of filtering by examining students' responses to the decoy items that engaged students would answer ``never'' while random responders were more likely to report higher frequencies. Preliminary analyses suggested over 13 percent of students as disengaged. We then use the filtered data to re-examine the relationship between financial literacy and socio-economic status. Results showing that such relationship morphed significantly for engaged students subsample than for the overall sample, further corroborating the non-ignorability of disengaged candidates for analyses and policy purposes. We conclude that the proposed identification method involving decoy items can be used to identify disengaged students in other PISA domains.
