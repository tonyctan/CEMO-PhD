\chapter[Derivation of Moderated Mediation Effect]{Derivation of Moderated\\Mediation Effect}

\section{Models with Mediators Only}

Consider a SEM model 
%with three independent variables ($X_1$, $X_2$ and $X_3$) and two mediators ($M_1$ and $M_2$) as
shown in \cref{fig:moderator} (excluding any paths in green), where
\begin{equation*}
    \left\{
    \begin{aligned}
        Y &= \mu_0 + b_1M_1 + b_2M_2 + c_1X_1 + c_2X_2 + c_3X_3\\
        M_1 &= \mu_1 + a_{11}X_1 + a_{21}X_2 + a_{31}X_3\\
        M_2 &= \mu_2 + a_{12}X_1 + a_{22}X_2 + a_{32}X_3
    \end{aligned}
    \right.
\end{equation*}
or, in matrix form
\begin{equation}
    \left\{
    \begin{aligned}
        Y &= \mu_0 + \T{b}\m{m} + \T{c}\m{x}\\
        \m{m} &= \m{\mu} + \T{A}\m{x}
    \end{aligned}
    \right.
\end{equation}
where
\begin{equation*}\label{eq:0}
    \md{x}{3}{1} =
        \begin{bmatrix}
            X_1\\
            X_2\\
            X_3
        \end{bmatrix},\ 
    \md{m}{2}{1} =
        \begin{bmatrix}
            M_1\\
            M_2
        \end{bmatrix},\ 
    \md{b}{2}{1} =
        \begin{pmatrix}
            b_1\\
            b_2
        \end{pmatrix},\ 
    \md{c}{3}{1} =
        \begin{pmatrix}
            c_1\\
            c_2\\
            c_3
        \end{pmatrix},\ 
    \md{\mu}{2}{1} =
        \begin{pmatrix}
            \mu_1\\
            \mu_2
        \end{pmatrix}\text{and}\ 
    \md{A}{3}{2} =
        \begin{pmatrix}
            a_{11}    &a_{12}\\
            a_{21}    &a_{22}\\
            a_{31}    &a_{32}
        \end{pmatrix}
\end{equation*}

\cref{eq:0} can be written as a total equation:
\begin{equation}\label{eq:tot0}
    Y = \mu_0 + \T{b}\m{\mu} + \T{b}\T{A}\m{x} + \T{c}\m{x} = \left( \mu_0 + \T{b}\m{\mu} \right) + \T{x} \left( \m{A}\m{b} + \m{c} \right)
\end{equation}
where $\mu_0 + \T{b}\m{\mu}$ is the intercept, $\m{A}\m{b}$ is the indirect effect and $\m{c}$ is the direct effect.

\section{Models with Moderated Mediators}

Now introduce two moderators $D_1$ and $D_2$ (green paths in \cref{fig:moderator}).

In scalar notation:
\begin{equation*}
    \begin{aligned}
        Y_\text{mod} &= \mu_0 + b_1M_1 + b_2M_2 + c_1X_1 + c_2X_2 + c_3X_3\\
        &+ f_1D_1 + f_2D_2\\
        &+ g_{11}X_1D_1 + g_{12}X_1D_2\\
        &+ g_{21}X_2D_1 + g_{22}X_2D_2\\
        &+ g_{31}X_3D_1 + g_{32}X_1D_2\\
        &+ h_{11}M_1D_1 + h_{12}M_1D_2\\
        &+ h_{21}M_2D_1 + h_{22}M_2D_2
    \end{aligned}
\end{equation*}
and in matrix notation:
\begin{equation}\label{eq:mod}
    Y_\text{mod} = \mu_0 + \T{b}\m{m} + \T{c}\m{x} + \T{f}\m{d} + \tr{\T{G}\m{x}\T{d}} + \tr{\T{H}\m{m}\T{d}}
\end{equation}
where,
\begin{equation*}
    \md{f}{2}{1} =
        \begin{pmatrix}
            f_1\\
            f_2
        \end{pmatrix},\ 
    \md{d}{2}{1} =
        \begin{bmatrix}
            D_1\\
            D_2
        \end{bmatrix},\ 
    \md{G}{3}{2} =
        \begin{pmatrix}
            g_{11}  &g_{12}\\
            g_{21}  &g_{22}\\
            g_{31}  &g_{32}
        \end{pmatrix},\ 
    \md{H}{2}{2} =
    \begin{pmatrix}
        h_{11}  &h_{12}\\
        h_{21}  &h_{22}
    \end{pmatrix},
\end{equation*}
and $\tr{\cdot}$ is the trace operator.

Since $\m{m} = \m{\mu} + \T{A}\m{x}$, \cref{eq:mod} can be expanded into:
\begin{equation}\label{eq:totmod}
    \begin{aligned}
        Y_\text{mod} &= \mu_0 + \T{b}\m{\mu} + \T{b}\T{A}\m{x} + \T{c}\m{x} + \T{f}\m{d} + \tr{\T{G}\m{x}\T{d}} + \tr{\T{H}\m{\mu}\T{d}} + \tr{\T{H}\T{A}\m{x}\T{d}}\\
        &= \left[ \mu_0 + \T{b}\m{\mu} + \T{f}\m{d} + \tr{\T{H}\m{\mu}\T{d}} \right] + \left[ \left( \T{b}\T{A} + \T{c} \right) \m{x} + \tr{\T{d}\left( \T{G} + \T{H}\T{A} \right) \m{x}} \right]\\
        &= \left[ \mu_0 + \T{b}\m{\mu} + \T{f}\m{d} + \tr{\T{H}\m{\mu}\T{d}} \right] + \left[ \left( \T{b}\T{A} + \T{c} \right) \m{x} + \T{d}\left( \T{G} + \T{H}\T{A} \right) \m{x} \right]\\
        &= \left[ \mu_0 + \T{b}\m{\mu} + \T{f}\m{d} + \tr{\T{H}\m{\mu}\T{d}} \right] + \T{x} \left[ \m{A}\m{b} + \m{c} + \m{G}\m{d} + \m{A}\m{H}\m{d} \right]\\
        &= \left[ \mu_0 + \T{b}\m{\mu} + \T{f}\m{d} + \tr{\T{H}\m{\mu}\T{d}} \right] + \T{x} \left[ \m{A} \left( \m{b} + \m{H}\m{d} \right) + \left( \m{c} + \m{G}\m{d} \right) \right]
    \end{aligned}
\end{equation}

\cref{eq:totmod} differs from \cref{eq:tot0} by one extra term $\m{f}\T{d} + \tr{\T{H}\m{\mu}\T{d}}$ in the intercept. The indirect effect $\m{A}\m{b}$ expanded to $\m{A} \left( \m{b} + \m{H}\m{d} \right)$ as a result of introducing the moderators and the direct effect grows from $\m{c}$ to $\m{c} + \m{G}\m{d}$.

\newpage

Expand the indirect and direct effects back to their scalar forms:

\begin{equation*}
    \begin{aligned}
        &\text{indirect effects}\\
        = &\m{A} \left( \m{b} + \m{H}\m{d} \right)\\
        = &\begin{pmatrix}
            a_{11}    &a_{12}\\
            a_{21}    &a_{22}\\
            a_{31}    &a_{32}
        \end{pmatrix}
        \left[\begin{pmatrix}
            b_1\\
            b_2
        \end{pmatrix} +
        \begin{pmatrix}
            h_{11}  &h_{12}\\
            h_{21}  &h_{22}
        \end{pmatrix}
        \begin{bmatrix}
            D_1\\
            D_2
        \end{bmatrix}
        \right]\\
        = &\begin{pmatrix}
            a_{11}    &a_{12}\\
            a_{21}    &a_{22}\\
            a_{31}    &a_{32}
        \end{pmatrix}
        \begin{pmatrix}
            b_1 + h_{11}D_1 + h_{12}D_2\\
            b_2 + h_{21}D_1 + h_{22}D_2
        \end{pmatrix}\\
        = &\begin{pmatrix}
            a_{11}b_1 + a_{11}h_{11}D_1 + a_{11}h_{12}D_2 +
            a_{12}b_2 + a_{12}h_{21}D_1 + a_{12}h_{22}D_2\\
            a_{21}b_1 + a_{21}h_{11}D_1 + a_{21}h_{12}D_2 +
            a_{22}b_2 + a_{22}h_{21}D_1 + a_{22}h_{22}D_2\\
            a_{31}b_1 + a_{31}h_{11}D_1 + a_{31}h_{12}D_2 +
            a_{32}b_2 + a_{32}h_{21}D_1 + a_{32}h_{22}D_2
        \end{pmatrix};\\
        &\text{direct effects}\\
        = &\m{c} + \m{G}\m{d}\\
        = &\begin{pmatrix}
            c_1\\
            c_2\\
            c_3
        \end{pmatrix} +
        \begin{pmatrix}
            g_{11}  &g_{12}\\
            g_{21}  &g_{22}\\
            g_{31}  &g_{32}
        \end{pmatrix}
        \begin{bmatrix}
            D_1\\
            D_2
        \end{bmatrix}\\
        = &\begin{pmatrix}
            c_1 + g_{11}D_1 + g_{12}D_2\\
            c_2 + g_{21}D_1 + g_{22}D_2\\
            c_3 + g_{31}D_1 + g_{32}D_2
        \end{pmatrix}.
    \end{aligned}
\end{equation*}

\section{Mplus Execution}

The \texttt{DEFINE:} and \texttt{MODEL:} sections of the Mplus code is given as following:

\begin{singlespace}\tiny
    \begin{lstlisting}
DEFINE:

    ! G matrix
    X1D1 = X1 * D1;
    X2D1 = X2 * D1;
    X3D1 = X3 * D1;
    X1D2 = X1 * D2;
    X2D2 = X2 * D2;
    X3D2 = X3 * D2;
    ! H matrix
    M1D1 = M1 * D1;
    M2D1 = M2 * D1;
    M1D2 = M1 * D2;
    M2D2 = M2 * D2;

MODEL:

    [Y] (mu0);
    Y on M1 (b1);
    Y on M2 (b2);
    ! ---
    Y on M1D1 (h11);
    Y on M2D1 (h21);
    Y on M1D1 (h12);
    Y on M2D1 (h22);
    ! ---
    Y on X1 (c1);
    Y on X2 (c2);
    Y on X3 (c3);
    ! ---
    Y on D1 (f1);
    Y on D2 (f2);
    ! ---
    Y on X1D1 (g11);
    Y on X2D1 (g21);
    Y on X3D1 (g31);
    Y on X1D2 (g12);
    Y on X2D2 (g22);
    Y on X3D2 (g32);

    [M1] (mu1);
    M1 on X1 (a11);
    M1 on X2 (a21);
    M1 on X3 (a31);

    [M2] (mu2);
    M2 on X1 (a12);
    M2 on X2 (a22);
    M2 on X3 (a32);
    \end{lstlisting}
\end{singlespace}

\ltikz{fig:moderator}{Moderated Mediation Model}{
\begin{tikzpicture}[
    manvar/.style={rectangle,draw=black,minimum width=1.5cm},
    ->,>=stealth',semithick,
    bend angle=-45,
    decoration={
        zigzag,
        amplitude=1pt,
        segment length=1mm,
        post=lineto,
        post length=4pt
    }
]

% MODEL DIAGRAM

% Draw independent vars (X)
    \node[manvar] (0X1) at (1,9) {$X_1$};
    \node[manvar] (0X2) at (1,6) {$X_2$};
    \node[manvar] (0X3) at (1,3) {$X_3$};

% Draw mediators (M)
    \node[manvar] (0M1) at (4,7) {$M_1$};
    \node[manvar] (0M2) at (4,5) {$M_2$};

% Draw moderators (D)
    \node[manvar] (0D1) at (7,8) {$D_1$};
    \node[manvar] (0D2) at (7,4) {$D_2$};

% Draw dependent var (Y)
    \node[manvar] (0Y) at (10,6) {$Y$};

% Link X with M
    \draw[blue,->] (0X1.east) to node[above,sloped,pos=0.5] {$a_{11}$} (0M1.west);
    \draw[blue,->] (0X2.east) to node[] {} (0M1.west);
    \draw[blue,->] (0X3.east) to node[] {} (0M1.west);

    \draw[blue,->] (0X1.east) to node[] {} (0M2.west);
    \draw[blue,->] (0X2.east) to node[] {} (0M2.west);
    \draw[blue,->] (0X3.east) to node[below,sloped,pos=0.5] {$a_{32}$} (0M2.west);

% Link M with Y
    \draw[blue,->] (0M1.east) to node[above,sloped] {$b_1$} (0Y.west);
    \draw[blue,->] (0M2.east) to node[below,sloped] {$b_2$} (0Y.west);

% Link X with Y
    \draw[red,->] (0X1.east) to node[above,sloped,pos=0.3] {$c_1$} (0Y.west);
    \draw[red,->] (0X2.east) to node[pos=0.3] {$c_2$} (0Y.west);
    \draw[red,->] (0X3.east) to node[below,sloped,pos=0.3] {$c_3$} (0Y.west);

% Moderate on direct pathways
    \draw[forestgreen,->] (6.25,8)--(5.5,7.5);
    \draw[forestgreen,->] (6.25,8)--(5.5,6);
    \draw[forestgreen,->] (6.25,8)--(5.5,4.5);

    \draw[forestgreen,->] (6.25,4)--(5.5,7.5);
    \draw[forestgreen,->] (6.25,4)--(5.5,6);
    \draw[forestgreen,->] (6.25,4)--(5.5,4.5);

% Moderate on indirect pathways
    \draw[forestgreen,->] (7,7.7)--(8.5,6.17);
    \draw[forestgreen,->] (7,7.7)--(8.5,5.83);

    \draw[forestgreen,->] (7,4.3)--(8.5,6.17);
    \draw[forestgreen,->] (7,4.3)--(8.5,5.83);


% STATISTICAL DIAGRAM

% Draw independent vars (X)
    \node[manvar] (X1) at (0,0) {$X_1$};
    \node[manvar] (X2) at (0,-1) {$X_2$};
    \node[manvar] (X3) at (0,-2) {$X_3$};

% Draw mediators (M)
    \node[manvar] (M1) at (3,-4) {$M_1$};
    \node[manvar] (M2) at (3,-5) {$M_2$};

% Draw dependent var (Y)
    \node[manvar] (Y) at (7,-3) {$Y$};

% Draw moderators (D)
    \node[manvar] (D1) at (7,1) {$D_1$};
    \node[manvar] (D2) at (9,1) {$D_2$};

% Draw X * D interactions
    \node[manvar] (X1D1) at (11,0) {$X_1D_1$};
    \node[manvar] (X2D1) at (11,-1) {$X_2D_1$};
    \node[manvar] (X3D1) at (11,-2) {$X_3D_1$};

    \node[manvar] (X1D2) at (13,0) {$X_1D_2$};
    \node[manvar] (X2D2) at (13,-1) {$X_2D_2$};
    \node[manvar] (X3D2) at (13,-2) {$X_3D_2$};

% Draw M * D interactions
    \node[manvar] (M1D1) at (11,-4) {$M_1D_1$};
    \node[manvar] (M2D1) at (11,-5) {$M_2D_1$};

    \node[manvar] (M1D2) at (13,-4) {$M_1D_2$};
    \node[manvar] (M2D2) at (13,-5) {$M_2D_2$};

% Link X with M (a)
    \draw[blue,->] (X1.east) to node[above,sloped,pos=0.7] {$a_{11}$} (M1.west);
    \draw[blue,->] (X1.east) to node[above,sloped] {} (M2.west);

    \draw[blue,->] (X2.east) to node[above,sloped] {} (M1.west);
    \draw[blue,->] (X2.east) to node[above,sloped] {} (M2.west);

    \draw[blue,->] (X3.east) to node[above,sloped] {} (M1.west);
    \draw[blue,->] (X3.east) to node[below,sloped,pos=0.4] {$a_{32}$} (M2.west);

% Link M with Y (b)
    \draw[blue,->] (M1.east) to node[above,sloped,pos=0.4] {$b_1$} (Y.west);
    \draw[blue,->] (M2.east) to node[below,sloped,pos=0.4] {$b_2$} (Y.west);

% Link X with Y (c)
    \draw[red,->] (X1.east) to node[above,sloped] {$c_1$} (Y.west);
    \draw[red,->] (X2.east) to node[above,sloped] {} (Y.west);
    \draw[red,->] (X3.east) to node[below,sloped] {$c_3$} (Y.west);

% Link D with Y (f)
    \draw[forestgreen,->] (D1.south) to node[below,sloped,rotate=180,pos=0.3] {$f_1$} (Y.north);
    \draw[forestgreen,->] (D2.south) to node[above,sloped,pos=0.4] {$f_2$} (Y.north);

% Link XD with Y (G)
    \draw[forestgreen,->] (X1D1.west) to node[below,sloped] {$g_{11}$} (Y.east);
    \draw[forestgreen,->] (X3D2.west) to node[above,sloped,pos=0.6] {$g_{32}$} (Y.east);

% Link MD with Y (H)
    \draw[forestgreen,->] (M1D1.west) to node[above,sloped,pos=0.4] {$h_{11}$} (Y.south);
    \draw[forestgreen,->] (M2D2.west) to node[below,sloped,pos=0.65] {$h_{22}$} (Y.south);
\end{tikzpicture}
}{A moderated mediation is shown in both model diagram (upper panel) and statistical diagram (lower panel). \textcolor{red}{Direct paths}, \textcolor{blue}{indirect paths} and \textcolor{forestgreen}{moderations} are differentiated by colour.}
