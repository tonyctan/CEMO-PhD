\maketitle

% Insert Abstract
% Insert a line between title and abstract
\vspace{0.5\baselineskip}

% Indent both sides
\leftskip0.5in\relax
\rightskip0.5in\relax

% APA7 Rule 2.9 Abstract
% Abstracts in paragraph format are written as a single paragraph without indentation of the first line.

% APA7 Rule 2.24 Paragraph indentation
% Exception dot point 4: The first line of the abstract should be flush left.

\noindent Single paragraph. No indentation. Maximum 250 words. Phasellus egestas purus et sem porta venenatis. Vestibulum ante ipsum primis in faucibus orci luctus et ultrices posuere cubilia curae; Sed pulvinar leo ut posuere sollicitudin. Integer sem tortor, tristique id nisl sed, aliquam pharetra tortor. Integer tortor purus, facilisis mollis tortor nec, sodales bibendum urna. Nulla quam enim, feugiat semper neque ut, sagittis ullamcorper urna. Donec et iaculis mi. Etiam congue tellus ut felis viverra, ac mollis eros venenatis. Vivamus eget ante at eros pulvinar tempor nec at arcu. Donec et vestibulum nunc. Fusce sed metus nisi. In leo turpis, mattis a tincidunt luctus, pellentesque et velit. Ut tortor mi, rutrum nec fringilla vitae, maximus et ligula. Vestibulum finibus semper ornare.

% APA7 Rule 2.10 Keywords
% Write the label "Keywords:" (in italic) one line below the abstract, indented 0.5 in. like a regular paragraph, followed by the keywords in lowercase (but capitalize proper nouns), separated by commas. The key words can be listed in any order. Do not use a period or other punctuation after the last keyword. If the keywords run onto a second line, the second line is not indented.

\textit{Keywords:} keyword 1, keyword 2, keyword 3

% Leave an empty line behind
\vspace{0.75\baselineskip}

% Restore indentation
\leftskip0in\relax
\rightskip0in\relax
% Restore left alignment
\raggedright
% Restore paragraph indentation
\setlength\parindent{0.5in}

% Never start your introduction with a heading "Introduction".

Maecenas nec ultricies tellus. Suspendisse tristique a ante non fringilla. Morbi et sem dignissim, aliquet leo vel, scelerisque purus. Sed iaculis, est vel molestie consequat, erat elit venenatis arcu, a tempor libero diam sit amet velit. Sed lectus nunc, viverra a molestie nec, lacinia nec purus. Cras vitae efficitur metus. Nullam ac auctor dolor. Proin quis lacinia urna. Maecenas feugiat congue justo eget iaculis. Pellentesque vehicula nunc sit amet viverra feugiat. Aenean tristique luctus sapien sed feugiat. Donec ut vestibulum lectus, ut fermentum turpis. Etiam rutrum eros a sem commodo consequat \parencite{ludtke:2008}.

\subsection{First Level 2 Heading}

Suspendisse varius iaculis sem, vel congue augue vulputate in. Vestibulum ante ipsum primis in faucibus orci luctus et ultrices posuere cubilia curae; Nullam vitae purus eget purus eleifend gravida finibus eget erat. Quisque eget tincidunt dolor. In ac lectus at lectus cursus scelerisque non a libero. Donec posuere neque vitae sapien aliquet maximus. Phasellus malesuada, lectus eget molestie facilisis, mauris nunc congue risus, in hendrerit magna dolor sit amet eros. Phasellus nec elit nibh. Sed lacinia, augue ac pretium blandit, quam tortor aliquet massa, aliquet eleifend lectus turpis id sapien. Praesent eget leo id enim blandit elementum \parencite{marsh:2009}.

\subsubsection{One Level 3 Heading}

\citeauthor{marsh:2009} and colleagues \citeyear{marsh:2009} nulla finibus dignissim ex vitae mattis. Phasellus ligula sapien, auctor vel finibus non, fermentum a diam. Ut commodo leo porttitor libero porta, ac accumsan ante egestas. Vivamus in sem eu felis interdum vulputate. Vivamus ut orci libero. Curabitur nisi odio, pulvinar sed ornare vitae, pellentesque dapibus ligula. Fusce et faucibus dolor. Aliquam erat volutpat. Quisque sagittis justo vitae dui posuere, a aliquet nunc ornare. Praesent tempor a est ac mattis. Interdum et malesuada fames ac ante ipsum primis in faucibus. Proin elementum luctus placerat. Sed rhoncus leo in magna cursus egestas. Aliquam a arcu a urna sodales convallis. Duis a convallis quam, hendrerit commodo tortor.

\subsubsection{Another Level 3 Heading}

Nullam id commodo turpis. Pellentesque sed fermentum quam. Aenean ut massa id quam pretium blandit ac in ex. Morbi ultricies tellus magna. Quisque ornare, ligula at tincidunt finibus, justo mauris tristique lectus, at bibendum odio sapien sed ante. Curabitur ultrices pulvinar magna quis feugiat. Suspendisse vehicula ex a sem vehicula convallis. Aliquam ut lacus porttitor, bibendum ex eu, varius lorem. Vestibulum ultricies elementum neque, ac pharetra libero mollis in. Etiam eu mollis nulla, eget blandit elit.

\subsection{Second Level 2 Heading}

Aenean nec ultrices turpis, vitae vehicula felis. Praesent nec nisl ultrices, facilisis arcu ac, ultrices tellus. Sed eget urna dolor. Proin tortor nibh, consequat ac metus ut, molestie mattis nibh. Nullam sit amet felis vel risus sagittis eleifend ut vel enim. Morbi suscipit vestibulum neque sed porta. Morbi eu sodales elit, sed laoreet justo. Ut in tellus egestas, consequat purus et, mattis urna. Proin pretium, eros et consectetur dapibus, arcu mauris sagittis elit, sit amet volutpat urna leo sollicitudin quam. Aenean nec porttitor ante, sit amet bibendum est. Nulla sagittis mattis gravida. Etiam elementum, augue et volutpat placerat, nibh lectus pretium tortor, vitae posuere dui elit eget libero. This thesis therefore aims to answer the following research questions:
% APA7 Rule 6.50 Lettered Lists and 6.51 Numbered Lists
\begin{MAEitemize}
    \item[RQ1.] To what extent the variation in $Y$ can be accounted for by the variation in $X$.
    \item[RQ2.] How does $M$ influence the relationship between $X$ and $Y$ as a mediator?
\end{MAEitemize}
