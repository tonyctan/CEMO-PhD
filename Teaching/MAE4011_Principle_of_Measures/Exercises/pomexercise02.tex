\documentclass[compress]{beamer}\usepackage[]{graphicx}\usepackage[]{xcolor}
% maxwidth is the original width if it is less than linewidth
% otherwise use linewidth (to make sure the graphics do not exceed the margin)
\makeatletter
\def\maxwidth{ %
  \ifdim\Gin@nat@width>\linewidth
    \linewidth
  \else
    \Gin@nat@width
  \fi
}
\makeatother

\definecolor{fgcolor}{rgb}{0.345, 0.345, 0.345}
\newcommand{\hlnum}[1]{\textcolor[rgb]{0.686,0.059,0.569}{#1}}%
\newcommand{\hlstr}[1]{\textcolor[rgb]{0.192,0.494,0.8}{#1}}%
\newcommand{\hlcom}[1]{\textcolor[rgb]{0.678,0.584,0.686}{\textit{#1}}}%
\newcommand{\hlopt}[1]{\textcolor[rgb]{0,0,0}{#1}}%
\newcommand{\hlstd}[1]{\textcolor[rgb]{0.345,0.345,0.345}{#1}}%
\newcommand{\hlkwa}[1]{\textcolor[rgb]{0.161,0.373,0.58}{\textbf{#1}}}%
\newcommand{\hlkwb}[1]{\textcolor[rgb]{0.69,0.353,0.396}{#1}}%
\newcommand{\hlkwc}[1]{\textcolor[rgb]{0.333,0.667,0.333}{#1}}%
\newcommand{\hlkwd}[1]{\textcolor[rgb]{0.737,0.353,0.396}{\textbf{#1}}}%
\let\hlipl\hlkwb

\usepackage{framed}
\makeatletter
\newenvironment{kframe}{%
 \def\at@end@of@kframe{}%
 \ifinner\ifhmode%
  \def\at@end@of@kframe{\end{minipage}}%
  \begin{minipage}{\columnwidth}%
 \fi\fi%
 \def\FrameCommand##1{\hskip\@totalleftmargin \hskip-\fboxsep
 \colorbox{shadecolor}{##1}\hskip-\fboxsep
     % There is no \\@totalrightmargin, so:
     \hskip-\linewidth \hskip-\@totalleftmargin \hskip\columnwidth}%
 \MakeFramed {\advance\hsize-\width
   \@totalleftmargin\z@ \linewidth\hsize
   \@setminipage}}%
 {\par\unskip\endMakeFramed%
 \at@end@of@kframe}
\makeatother

\definecolor{shadecolor}{rgb}{.97, .97, .97}
\definecolor{messagecolor}{rgb}{0, 0, 0}
\definecolor{warningcolor}{rgb}{1, 0, 1}
\definecolor{errorcolor}{rgb}{1, 0, 0}
\newenvironment{knitrout}{}{} % an empty environment to be redefined in TeX

\usepackage{alltt}

% Load ttbeamer template depending on operating system
\usepackage{ifplatform}
\ifwindows
  \usepackage{M:/pc/Dokumenter/ttbeamer}
\fi
\iflinux
  \usepackage{/home/tony/uio/pc/Dokumenter/ttbeamer}
\fi
\ifmacosx
  \usepackage{/Users/tctan/uio/pc/Dokumenter/ttbeamer}
\fi

\title{Exercise 2}
\author[]{Tony Tan \\\vspace{6pt} {\em{University of Oslo}} }
\date{Wednesday, 02 November 2022}
\IfFileExists{upquote.sty}{\usepackage{upquote}}{}
\begin{document}



\begin{frame}[fragile]
\titlepage
\end{frame}


\section{Lecture 5}

\begin{frame}[fragile]
  \frametitle{L5 Task 1}
    Two alternate-form tests were given to the same population, yielding estimated standard deviation of $s_1 = 7.5$, $s_2 = 8$ and an estimated covariance of $s_{12} = 12$. Calculate the estimated standard error of measurement forf the test
\end{frame}


\begin{frame}[fragile]
  \frametitle{L5 Task 1: Solution}
    We first estimate the \ttemph{reliability coefficient}
    \begin{equation*}
      \begin{aligned}
        \text{Population: } \rho_{YY'} &= \frac{\cov{Y}{Y'}}{\sigma_Y \sigma_{Y'}} = \frac{\sigma_T^2}{\sqrt{\sigma_Y^2} \sqrt{\sigma_{Y'}^2}} = \frac{\sigma_T^2}{\sigma_Y^2} \\
        \text{Sample: } \hat{\rho}_{YY'} &= \frac{S_{12}}{s_1 \cdot s_2} = \frac{12}{7.5 \times 8} = 0.2.
      \end{aligned}
    \end{equation*}

    We then obtain the \ttemph{standard error of measurement} as
      \[ \hat{\text{SEM}} (Y) = \sqrt{s_Y^2 (1 - \hat{\rho}_{YY'})} = \sqrt{7.5 \times 8 \times (1 - 0.2)} \approx 6.93. \]
\end{frame}


\begin{frame}[fragile]
  \frametitle{L5 Task 2}
    \begin{itemize}
      \item We have autumn ($A$) and spring ($S$) test-takers of the SweSAT with $A \sim \N{60}{12}$ and $S \sim \N{70}{16}$ true score distributions respectively.
      \item The test takers are given the same test which is defined by the classical true score model with an error score $E \sim \N{0}{4}$.
    \end{itemize}

    Calculate the reliability of the test for population $A$ and $S$.
\end{frame}


\begin{frame}[fragile]
  \frametitle{L5 Task 2: Solution}
    Recall that the reliability coefficient is defined as
      \[ \rho_{YY'} = \frac{\sigma_T^2}{\sigma_Y^2} = \frac{\sigma_T^2}{\sigma_T^2 + \sigma_E^2}. \]

    For population $A$, we have
      \[ \hat{\rho}_{YY'}^A = \frac{12}{12 + 4} = 0.75. \]

    Similarly for population $S$
    \[ \hat{\rho}_{YY'}^S = \frac{16}{16 + 4} = 0.80. \]
\end{frame}


\section{Lecture 6}

\begin{frame}[fragile]
  \frametitle{L6 Task 1}
    The following estimated factor loadings and error variances were obtained from a single factor model for five mathematics items

    \begin{center}
      \begin{tabular}{cccccc}
        \hline
          &Item 1 &Item 2 &Item 3 &Item 4 &Item 5 \\
          $\hat{\lambda}_j$ &0.109 &0.202 &0.322 &0.068 &0.148 \\
          $\hat{\Psi}_j^2$ &0.196 &0.173 &0.140 &0.173 &0.147 \\
        \hline
      \end{tabular}
    \end{center}

    Calculate the estimated coefficient omega.
\end{frame}


\begin{frame}[fragile]
  \frametitle{L6 Task 1: Diagram}
    \begin{tikzpicture}[
      latvar/.style={ellipse,draw=black,minimum width=1cm,minimum height=1cm},
      manvar/.style={rectangle,draw=black,minimum width=1.5cm},
      mean/.style={fill=black!10!white,regular polygon,regular polygon sides=3},
      ->,>=stealth',semithick, % Make arrow heads better looking
      <->,>=latex' % Make arc's arrow heads rorate with the arc endings
    ]

      % Set 5 items
        \node[manvar] (i1) at (-4,3.5) {Item 1};
        \node[manvar] (i2) at (-2,3.5) {Item 2};
        \node[manvar] (i3) at (0,3.5) {Item 3};
        \node[manvar] (i4) at (2,3.5) {Item 4};
        \node[manvar] (i5) at (4,3.5) {Item 5};

      % Set latent variable
        \node[latvar] (g) at (0,0) {$g$};

      % Set 5 residuals
        \node[latvar] (e1) at (-4,5) {$\epsilon_1$};
        \node[latvar] (e2) at (-2,5) {$\epsilon_2$};
        \node[latvar] (e3) at (0,5) {$\epsilon_3$};
        \node[latvar] (e4) at (2,5) {$\epsilon_4$};
        \node[latvar] (e5) at (4,5) {$\epsilon_5$};

      % Link items to g
        \draw[<-] (i1) to node[below,sloped,pos=0.4] {$\hat{\lambda}_1=0.109$} (g);
        \draw[<-] (i2) to node[below,sloped,pos=0.42] {$\hat{\lambda}_2=0.202$} (g);
        \draw[<-] (i3) to node[below,sloped] {$\hat{\lambda}_3=0.322$} (g);
        \draw[<-] (i4) to node[above,sloped] {$\hat{\lambda}_4=0.068$} (g);
        \draw[<-] (i5) to node[above,sloped] {$\hat{\lambda}_5=0.148$} (g);

      % Link items to g
        \draw[->] (e1) to node[right,pos=0.4] {$1$} (i1);
        \draw[->] (e2) to node[right,pos=0.4] {$1$} (i2);
        \draw[->] (e3) to node[right,pos=0.4] {$1$} (i3);
        \draw[->] (e4) to node[right,pos=0.4] {$1$} (i4);
        \draw[->] (e5) to node[right,pos=0.4] {$1$} (i5);

      % Draw error variance arcs
        \draw[<->] (-4,5.5) arc (0:257:0.4);
        \draw[<->] (-2,5.5) arc (0:257:0.4);
        \draw[<->] (0,5.5) arc (0:257:0.4);
        \draw[<->] (2,5.5) arc (0:257:0.4);
        \draw[<->] (4,5.5) arc (0:257:0.4);

      % Draw error variance labels
        \node[below,rotate=30] at (-5,6.3) {$\hat{\Psi}_1^2=0.196$};
        \node[below,rotate=30] at (-3,6.3) {$\hat{\Psi}_2^2=0.173$};
        \node[below,rotate=30] at (-1,6.3) {$\hat{\Psi}_3^2=0.140$};
        \node[below,rotate=30] at (1,6.3) {$\hat{\Psi}_4^2=0.173$};
        \node[below,rotate=30] at (3,6.3) {$\hat{\Psi}_5^2=0.147$};
    \end{tikzpicture}
\end{frame}


\begin{frame}[fragile]
  \frametitle{L6 Task 1: Solution}
    Coefficient omega is equal to the reliability of the sum score when a \ttemph{single factor models} holds. We obtain the following estimate of the reliability with the estimated factor model parameters:
      \[ \hat{\omega} = \frac{ \left( \sum_{j=1}^5 \hat{\lambda}_j \right)^2 }{\left( \sum_{j=1}^5 \hat{\lambda}_j \right)^2 + \sum_{j=1}^5 \hat{\Psi}_j^2} \approx \frac{0.849^2}{0.849^2 + 0.829} \approx 0.465. \]

    If  the factor model is the \ttemph{true model\color{black},} $\hat{\omega}$ is an unbiased estimator of the reliability of the sum score.
\end{frame}


\begin{frame}[fragile]
  \frametitle{L6 Task 2}
    A \ttemph{restricted model} with the factor loadings set equal was also estimated. The estimated factor loading was $\hat{\lambda} = 0.164$ and the error variances are given in the table below:

    \begin{center}
      \begin{tabular}{cccccc}
        \hline
          &Item 1 &Item 2 &Item 3 &Item 4 &Item 5 \\
          $\hat{\Psi}_j^2$ &0.181 &0.188 &0.217 &0.151 &0.142 \\
        \hline
      \end{tabular}
    \end{center}

    Calculate the estimated coefficient alpha/omega.
\end{frame}


\begin{frame}[fragile]
  \frametitle{L6 Task 2: Diagram}
    \begin{tikzpicture}[
      latvar/.style={ellipse,draw=black,minimum width=1cm,minimum height=1cm},
      manvar/.style={rectangle,draw=black,minimum width=1.5cm},
      mean/.style={fill=black!10!white,regular polygon,regular polygon sides=3},
      ->,>=stealth',semithick, % Make arrow heads better looking
      <->,>=latex' % Make arc's arrow heads rorate with the arc endings
    ]

      % Set 5 items
        \node[manvar] (i1) at (-4,3.5) {Item 1};
        \node[manvar] (i2) at (-2,3.5) {Item 2};
        \node[manvar] (i3) at (0,3.5) {Item 3};
        \node[manvar] (i4) at (2,3.5) {Item 4};
        \node[manvar] (i5) at (4,3.5) {Item 5};

      % Set latent variable
        \node[latvar] (g) at (0,0) {$g$};

      % Set 5 residuals
        \node[latvar] (e1) at (-4,5) {$\epsilon_1$};
        \node[latvar] (e2) at (-2,5) {$\epsilon_2$};
        \node[latvar] (e3) at (0,5) {$\epsilon_3$};
        \node[latvar] (e4) at (2,5) {$\epsilon_4$};
        \node[latvar] (e5) at (4,5) {$\epsilon_5$};

      % Link items to g
        \draw[<-] (i1) to node[below,sloped,pos=0.4] {$\hat{\lambda}=0.164$} (g);
        \draw[<-] (i2) to node[below,sloped,pos=0.42] {$\hat{\lambda}=0.164$} (g);
        \draw[<-] (i3) to node[below,sloped] {$\hat{\lambda}=0.164$} (g);
        \draw[<-] (i4) to node[above,sloped] {$\hat{\lambda}=0.164$} (g);
        \draw[<-] (i5) to node[above,sloped] {$\hat{\lambda}=0.164$} (g);

      % Link items to g
        \draw[->] (e1) to node[right,pos=0.4] {$1$} (i1);
        \draw[->] (e2) to node[right,pos=0.4] {$1$} (i2);
        \draw[->] (e3) to node[right,pos=0.4] {$1$} (i3);
        \draw[->] (e4) to node[right,pos=0.4] {$1$} (i4);
        \draw[->] (e5) to node[right,pos=0.4] {$1$} (i5);

      % Draw error variance arcs
        \draw[<->] (-4,5.5) arc (0:257:0.4);
        \draw[<->] (-2,5.5) arc (0:257:0.4);
        \draw[<->] (0,5.5) arc (0:257:0.4);
        \draw[<->] (2,5.5) arc (0:257:0.4);
        \draw[<->] (4,5.5) arc (0:257:0.4);

      % Draw error variance labels
        \node[below,rotate=30] at (-5,6.3) {$\hat{\Psi}_1^2=0.181$};
        \node[below,rotate=30] at (-3,6.3) {$\hat{\Psi}_2^2=0.188$};
        \node[below,rotate=30] at (-1,6.3) {$\hat{\Psi}_3^2=0.217$};
        \node[below,rotate=30] at (1,6.3) {$\hat{\Psi}_4^2=0.151$};
        \node[below,rotate=30] at (3,6.3) {$\hat{\Psi}_5^2=0.142$};
    \end{tikzpicture}
\end{frame}


\begin{frame}[fragile]
  \frametitle{L6 Task 2: Solution}
      \[ \hat{\alpha} = \hat{\omega} = \frac{ 5^2 \times \hat{\lambda}^2 }{ 5^2 \times \hat{\lambda}^2 + \sum_{j=1}^5 \hat{\Psi}_j^2 } = \frac{ 5^2 \times 0.164^2 }{ 5^2 \times 0.164^2 + 0.879 } \approx 0.433 \]

    Comparing Task 2 with Task 1: Cronbach's alpha is only valid under \ttemph{tau-equivalence} and it \ttemph{underestimates} the reliability of the scale.
\end{frame}

\end{document}
