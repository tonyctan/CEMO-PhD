\documentclass[compress]{beamer}\usepackage[]{graphicx}\usepackage[]{xcolor}
% maxwidth is the original width if it is less than linewidth
% otherwise use linewidth (to make sure the graphics do not exceed the margin)
\makeatletter
\def\maxwidth{ %
  \ifdim\Gin@nat@width>\linewidth
    \linewidth
  \else
    \Gin@nat@width
  \fi
}
\makeatother

\definecolor{fgcolor}{rgb}{0.345, 0.345, 0.345}
\newcommand{\hlnum}[1]{\textcolor[rgb]{0.686,0.059,0.569}{#1}}%
\newcommand{\hlstr}[1]{\textcolor[rgb]{0.192,0.494,0.8}{#1}}%
\newcommand{\hlcom}[1]{\textcolor[rgb]{0.678,0.584,0.686}{\textit{#1}}}%
\newcommand{\hlopt}[1]{\textcolor[rgb]{0,0,0}{#1}}%
\newcommand{\hlstd}[1]{\textcolor[rgb]{0.345,0.345,0.345}{#1}}%
\newcommand{\hlkwa}[1]{\textcolor[rgb]{0.161,0.373,0.58}{\textbf{#1}}}%
\newcommand{\hlkwb}[1]{\textcolor[rgb]{0.69,0.353,0.396}{#1}}%
\newcommand{\hlkwc}[1]{\textcolor[rgb]{0.333,0.667,0.333}{#1}}%
\newcommand{\hlkwd}[1]{\textcolor[rgb]{0.737,0.353,0.396}{\textbf{#1}}}%
\let\hlipl\hlkwb

\usepackage{framed}
\makeatletter
\newenvironment{kframe}{%
 \def\at@end@of@kframe{}%
 \ifinner\ifhmode%
  \def\at@end@of@kframe{\end{minipage}}%
  \begin{minipage}{\columnwidth}%
 \fi\fi%
 \def\FrameCommand##1{\hskip\@totalleftmargin \hskip-\fboxsep
 \colorbox{shadecolor}{##1}\hskip-\fboxsep
     % There is no \\@totalrightmargin, so:
     \hskip-\linewidth \hskip-\@totalleftmargin \hskip\columnwidth}%
 \MakeFramed {\advance\hsize-\width
   \@totalleftmargin\z@ \linewidth\hsize
   \@setminipage}}%
 {\par\unskip\endMakeFramed%
 \at@end@of@kframe}
\makeatother

\definecolor{shadecolor}{rgb}{.97, .97, .97}
\definecolor{messagecolor}{rgb}{0, 0, 0}
\definecolor{warningcolor}{rgb}{1, 0, 1}
\definecolor{errorcolor}{rgb}{1, 0, 0}
\newenvironment{knitrout}{}{} % an empty environment to be redefined in TeX

\usepackage{alltt}

% Load ttbeamer template depending on operating system
\usepackage{ifplatform}
\ifwindows
  \usepackage{M:/pc/Dokumenter/ttbeamer}
\fi
\iflinux
  \usepackage{/home/tony/uio/pc/Dokumenter/ttbeamer}
\fi
\ifmacosx
  \usepackage{/Users/tctan/uio/pc/Dokumenter/ttbeamer}
\fi

\title{Lab 1 - Correlations, reliability and consistency}
\author[]{Tony Tan \\\vspace{6pt} {\em{University of Oslo}} }
\date{Friday, 28 October 2021}
\IfFileExists{upquote.sty}{\usepackage{upquote}}{}
\begin{document}



\begin{frame}[fragile]
\titlepage
\end{frame}


\section{Learning intention}

\begin{frame}[fragile]
  \frametitle{Today}
  Estimate and interpret
    \begin{itemize}
      \item Correlations between variables
      \item Test-retest reliability
      \item Alternate/parallel test-forms
      \item Internal consistency
        \begin{itemize}
          \item Cronbach's alpha
          \item Single factor model
          \item McDonald's omega
        \end{itemize}
    \end{itemize}
\end{frame}


\section{Correlation}

\begin{frame}[fragile]
  \frametitle{Correlations between variables}

      \[ \rho_{XY} = \frac{\cov{X}{Y}}{\sigma_X \, \sigma_Y} \]

    \begin{itemize}
      \item Correlation is a measure of the strength of the \ttemph{linear relationship} between two variables.
      \item The correlation coefficient is a number between $-1$ and $1$, where $0$ indicates no linear relationship, and $1$ or $-1$ indicates perfect a linear relationship.
      \item Correlations are superior to \\ttemph{covariances} for interpreting interrelationship between two variables thanks to the standardisation procedure.
    \end{itemize}
\end{frame}


\end{document}
