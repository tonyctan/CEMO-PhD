\documentclass[compress]{beamer}\usepackage[]{graphicx}\usepackage[]{xcolor}
% maxwidth is the original width if it is less than linewidth
% otherwise use linewidth (to make sure the graphics do not exceed the margin)
\makeatletter
\def\maxwidth{ %
  \ifdim\Gin@nat@width>\linewidth
    \linewidth
  \else
    \Gin@nat@width
  \fi
}
\makeatother

\definecolor{fgcolor}{rgb}{0.345, 0.345, 0.345}
\newcommand{\hlnum}[1]{\textcolor[rgb]{0.686,0.059,0.569}{#1}}%
\newcommand{\hlstr}[1]{\textcolor[rgb]{0.192,0.494,0.8}{#1}}%
\newcommand{\hlcom}[1]{\textcolor[rgb]{0.678,0.584,0.686}{\textit{#1}}}%
\newcommand{\hlopt}[1]{\textcolor[rgb]{0,0,0}{#1}}%
\newcommand{\hlstd}[1]{\textcolor[rgb]{0.345,0.345,0.345}{#1}}%
\newcommand{\hlkwa}[1]{\textcolor[rgb]{0.161,0.373,0.58}{\textbf{#1}}}%
\newcommand{\hlkwb}[1]{\textcolor[rgb]{0.69,0.353,0.396}{#1}}%
\newcommand{\hlkwc}[1]{\textcolor[rgb]{0.333,0.667,0.333}{#1}}%
\newcommand{\hlkwd}[1]{\textcolor[rgb]{0.737,0.353,0.396}{\textbf{#1}}}%
\let\hlipl\hlkwb

\usepackage{framed}
\makeatletter
\newenvironment{kframe}{%
 \def\at@end@of@kframe{}%
 \ifinner\ifhmode%
  \def\at@end@of@kframe{\end{minipage}}%
  \begin{minipage}{\columnwidth}%
 \fi\fi%
 \def\FrameCommand##1{\hskip\@totalleftmargin \hskip-\fboxsep
 \colorbox{shadecolor}{##1}\hskip-\fboxsep
     % There is no \\@totalrightmargin, so:
     \hskip-\linewidth \hskip-\@totalleftmargin \hskip\columnwidth}%
 \MakeFramed {\advance\hsize-\width
   \@totalleftmargin\z@ \linewidth\hsize
   \@setminipage}}%
 {\par\unskip\endMakeFramed%
 \at@end@of@kframe}
\makeatother

\definecolor{shadecolor}{rgb}{.97, .97, .97}
\definecolor{messagecolor}{rgb}{0, 0, 0}
\definecolor{warningcolor}{rgb}{1, 0, 1}
\definecolor{errorcolor}{rgb}{1, 0, 0}
\newenvironment{knitrout}{}{} % an empty environment to be redefined in TeX

\usepackage{alltt}

% Load ttbeamer template depending on operating system
\usepackage{ifplatform}
\ifwindows
	\usepackage{M:/pc/Dokumenter/ttbeamer}
\fi
\iflinux
	\usepackage{/home/tony/uio/pc/Dokumenter/ttbeamer}
\fi
\ifmacosx
	\usepackage{/Users/tctan/uio/pc/Dokumenter/ttbeamer}
\fi

\title{Lab 1 - Correlations, reliability and consistency}
\author[]{Tony Tan \& Jarl Kristensen \\\vspace{6pt} {\em{University of Oslo}} }
\date{Monday, 7 November 2022}
\IfFileExists{upquote.sty}{\usepackage{upquote}}{}
\begin{document}



\begin{frame}[fragile]
\titlepage
\end{frame}


\begin{frame}[fragile]
	\frametitle{Today}
	Estimate and interpret:
		\begin{itemize}
			\item Correlations between variables
			\item Test-retest reliability
			\item Alternate/parallel test-forms
			\item Internal consistency
				\begin{itemize}
					\item Cronbach's alpha
					\item Single factor model
					\item McDonald's omega
				\end{itemize}
		\end{itemize}
\end{frame}


\begin{frame}[fragile]
	\frametitle{Correlation}

			\[ \rho_{XY} = \frac{\cov{X}{Y}}{\sigma_X \, \sigma_Y} \]

		\begin{itemize}
			\item Correlation is a measure of the strength of the \ttemph{linear relationship} between two variables.
			\item The correlation coefficient is a number between $-1$ and $1$, where $0$ indicates no linear relationship, and $1$ or $-1$ indicates perfect a linear relationship.
			\item Correlations are superior to \ttemph{covariances} for interpreting interrelationship between two variables thanks to the standardisation procedure.
		\end{itemize}
\end{frame}


\begin{frame}[fragile]
	\frametitle{Task 1: Correlation}
		Estimate the correlation between a pair of variables in \pk{likert\_data.rds} and comment on the relationship between the variables you chose.
\end{frame}


\begin{frame}[fragile]
	\frametitle{Test reliability}
		\begin{itemize}
			\item Test-retest reliability
			\item Alternate test forms
		\end{itemize}

		Use the correlation of the \ttemph{sum scores} of the two test forms to estimate the reliability of the test.
\end{frame}


\begin{frame}[fragile]
	\frametitle{Cronbach's $\alpha$}
		\[ \alpha = \frac{m}{m-1} \left[ 1 - \frac{\sum_{j=1}^m \V{X_j}}{\sigma_Y^2} \right] \]
\end{frame}


\begin{frame}[fragile]
	\frametitle{Task 2: Test reliability}
		\begin{itemize}
			\item Estimate coefficient alpha for one of the scales in the \pk{dich\_data.rds} dataset.
			\item How would you describe the estimated value?
		\end{itemize}
\end{frame}


\begin{frame}[fragile]
	\frametitle{Cronbach's $\alpha$: Pitfalls}
		\begin{itemize}
			\item Dependent on the number of items
			\item Assumes a \ttemph{single factor model} with \ttemph{equal loadings} for all items
			\item A \ttemph{lower bound} estimate of reliability as long as assumptions are met
		\end{itemize}
\end{frame}


\begin{frame}[fragile]
	\frametitle{Single factor model}
		\begin{itemize}
			\item \pk{lavaan} and other \cR packages
			\item This lab focuses on \pk{lavaan}
			\item Lecture 6: single factor model
		\end{itemize}
\end{frame}


\begin{frame}[fragile]
	\frametitle{\pk{lavaan}: Syntax}
		\begin{center}
			\begin{tabular}{cll}
				\hline
				Operator  & \multicolumn{1}{c}{Reading} & \multicolumn{1}{c}{Meaning}  \\
				\hline
				$=\sim$ & is measured by & define a latent variable \\
				$\sim$ & is regressed on & define a regression model \\
				$\sim\sim$ & is correlated with & specify covariances \\
				\hline
			\end{tabular}
		\end{center}
\end{frame}

\begin{frame}[fragile]
	\frametitle{\pk{lavaan}: Example code}
\begin{knitrout}
\definecolor{shadecolor}{rgb}{0.969, 0.969, 0.969}\color{fgcolor}\begin{kframe}
\begin{alltt}
        \hlcom{# Load lavaan}
        \hlkwd{library}\hlstd{(lavaan)}

        \hlcom{# Define latent variable}
        \hlstd{lat_var} \hlkwb{=} \hlstr{" y =~ x1 + x2 + x3 "}

        \hlcom{# Run a confirmatory factor analysis}
        \hlstd{cfa} \hlkwb{<-} \hlkwd{cfa}\hlstd{(lat_var,} \hlkwc{data} \hlstd{= mydata)}

        \hlcom{# Model evaluation}
        \hlkwd{summary}\hlstd{(cfa,} \hlkwc{fit.measures} \hlstd{=} \hlnum{TRUE}\hlstd{)}

        \hlcom{# Extract model coefficients}
        \hlkwd{coef}\hlstd{(cfa)}
\end{alltt}
\end{kframe}
\end{knitrout}
\end{frame}


\begin{frame}[fragile]
	\frametitle{Task 3: Single factor model}
		\begin{itemize}
			\item Estimate the single factor model using the scale you chose in Task 2
			\item Evaluate the model fit.
			\item Evaluate if the $\alpha$ you calculated in Task 2 violates the assumptions $\alpha$ relies on.
		\end{itemize}
\end{frame}

\begin{frame}[fragile]
	\frametitle{Coefficient omega}
		\[ \omega = \frac{\sigma_C^2}{\sigma_C^2 + \sigma_U^2} = \frac{\sigma_Y^2}{\sigma_{T_Y}^2 + \sigma_{E_Y}^2} = \frac{\left( \sum_{j=1}^m \lambda_j \right)^2 }{ \left( \sum_{j=1}^m \lambda_j \right)^2 + \sum_{j=1}^m \Psi_i^2 } \]

		Coefficient omega is the ratio of the true score variance (common) to total score variance (common + unique).
	\end{frame}


\begin{frame}[fragile]
	\frametitle{Task 4: Standardised model}
		\begin{itemize}
			\item Use the coefficients from Task 3 to calculate $\omega$.
			\item Evaluate the reliability of the scale.
			\item Compare $\omega$ (Task 4) and $\alpha$ (Task 3).
		\end{itemize}
\end{frame}

\end{document}
