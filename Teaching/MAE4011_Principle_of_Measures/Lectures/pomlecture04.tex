\documentclass[compress]{beamer}\usepackage[]{graphicx}\usepackage[]{xcolor}
% maxwidth is the original width if it is less than linewidth
% otherwise use linewidth (to make sure the graphics do not exceed the margin)
\makeatletter
\def\maxwidth{ %
  \ifdim\Gin@nat@width>\linewidth
    \linewidth
  \else
    \Gin@nat@width
  \fi
}
\makeatother

\definecolor{fgcolor}{rgb}{0.345, 0.345, 0.345}
\newcommand{\hlnum}[1]{\textcolor[rgb]{0.686,0.059,0.569}{#1}}%
\newcommand{\hlstr}[1]{\textcolor[rgb]{0.192,0.494,0.8}{#1}}%
\newcommand{\hlcom}[1]{\textcolor[rgb]{0.678,0.584,0.686}{\textit{#1}}}%
\newcommand{\hlopt}[1]{\textcolor[rgb]{0,0,0}{#1}}%
\newcommand{\hlstd}[1]{\textcolor[rgb]{0.345,0.345,0.345}{#1}}%
\newcommand{\hlkwa}[1]{\textcolor[rgb]{0.161,0.373,0.58}{\textbf{#1}}}%
\newcommand{\hlkwb}[1]{\textcolor[rgb]{0.69,0.353,0.396}{#1}}%
\newcommand{\hlkwc}[1]{\textcolor[rgb]{0.333,0.667,0.333}{#1}}%
\newcommand{\hlkwd}[1]{\textcolor[rgb]{0.737,0.353,0.396}{\textbf{#1}}}%
\let\hlipl\hlkwb

\usepackage{framed}
\makeatletter
\newenvironment{kframe}{%
 \def\at@end@of@kframe{}%
 \ifinner\ifhmode%
  \def\at@end@of@kframe{\end{minipage}}%
  \begin{minipage}{\columnwidth}%
 \fi\fi%
 \def\FrameCommand##1{\hskip\@totalleftmargin \hskip-\fboxsep
 \colorbox{shadecolor}{##1}\hskip-\fboxsep
     % There is no \\@totalrightmargin, so:
     \hskip-\linewidth \hskip-\@totalleftmargin \hskip\columnwidth}%
 \MakeFramed {\advance\hsize-\width
   \@totalleftmargin\z@ \linewidth\hsize
   \@setminipage}}%
 {\par\unskip\endMakeFramed%
 \at@end@of@kframe}
\makeatother

\definecolor{shadecolor}{rgb}{.97, .97, .97}
\definecolor{messagecolor}{rgb}{0, 0, 0}
\definecolor{warningcolor}{rgb}{1, 0, 1}
\definecolor{errorcolor}{rgb}{1, 0, 0}
\newenvironment{knitrout}{}{} % an empty environment to be redefined in TeX

\usepackage{alltt}

% Load ttbeamer template depending on operating system
\usepackage{ifplatform}
\ifwindows
  \usepackage{M:/pc/Dokumenter/ttbeamer}
\fi
\iflinux
  \usepackage{/home/tony/uio/pc/Dokumenter/ttbeamer}
\fi
\ifmacosx
  \usepackage{/Users/tctan/uio/pc/Dokumenter/ttbeamer}
\fi

\title{Lecture 4 - Levels of Measurement, Types of Measurement and Scale Scores}
\author[]{Tony Tan \\\vspace{6pt} {\em{University of Oslo}} }
\date{Thursday, 27 October 2022}
\IfFileExists{upquote.sty}{\usepackage{upquote}}{}
\begin{document}



\begin{frame}[fragile]
\titlepage
\end{frame}


\section{Introduction}

\begin{frame}[fragile]
  \frametitle{Last time}
    \begin{itemize}
      \item Review of some essential probability theory required for the course
      \item Concepts of expected value, variance, covariance
      \item Statistical reasoning and the concepts of parameters, estimators and estimates
    \end{itemize}
\end{frame}


\begin{frame}[fragile]
  \frametitle{Today}
    \begin{itemize}
      \item Levels of measurement -- nominal, ordinal, interval, ratio
      \item Mean, median, mode
      \item Criterion-referenced and norm-referenced tests
      \item Different ways to define test scores
      \item Some exercises
    \end{itemize}
\end{frame}


\section{Levels of measurement}

\begin{frame}[fragile]
  \frametitle{Measurement}
    \begin{itemize}
      \item The purpose of measurement is to \ttemph{quantify} an attribute
      \item We assign a number to an item response
      \item Based on item responses we assign a test score
      \item The test score is determined by the item responses but the reverse is typically not true -- the same test score can be obtained from different item responses
    \end{itemize}
\end{frame}


\begin{frame}[fragile]
  \frametitle{Ceiling and floor of a test}
    \begin{itemize}
      \item Any test has a floor and ceiling in that there is a range of levels of the attribute that the test can actually measure
      \item With a mathematics test containing 10 binary items, some test-takers may score 0 and some may score 10
      \item The underlying construct can be too low or too high for the test to be able to measure it
      \item Example: giving a university calculus exam to a class of second-graders
      \item Example: giving a diagnostic test for dementia to a group of university students
    \end{itemize}
\end{frame}
%//ref Crocker, L. & Algina, J. (1986). Introduction to classical and modern test theory. Wiley.


\begin{frame}[fragile]
  \frametitle{Measurement scales -- levels of measurement}
    \begin{itemize}
      \item Certain physical measurements have specific properties
      \item We can think of a measurement of length as being twice as large as another measurement of length
      \item We can think of the interval between 5\C and 6\C as being the same as that between 25\C and 26\C
      \item To what extent to item scores and test scores fulfill these properties?
    \end{itemize}
\end{frame}


\begin{frame}[fragile]
  \frametitle{Ratio scale}
    \begin{itemize}
      \item Some measurements have the property that the doubling of the measurement can be be interpreted as being twice as large
      \item For example, 20\,cm is twice as long as 10\,cm
      \item We of course also have that the difference between 30\,cm and 20\,cm is the same as the difference between 20\,cm and 10\,cm
    \end{itemize}
\end{frame}


\begin{frame}[fragile]
  \frametitle{Interval scale}
    \begin{itemize}
      \item The difference between two observations is interpreted in the same way
      \item Consider temperature as measured by celsius degrees (\textcelsius)
        \begin{itemize}
          \item The difference between 3\C and 2\C and the difference between 1003\C and 1002\C has the same interpretation
          \item We can of course also say that 3\C is higher than 2\C
          \item However, we can't say that 10\C is twice as warm as 5\C
        \end{itemize}
    \end{itemize}
\end{frame}


\begin{frame}[fragile]
  \frametitle{Ordinal scale}
    \begin{itemize}
      \item Some measurements have an ordering, but do not have the property of equal intervals
      \item Mohs scale for hardness of minerals
        \begin{itemize}
          \item The hardness is determined by which mineral scratches another mineral
          \item A mineral $A$ is harder than a mineral $B$ if $A$ scratches $B$
          \item However, if $A$ scratches $B$ and $B$ scratches $C$, we wouldn't be able to say that THE difference in hardness between $A$ and $B$ is the same as the difference in hardness between $B$ and $C$
          \item We could of course also not say that $A$ is twice as hard as $B$ or as $C$
        \end{itemize}
    \end{itemize}
\end{frame}


\begin{frame}[fragile]
  \frametitle{Nominal scale}
    \begin{itemize}
      \item If there is no ordering of the measurements, we have a nominal scale
      \item It won't make sense to speak of a measurement being twice as large, having equal distances or having any order
    \end{itemize}
\end{frame}


\begin{frame}[fragile]
  \frametitle{The scale of item scores?}
    \begin{itemize}
      \item Consider a \ttemph{Likert scale}
      \begin{itemize}
        \item Strongly agree, Agree, Neither agree nor disagree, Disagree, Strongly disagree
        \item If we assign integer scores from 1 to 5 to the categories, we are imposing a scale on the items
          \begin{itemize}
            \item Is Agree twice as large as Disagree?
            \item Is the difference between Strongly agree and Agree the same as that between Agree and Neither agree nor disagree?
          \end{itemize}
        \item It seems that the scale is actually ordinal
      \end{itemize}
    \end{itemize}
\end{frame}


\begin{frame}[fragile]
  \frametitle{The scale of test scores?}
    \begin{itemize}
      \item Consider a test score defined by the number of items correct on a 20 item test with binary item scores
      \item Does this test score have the property of equal intervals?
      \item Is a score of 10 twice as good as a score of 5?
    \end{itemize}
\end{frame}


\begin{frame}[fragile]
  \frametitle{Different test scores}
    \begin{itemize}
      \item Consider again a 20-item test with binary item scores
      \item We are not forced to define a test score which takes integer values from 0 to 20
      \item We can apply a transformation, such as the test score to the power of 2 or another transformation
      \item The choice of \ttemph{metric} is the choice of how numbers are assigned to observations
    \end{itemize}
\end{frame}


\section{Measures of central tendency}

\begin{frame}[fragile]
  \frametitle{The expected value (revision)}
    Let $X$ be a discrete R.V. that can take $k$ different values with probabilities $p_1, \dots, p_k$. Then
    \[ \E{X}=\sum_{i=1}^k x_i\, p_i. \]
\end{frame}


\begin{frame}[fragile]
  \frametitle{The median}
    \begin{itemize}
      \item The median denotes the value $c$ such that
        \[ \p{X \leq c} \geq 0.5 \text{ and } \p{X \geq c} \geq 0.5. \]
      \item We can think of the median as the value of the R.V. $X$ which lies in the middle of the \ttemph{probability mass function} or the \ttemph{probability density function}.
      \item Such a measure will give a better idea of the typical value of $X$ when the density of $X$ is \ttemph{skewed}.
    \end{itemize}
\end{frame}


\begin{frame}[fragile]
  \frametitle{The mode}
    \begin{itemize}
      \item For a discrete R.V. $X$, the mode denotes the value of $X$ which has the \ttemph{highest probability mass} associated with it.
      \item For a continuous R.V., the mode denotes the value of $X$ for which the density $f(x)$ reaches its highest value.
      \item Note that the mode may not be unique.
    \end{itemize}
\end{frame}


\begin{frame}[fragile]
  \frametitle{Example}
    Let $X$ be a discrete R.V. defined by

    \begin{center}
      \begin{tabular}{ccccc}
        \hline
        $X$ &1  &2  &3  &4 \\
        $\p{X}$ &0.1  &0.5  &0.2  &0.2\\
        \hline
      \end{tabular}
    \end{center}

    What is a) the expected value, b) the median and c) the mode of $X$?
    \begin{itemize}
      \item[a] $\E{X} = 1 \times 0.1 + 2 \times 0.5 + 3 \times 0.2 + 4\times 0.2 = 2.5$.
      \item[b] Since $\p{X \leq 2} = 0.6$ and $\p{X \geq 2} = 0.9$, the median of $X$ is $2$.
      \item[c] Since $\p{X = 2}$ reaches the higest value 0.5, the mode of $X$ is $2$.
    \end{itemize}
\end{frame}


\begin{frame}[fragile]
  \frametitle{Symmetric distributions}
    \begin{itemize}
      \item Let $X$ be a R.V. taking values $1$, $2$ and $3$ such that
        \[ \p{X=1} = 0.2,\ \p{X=2} = 0.6, \text{ and } \p{X=3} = 0.2. \]
      This is a \ttemph{symmetric} distribution -- whose mean and median coincide.
      \item Consider a $\N{0}{1}$ distribution. This is also a symmetric distribution, whose mean, median, and mode are all equal.
    \end{itemize}
\end{frame}


\begin{frame}[fragile]
  \frametitle{The sample mean}
    Let $\{x_i\}_{i=1}^n$ denote a sample. The sample mean is
      \[ \bar{x} = \frac{1}{n} \sum_{i=1}^n x_i \]

    In \CR, we can calculate the sample mean of a vector $\m{x}$ by typing
\begin{knitrout}
\definecolor{shadecolor}{rgb}{0.969, 0.969, 0.969}\color{fgcolor}\begin{kframe}
\begin{alltt}
  \hlstd{x} \hlkwb{<-} \hlkwd{c}\hlstd{(}\hlnum{12}\hlstd{,} \hlnum{8}\hlstd{,} \hlnum{19}\hlstd{,} \hlnum{13}\hlstd{,} \hlnum{8}\hlstd{)}
  \hlkwd{mean}\hlstd{(x)}
\end{alltt}
\begin{verbatim}
## [1] 12
\end{verbatim}
\end{kframe}
\end{knitrout}
\end{frame}

\begin{frame}[fragile]
  \frametitle{The sample median}
    Let $\{x_i\}_{i=1}^n$ denote a sample. The sample median is the middle value of the observations.

    The median may be a more suitable measure of central tendency when the distribution is skewed.

    In \CR, we can find the sample median by typing
\begin{knitrout}
\definecolor{shadecolor}{rgb}{0.969, 0.969, 0.969}\color{fgcolor}\begin{kframe}
\begin{alltt}
  \hlkwd{median}\hlstd{(x)}
\end{alltt}
\begin{verbatim}
## [1] 12
\end{verbatim}
\end{kframe}
\end{knitrout}
\end{frame}

\begin{frame}[fragile]
  \frametitle{The sample mode}
    The sample mode is equal to the most common occurence from a set of observations. It is meaningful for \ttemph{categorical data} as an indicator of the most frequent observation.

    We can find the sample mode by using the \cR function \pk{table()}:
\begin{knitrout}
\definecolor{shadecolor}{rgb}{0.969, 0.969, 0.969}\color{fgcolor}\begin{kframe}
\begin{alltt}
  \hlkwd{table}\hlstd{(x)}
\end{alltt}
\begin{verbatim}
## x
##  8 12 13 19 
##  2  1  1  1
\end{verbatim}
\end{kframe}
\end{knitrout}

    The mode is $8$ since it has the highest count.
\end{frame}


\section{Measurement reference}

\begin{frame}[fragile]
  \frametitle{Criterion-referenced measurement}
    \begin{itemize}
      \item Test scores are interpreted as an \ttemph{absolute measure} of an underlying construct
      \item If an individual reaches a particular score, the individual is seen as having mastered this construct or having fulfilled this level of proficiency
      \item A test designed in this way is said to be a \ttemph{criterion-reference measurement}
      \begin{itemize}
        \item Driver licence test
        \item Registered Nurse test
        \item Citizenship tests (\textit{Pr{\o}ve i samfunnskunnskap})
      \end{itemize}
    \end{itemize}
\end{frame}


\begin{frame}[fragile]
  \frametitle{Norm-referenced measurement}
    \begin{itemize}
      \item Test scores are seen as indicative of the level of proficiency \ttemph{with reference to} a particular population
      \item The norms have been established from previous research with individuals from the same population
      \begin{itemize}
        \item College entrance exams, SAT, ACT
        \item Graduate Record Examinations (GRE)
        \item Wechsler Intelligence Scale for Children
        \end{itemize}
    \end{itemize}
\end{frame}


\begin{frame}[fragile]
  \frametitle{The duality of reference}
    \begin{itemize}
      \item The same test can be interpreted in a criterion-referenced manner and in a norm-referenced manner
      \item  Original motive for the Binet and Simon intelligence test was to identify children with intellectual disabilities -- having nothing to do with national norms for intelligence
      \item Later the intelligence test has been used to refer to a population norm
    \end{itemize}
\end{frame}


\section{Scale scores}

\begin{frame}[fragile]
  \frametitle{Standardised scores ($Z$-scores)}
    For ease of interpretation, we can standardise raw scores $Y$ into $\N{0}{1}$:
      \[ Z =\frac{Y - \mu_Y}{\sigma_Y}. \]

    If the resulting standardised scores follow a distribution similar to the \ttemph{standard normal distribution}, we can interpret individual scores as \ttemph{percentiles} of this distribution.
\end{frame}


\begin{frame}[fragile]
  \frametitle{Linearly transformed scores}
    Wechsler Intelligence Test has mean $100$ and standard deviation $15$. This distribution can be obtained from the standard normal by a linear transformation:
      \[ S = cZ + a, \]
    where $c = 15$ and $a = 100$.
\end{frame}

\begin{frame}[fragile]
  \frametitle{Normalised scores and other non-linearly transformed scores}
    \begin{itemize}
      \item If the raw scores are not normally distributed, it is often desirable to \ttemph{rescale} the scores to approximately normal
      \item We can apply square-root or other non-linear transformations, or link the percentiles of the raw-score distribution to those from the normal distributions
      \item Criterion-referenced tests often have a threshold for certain levels of skill
    \end{itemize}
\end{frame}

\begin{frame}[fragile]
  \frametitle{Permissible operations}
    \begin{itemize}
      \item Certain statistical operations are \ttemph{inappropriate} for certain types of item or test scores
      \item If we have an \ttemph{ordinal} level of measurement we should compute its \ttemph{median} rather than the mean
      \item See Chapter 4 and 18 of McDonald (1999) for more details
    \end{itemize}
\end{frame}

\begin{frame}[fragile]
  \frametitle{Statistical tests and levels of measurement}
    \begin{itemize}
      \item A \ttemph{hypothesis test} of equal means requires approximately normally distributed scale scores ($t$-test)
      \begin{itemize}
        \item Does not require the original scale to have interval level properties
      \end{itemize}
      \item Hypothesis tests can be affected by a change of scale (see p. 60--61, McDonald (1999))
    \end{itemize}
\end{frame}

\begin{frame}[fragile]
  \frametitle{Review}
    \begin{itemize}
      \item Levels of measurement
      \item Measures of central tendency
      \item Criterion-referenced and norm-referenced testing
      \item Linear and non-linear transformations of test scores
    \end{itemize}
\end{frame}


\section{Exercises}

\begin{frame}[fragile]
  \frametitle{L4 Task 1}
    Consider a R.V. $X$ that takes values 1, 2, 3 and 4 with corresponding probabilities 0.1, 0.2, 0.4 and 0.3.
    \begin{description}
      \item[a] What is its median?
      \item[b] What is its mode?
    \end{description}
\end{frame}

\begin{frame}[fragile]
  \frametitle{L4 Task 2}
    Which of the following are symmetric distributions?
    \begin{description}
      \item[a] A $t_{49}$-distribution.
      \item[b] A $\chi_1^2$-distribution.
      \item[c] For the R.V. $X$ with PMF
    \end{description}
    \begin{center}
      \begin{tabular}{cccccc}
        \hline
        $X$ &1  &2  &3  &4  &5\\
        $\p{X}$ &0.3  &0.15  &0.1  &0.15  &0.3\\
        \hline
      \end{tabular}
    \end{center}
\end{frame}

\end{document}
