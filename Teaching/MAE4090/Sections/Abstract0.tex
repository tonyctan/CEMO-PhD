\section*{Popular Abstract}

\noindent Preparing youth for life-long success with numeracy, literacy and science capability has been the core mission for all schooling systems. The post-financial crises and post-COVID era, in addition, imposed increasing demand for financial literacy on school leavers. Although financial education programs were generally reported as effective in promoting learners' financial literacy outcome, paradoxical results of non-findings or even negative findings were not unheard of. Any claim that education efforts did not matter, or even harmful, for learners' development deserves immediate attention because if school were committing something wrong, school leaders and policy makers would want to know what, which and where the problems were so that harmful practices can be reverted into good pedagogy. Alternatively, it could instead be the instrument some researchers employed that led to such underwhelming results. A closer examination of how school effectiveness is measured would also promote methodology practices and the resultant policy advice. Using 2018 PISA financial literacy data, this study examined how students' financial literacy scores changed systematically as educational efforts, parental involvement, school safety as well as resource allocation varied. Analyses showed that all four aspects of school climate mattered greatly in explaining differences in students' financial literacy scores. Negative results reported by some papers were likely the results of certain design choices. School financial education should definitely not be withdrawn but more carefully designed with increase emphases on students' financial problem-solving skills in addition to knowledge and confidence training.
