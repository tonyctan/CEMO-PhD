\section*{Abstract}

% APA7 Rule 2.9 Abstract
% Abstracts in paragraph format are written as a single paragraph without indentation of the first line.

% APA7 Rule 2.24 Paragraph indentation
% Exception dot point 4: The first line of the abstract should be flush left.

\noindent Repeated financial crises and the current pandemic emergency all exposed the harsh consequences of financial illiteracy shared by large proportions of the general population. Although remedial plans were shown to be most effective if introduced early in life, the exact relationships among student-, family- and school-factors behind youth's financial literacy outcomes were not yet fully understood. Using the latest Programme for International Student Assessment (PISA) 2018 financial literacy data and the theoretical framework of school climate recently proposed by Wang \& Degol (2016), this study examined the mechanism for individuals' financial literacy performance in the context of their school environment. A multilevel structural equation model (MSEM) revealed that 33.5\% of the variation in students' financial literacy scores could be explained by student-level variables and 47.7\% by school-level factors for the full PISA 2018 sample. The MSEM also highlighted key roles financial knowledge and financial confidence played in mediating students' financial literacy performance. Both financial education and financial socialisation were positively associated with financial knowledge and confidence, but their direct effects on financial literacy scores were negative once the mediation effects have been accounted for. Strong contextual effects suggested the important role of school environment for facilitating individual-level effects. This study took a person-ecological approach for reconciling two strands of research efforts that focused either on students or on schools. It also confirmed the importance of school education, parental involvement, safety and educational resources for bringing about greater financial knowledge and confidence and identified potential improvement opportunity for pedagogical practices for further advancing students' financial problem-solving capabilities.

% APA7 Rule 2.10 Keywords
% Write the label "Keywords:" (in italic) one line below the abstract, indented 0.5 in. like a regular paragraph, followed by the keywords in lowercase (but capitalize proper nouns), separated by commas. The key words can be listed in any order. Do not use a period or other punctuation after the last keyword. If the keywords run onto a second line, the second line is not indented.

\textit{Keywords:} school climate, financial literacy, PISA, multilevel modelling, structural equation modelling, contextual effect