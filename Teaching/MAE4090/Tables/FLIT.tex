% Table generated by Excel2LaTeX from sheet 'Sheet1'
\MAEptable{tab:FLIT}{Structure of PISA 2018 Financial Literacy Construct}{
      \begin{tabular}{c @{\hskip 2cm} l @{\hskip 1.5cm} c}
      \toprule
            &       & Distribution of \\
            Domain$\, ^\text{a}$ & \multicolumn{1}{c}{Content areas} & score points (\%) \\
      \midrule
      Content & Money and transactions & 30--40 \\
            & Planning and managing finances & 25--35 \\
            & Risk and reward & 15--25 \\
            & Financial landscape & 10--20 \\
            &       &  \\
      Process & Identify financial information & 15--25 \\
            & Analyse information in a financial context & 15--25 \\
            & Evaluate financial issues & 25--35 \\
            & Applying financial knowledge and understanding & 25--35 \\
            &       &  \\
      Contexts & Education and work & 10--20 \\
            & Home and family & 30--40 \\
            & Individual & 35--45 \\
            & Societal & 5--15 \\
      \bottomrule
      \end{tabular}
}{This table synthesised Table 5.1 to 5.3 of \textit{PISA 2018 Assessment and Analytical Framework} \parencite[][p. 155]{PISAframework}. The PISA organiser used the term ``score points'' instead of ``items'' because partial credits can be awarded for some questions.\\
$^\text{a}$ \emph{Content} comprises the areas of knowledge and understanding that are essential in the area of literacy in question; \emph{processes} describes the mental strategies or approaches that are called upon to negotiate the material; and \emph{contexts} refers to the situations in which the knowledge, skills and understandings of the domain are applied, ranging from the personal to the global. \parencite[][pp. 130--131]{PISAframework}}