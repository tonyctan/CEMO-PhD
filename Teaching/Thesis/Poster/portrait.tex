%%%%%%%%%%%%%%%%%%%%%%%%%%%%%%%%%%%%%%%%%%%%
% https://github.com/martinhelso/uioposter %
%%%%%%%%%%%%%%%%%%%%%%%%%%%%%%%%%%%%%%%%%%%%
% Class options                            %
%%%%%%%%%%%%%%%%%%%%%%%%%%%%%%%%%%%%%%%%%%%%
% Orientation:                             %
% portrait (default), landscape            %
%                                          %
% Paper size:                              %
% a0paper (default), a1paper, a2paper,     %
% a3paper, a4paper, a5paper, a6paper       %
%                                          %
% Language:                                %
% english (default), norsk                 %
%%%%%%%%%%%%%%%%%%%%%%%%%%%%%%%%%%%%%%%%%%%%
\documentclass{uioposter}

\usepackage[absolute, overlay]{textpos}            % Figure placement
\setlength{\TPHorizModule}{\paperwidth}
\setlength{\TPVertModule}{\paperheight}

% Wrap figure
\usepackage{wrapfig}

% Table rules
\usepackage{booktabs}

% Bibliography
\usepackage{ifplatform}
\usepackage[style=apa,sortcites=true,sorting=nyt,backend=biber, useprefix=false]{biblatex}
\ifwindows
    \addbibresource{M:\\pc\\Dokumenter\\MSc\\Thesis\\Bibliography\\Master.bib}
\else
    \addbibresource{~/uio/pc/Dokumenter/MSc/Thesis/Bibliography/Master.bib}
\fi


\title{School Climate and Youth's Financial Literacy Outcomes}
\author
{%
    Candidate: Tony Tan%\inst{1}
    \and
    Supervisors: Prof Ronny Scherer%\inst{1}
    \and
    Dr Chia-Wen Chen%\inst{1}
}
%% Optional:
% \institute
% {
%     \inst{1} Centre for Educational Measurement
%     \and
%     \inst{2} Det utdanningsvitenskapelige fakultet
% }
% Or:
\institute{Centre for Educational Measurement, Faculty of Educational Sciences, University of Oslo}


%% Remove footline:
%\setbeamertemplate{footline}{}


\begin{document}

\begin{frame}

\begin{columns}[onlytextwidth]

% Begin left column
\begin{column}{0.5\textwidth - 1.5cm}

    \begin{block}{Introduction}
        \begin{wrapfigure}{r}{0.5\linewidth}
            \centering
            \includegraphics[width=0.5\textwidth]{../Figures/distribution.pdf}
        \end{wrapfigure}
        Repeated economic crises in recent times has exposed the cost of financial \emph{illiteracy}. Redress schemes are more effective if introduced early in life \parencite{lusardi:2014}. PISA has been tracking 15-year-olds' financial literacy levels since 2012 with the latest 2018 results showing sizeable differences across the globe. The aim of this study is to identify school climate variables that covary strongly with youth's financial literacy outcomes for the purpose of lending support to school leaders and policy makers in their evidence-based decision making with research questions:

        \begin{enumerate}
            \item[RQ1:] To what extent can the variation in students' financial literacy outcomes be accounted for by each of the school climate variables?
            \item[RQ2:] In particular, how do cognitive and affective pathways interact during classroom financial literacy interventions?
        \end{enumerate}
    \end{block}

    % Begin{exampleblock}{Black}
    %      Use an \structure{gray text}
    % \end{exampleblock}

    % \begin{alertblock}{How do you make it pop?}
    %     Use an \alert{alertblock}!
    % \end{alertblock}

    \begin{block}{Methods}
        PISA 2018 financial literacy data set: 20 participating countries\footnote{Brazil, Bulgaria, Canada, Chile, Estonia, Finland, Georgia, Indonesia, Italy, Latvia, Lithuania,\\the Netherlands, Peru, Poland, Portugal, Russian Federation, Serbia, Slovak Republic, Spain, USA}, 6631 schools, 107162 students

        Missing data: multilevel joint modelling \parencite{asparouhov:2010} with ten sets of imputed data merged with ten plausible values

        Multilevel SEM repeated ten times over each plausible value with results pooled in accordance with \textcite{rubin:1987} using Mplus 8.5

        School Climate Variables \parencite{wang:2016}:
        \vspace{-1cm}\small
        % Table generated by Excel2LaTeX from sheet 'Sheet1'
\begin{table}
  \centering
    \begin{tabular}{clc}
    \toprule
    Aspect of      & \multicolumn{1}{c}{Operationalisation} & Variable \\
    school climate & \multicolumn{1}{c}{from 2018 PISA data files} & label \\
    \midrule
    Academic & 931: Financial education in school lessons & \texttt{FLSCHOOL} \\
    Community & 932: Parental involvement in matters of fin lit & \texttt{FLFAMILY} \\
    Safety & 916: Stdent's experience of being bullied (reverse) & \texttt{NOBULLY} \\
    Inst env & 188: Shortage of educational material & \texttt{EDUSHORT} \\
    \bottomrule
    \end{tabular}
\end{table}

    \end{block}

\end{column}
% End left column

% Begin right column
\begin{column}{0.5\textwidth - 1.5cm}

    \begin{block}{Results}
        \vspace{-1.2cm}
        \begin{figure}
            \includegraphics[width=0.785\textwidth]{../Figures/Poster_result.pdf}
        \end{figure}
        \vspace{-0.6cm}

    \textbf{Academic} share \emph{negative} correlation with financial literacy while its effect through affective variables are positive. Effect of \textbf{family} is fully mediated by affective pathways.
    \end{block}

    \begin{block}{Discussion}
        \begin{enumerate}
            \item[RQ1:] All four school climate variables covary significantly with students' financial literacy outcomes
            \item[RQ2:] Classroom activities correlate positively with financial literacy via affective pathways, but negatively via cognitive pathway
        \end{enumerate}

        \textbf{Conclusion}: It is the joint effort from school, family, safety and resources to bring about a future-ready, financially literate generation. \textbf{Take away message}: Youth at 15 reacts best to \emph{affective} approaches to financial education. \textbf{Implications}: Pedagogically, it may not be technical curricula targeting the brain, but empowerment appealing to the heart, that would carry the most promise.
    \end{block}

    \begin{block}{References}
        % Shrink font size
        \renewcommand*{\bibfont}{\tiny}
        % Insert reference list here
        \printbibliography
    \end{block}

\end{column}
% End right column

\end{columns}

% Bottom banner
\begin{textblock}{0.5}(0.18, 0.94)
    \color{white}
    \sffamily
    \textbf{Corresponding author}
    \\
    Tony Tan\ \ \url{tctan@uio.no}\\
    PO Box 1161, Blindern, 0318 OSLO, Norway
\end{textblock}

\end{frame}

\end{document}