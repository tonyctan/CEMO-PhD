\section{Discussion}
\label{sec:5}

%\epigraph{It takes a village to raise a child.}{African proverb}

\subsection{Overview}
%//mark Brief summary
%//mark Remind readers what my research questions are
%//mark The implication of this study
``It takes a village to raise a child.'' This study looked into the dual mechanisms of how factors associated with 15-year-old students' financial literacy related to each other (RQ 1) and how the surrounding school environment may facilitate such relationships (RQ 2). MSEM results showed that 33.5\% of the variation in students' \texttt{FLIT} scores can be explained by student-level variables and 47.7\% by school-level factors (see \cref{sec:rsq}), suggesting the importance of schools in cultivating youth's financial literacy outcomes. By accounting for the hierarchical data structure, sampling weights, missing data imputation, as well as measurement error and sampling error, this study was able to ascertain the marginal effects of the four school climate variables: \textsc{academic}, \textsc{community}, \textsc{safety} and \textsc{institutional environment} \parencite{wang:2016} respectively (see \cref{fig:results} and \cref{tab:est1}). This study added empirical evidence to \poscite{kutsyuruba:2015} review article by showing the importance of school safety for students' financial knowledge, confidence, and application behaviour. The student-level model extended \poscite{jorgensen:2010} structural equation approach to financial literacy and confirmed the key roles financial knowledge ($R^2 = .136$) and confidence ($R^2 = .077$) played in mediating youth's financial literacy achievement.

This study also revealed a key insight that was initially less intuitive. At both individual- and school-levels, the associations between explicit teaching of finance-related topics (FEdu) and contemporaneous financial literacy performance (FB) were found to be \emph{negative}. In addition, the relationships between parental involvement (FSoc) for cultivating youth's financial literacy outcomes were shown to be positive along the mediation pathways (via FC and FA) but negative along the application pathway (FB). These two effects were similar in size but opposite in sign. At the school-level, both classroom activities and parental care, on average, tended to be more visible around students who were yet to demonstrate their mastery of financial capabilities. Sizeable contextual effects further suggested schools rather than learners as the source of the observed negative correlations between financial literacy outcome (FB) and teaching efforts (FEdu), and between FB and financial socialisation (FSoc).

\subsection{Responses to the Research Questions and Hypotheses}

\subsubsection{Research Question 1}

All four school climate variables explained variation in youth's financial literacy outcomes. Financial knowledge (FC) and confidence (FA) played significant mediation roles for explaining financial literacy scores (FB), confirming Hypotheses 12 and 13. This result was partially consistent with \poscite{jorgensen:2010} mediation model in which the author focused solely on the relationships between parental influence and financial behaviour, mediated by financial knowledge and attitude. This study effectively corrected \poscite{jorgensen:2010} omitted variable problem by adding back FEdu, Safety and more demographic controls at $L1$ and an additional structure at $L2$, subsequently re-establishing FC as a significant mediator. Such result was fully expected under the family financial socialisation theory \parencite{danes:2007} where financial knowledge development shall be an important component.

Financial education (FEdu) showed positive effects along the mediation pathways (confirming H1, H2) but a negative effect along the direct pathway (contradicting H3) with financial literacy scores (FB). Since the direct effect overshadowed the mediation effects, the total effect between FEdu and FB appeared to be negative. This result positioned the current study in line with a series of papers reporting non-significant or negative findings. Studies using the test-retest design \parencite{mandell:2009}, randomised experiment with treatment-control groups \parencite{becchetti:2013, collins:2013} as well as an archival study using PISA 2012 data \parencite{farinella:2017} all questioned the effectiveness of financial education courses. Additionally, \poscite{mountain:2020} 5-year-horizon longitudinal study identified a negative association between long-term financial behaviours and attending workshops and seminars, mediated by financial knowledge. In light of these publications, the negative direct pathway identified by the current study shall not be dismissed as an statistical irregularity but an invitation for further considerations (see \nameref{sec:conj} below).

Similar to FEdu, parental involvement at home (FSoc) had positive mediation pathways (confirming H4 and H5) but an equi-magnitude negative direct pathway (contradicting H6), leading to a non-significant total relationship between FSoc and FB. This result shall be differentiated from the positive FSoc-FB association by \textcite{morenoherrero:2018a} since the latter design did not involve FEdu, Safety or any mediators, leading to a possible redistribution of explanatory power from the omitted variables into FSoc.

Safety was found to have positive effects for students' financial knowledge, confidence, as well as application behaviour (confirming H7, H8 and H9), linking \poscite{kutsyuruba:2015} school safety review to the financial literacy research.

\subsubsection{Research Question 2}

All four school climate variables at the school-level were shown to be statistically significant for explaining the variation in school-average financial literacy scores. MSEM results revealed that educational resource shortages as well as high student-teacher ratios both correlated with lower average financial literacy performance, confirming H10 and H11 and the applicability of prior studies \parencite{finn:1999, miles:1998, uline:2008} to the field of financial literacy research.

Adding to existing literature, FEdu, FSoc and Safety were all shown to have significant contextual effects, suggesting individual students' financial literacy capability was strongly affected by their school environment. Along with the higher $R^2$ observed at $L2$ (see \cref{sec:rsq}), and the strong design effect calculated in \cref{eqn:design}, the current study consistently highlighted school-level factors as the driving force behind the systematic variations in students' PISA 2018 financial literacy performance.

\subsection{Conjectures about Negative Pathways}\label{sec:conj}

Although causal inferences could not be established from a correlational study design, a negative association between input and output variables may still \emph{suggest} some interesting possibilities for future studies. If one hypothesises a causal direction \texttt{FLSCHOOL} $\longrightarrow$ \texttt{FLIT}, the negative relationship between the two variables could signal potential improvement opportunities for current financial education practices. While students have benefited from educational interventions with growing knowledge and confidence, existing pedagogy may yet to explicitly train students to link their learning to real-world finance problem-solving. Bridging the disconnect between minds and hands has long been emphasised in science \parencite{harlen:1999} and mathematics \parencite{smith:1996} education and voices for learning from sister subjects' success started to grow in the field of financial education \parencite{marleypayne:2021}. Parents may similarly adapt by introducing financial problem-solving skills in addition to sharing knowledge and affects at home. Alternatively, a causal direction \texttt{FLSCHOOL} $\longleftarrow$ \texttt{FLIT} may suggest that educational and parental attention was being directed preferentially towards students who were most in need of developing problem-solving skills---it was not the quality of interventional efforts but the insufficient quantity that needed to be addressed. Future research may investigate the plausibility of such constraint optimisation behaviour by teachers and parents and estimate the sizes of the Lagrange multipliers as evidence for the potential marginal improvement should schooling and parenting resources were expanded. A third possibility involves a hidden confounder \texttt{FLSCHOOL} $\longleftarrow$ confound $\longrightarrow$ \texttt{FLIT}. \poscite{jappelli:2010} observation that students' financial literacy tended to be lower in countries with stronger social safety net could serve as a starting point for this line of investigation under the reasonable assumption that such countries also devote higher social resources into education input. Should this direction of study become fruitful, financial educators would then be reminded the importance of social arrangement as a moderator, where it would be desirable to re-allocate educational resources taking into account each society's social contracts.

A non-linear relationship could be a fourth possibility for the negative association between \texttt{FLSCHOOL} and \texttt{FLIT}. Using 2015 TIMSS data, \textcite{teig:2018} demonstrated a curvilinear relationship between inquiry-based teaching practice and students' science achievement with high frequency inquiry-based teaching being linked to a reduced performance. A quadratic relationship was reported between learning time and science achievement using PISA 2015 data \parencite{zhang:2021} especially in Eastern cultures, possibly indicating that non-linearity could become a relative common consideration when analysing large-scale international assessment data. A verification of similar curvilinear relationship in the financial literacy filed is important so that educational and parental resources can be further optimised.

A final hypothesis can be made based on the implementation lags observed by \textcite{bernheim:2001}. Financial literacy could be unique in a sense that it requires a longer time for FEdu and FSoc to be consolidated, incorporated and then turned into observable behaviour improvement, including application and problem-solving behaviour. That is to say the negative relationship between $\texttt{FLSCHOOL}_t$ and $\texttt{FLIT}_{t-1}$ reflected the maturing effect of financial skill acquisition process. A longitudinal study is required in order to confirm this intertemporal growth model.

\subsection{Limitations}
%//mark Word in positive form

The correlational research design used by this study limited the possible causal inferences. Using \poscite{shadish:2002} taxonomy, this study demonstrated strong statistical conclusion validity by showing both the presence and strength of the covariation between school climate variables and students' financial literacy outcomes. It was unable to, however, demonstrate whether school climate preceded financial literacy in time, neither was it able to exclude all other relationships as plausible explanations for the covariation between the two. By this measure, the current study's internal validity is not yet strong. As the scholarly world is yet to reach consensus on the best construct to represent financial literacy, this study inherited one particular version of financial literacy operationalised by the PISA organiser, whose construct validity continues to attract scrutiny by both theorists and practitioners \parencite{schuhen:2014}. Lastly, statistical parameters derived in this study were based on data drawn from predominantly industrialised countries, questioning its strength on external validity.

The other limitation originated from the data design. Since this study pooled all 20 participating countries into a global data structure, the subsequent analyses and statistical results must be interpreted as the global, rather than country-specific outcomes. This observation is important for education policy making since global averages may not serve the interests of local conditions correctly. Since industrialised economies were over-represented in the 20-country sample, pedagogical and policy implications may be skewed towards countries with similar socio-economic profiles. Further studies are encouraged to replicate procedures employed by this project by counties in order to obtain evidence better situated with the local environment.

Based on the limitations discussed above, future research efforts may consider upgrading the study design from a correlational to a causal one by using, amongst others, instrumental variable \parencite{pokropek:2016} or panel data \parencite{salasvelasco:2019} techniques. Country-by-country comparisons would also provide additional insight into the similarities and differences across economies, aiding pedagogy design and education policy formation processes.

\subsection{Contribution and Conclusions}

This research project contribute to financial literacy literature in a number of ways. It first of all linked a substantive theoretical framework of school climate to youth's financial literacy development process in order to examine how individuals' capability is formed \emph{in the context of} their school environment. This person-ecological approach reconciled two strands of research efforts that focused either on students or on schools into one unified structure. In terms of methodology, this study attempted a recent development in the MSEM literature using a multilevel latent covariate approach (MLC, \textcite{ludtke:2008}; ``doubly-latent model'', \textcite{marsh:2009}) to correct for unreliability at higher-level when lower-level constructs were aggregated up. The successful application of this new technique to the most recent PISA 2018 data set showcased the advancement in the field of educational measurement.

%//mark Bird-eye view
%//mark What conclusion I can draw from this paper/study?

A well-functioning society relies on citizens' financial literacy for the betterment of their own well-being and that of the collective. Policy-makers, school leaders, teachers and parents all have progressively come to terms with the cost of neglect and demanded evidence-based action plans. The current research project answered this call by exploring four aspects of school climate using the latest international large-scale assessment data---Education matters. Parenting matters. Safety and resource fundings do matter. These conclusions shed light to the policy priorities that can be actioned upon without delay. This study served only as a starting point for a vibrant scholarly conversation about better preparing our young for an ever-challenging future. May they benefit and succeed.